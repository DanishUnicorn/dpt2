\chapter{Literature résumés}
\setlength{\headheight}{12.71342pt}
\addtolength{\topmargin}{-0.71342pt}

This section of the course notes is designed to streamline access to the key findings from each reading material (RM), providing a concise and accessible overview of essential information. Created through experimentation with various AI platforms, this chapter also serves to enhance my prompt engineering skills, exploring diverse methods of note-taking for maximum efficiency and clarity. The procedures for creating these summaries have varied, but all methods share a common approach: each RM has been fully read, with summaries and notes prepared after completing each respective subsection. By using these AI-co-op'ed approaches, these notes aim to be both a reliable reference and a resource for continuous improvement in capturing complex concepts.

\section{RM 01: BCPP - Chapter 01}
\textbf{Ylva Ardö and E. Waagner Nielsen, 2017, Department of Food Science University of Copenhagen, Denmark}

\vspace{1em}
\textbf{Summary}

Cheese is produced as a concentrate of milk nutrients through the coagulation of milk by rennet enzymes or acidification. The resulting coagulum undergoes syneresis, which is the contraction of the cheese curd and the expelling of whey consisting of water and water-soluble milk compounds. Syneresis is accelerated by cutting the gel, heating, and stirring to produce firm curd grains that may be pressed and formed into various shapes. Preservation is maintained through low moisture content, acidification by lactic acid bacteria, and salting. Spoilage by moulds on the surface is prevented by drying, cleaning, oxygen exclusion through brine or coatings, or by stimulating a controlled surface microflora.

\vspace{1em}
Cheese milk treatment involves keeping milk at low temperatures to prevent the growth of detrimental bacteria. Pasteurization at 72\textdegree C for 15 seconds typically kills about 99\% of raw milk bacteria and inactivates 95\% of milk lipase, which contributes to why raw milk cheeses often develop a richer flavour. More intense heating denatures whey proteins, causing them to precipitate with caseins and increase total protein retention, though it may result in slower renneting and a weaker coagulum. Spore-forming bacteria like \textit{Clostridium tyrobutyricum} are removed via bactofugation, which typically removes more than 90\% of spores, or microfiltration, which can remove more than 99.9\% from the skim milk part. Ultrafiltration concentrates casein micelles and whey proteins into a retentate, allowing for higher cheese yields since whey proteins are retained.

\vspace{1em}
In renneted cheese production, starter cultures produce lactic acid from lactose. Mesophilic starters are used when the curd is heated up to approximately 40\textdegree C, while thermophilic cultures are used for cheeses scalded at higher temperatures. Following acidification, syneresis is controlled through specific moulding methods: 
\begin{enumerate} 
    \item Curd grains are pressed under whey before moulding to avoid air between the cheese grains (round eyed cheeses). 
    \item Curd grains are pressed under whey before moulding and then boiled in whey (Halloumi cheese). 
    \item Grains are separated from the whey before moulding so that air is trapped in the curd to create an open texture (Havarti). 
    \item Grains are separated from the whey and the curd is left for continued acidification (cheddaring) and then milled, salted and moulded (Cheddar). 
    \item Concentrated retentate from ultrafiltration is cast directly in moulds or packaging (white salad cheese). 
    \item After acidification to pH 4.6, the whey is separated from the precipitated curd (Quarg, Skyr). 
    \item Grains of acid precipitated coagulum are separated from the whey and packed as loose grains (Cottage cheese). 
\end{enumerate}

\vspace{1em}
Salting is essential for preservation, taste, and consistency, with most semi-hard cheeses requiring 1 to 2\% NaCl. Salting occurs by adding salt to the curd before moulding or by diffusion into the cheese through immersion in 20 to 22\% NaCl brine at 10 to 15\textdegree C. During ripening, which can last weeks or years, casein, fat, lactose, and citrate are catabolised. Proteolysis breakdown significantly influences structure and produces peptides and amino acids that contribute to the background flavour, while lipolysis hydrolyzes milk fat. Final finishing includes vacuum packing, waxing, or wrapping in foil to prevent moisture loss.


\section{RM 02: BCPP - Chapter 06}
\textbf{Ylva Ardö and E. Waagner Nielsen, 2017}

\vspace{1em}
\textbf{Summary}

The recovery of milk solids and the resulting yield of cheese are fundamentally determined by the gross composition of the milk source and the efficiency of the coagulation process. When milk is coagulated by rennet or acid, caseins form a three-dimensional network that encloses other constituents. As the coagulum shrinks during syneresis, water and soluble compounds are squeezed out as whey, while larger particles like fat globules and bacteria are retained.

\vspace{1em}
Retention of protein is primarily focused on caseins, which constitute 75 to 80\% of milk proteins and are defined by their insolubility at pH 4.6. In rennet-induced coagulation at pH 6.6, the enzyme catalyzes the hydrolysis of $\kappa$-casein into para-$\kappa$-casein and glyco-macro-peptide (GMP). While para-$\kappa$-casein remains in the micelles, the soluble GMP follows the whey. Typical protein retention in rennet cheese is 76\%, though actual values range from 75 to 78\%. Total retention of nitrogen compounds is influenced by several factors: 
\begin{enumerate}
    \item Heat treatment of milk: This causes denaturation of whey proteins, which then precipitate with caseins to increase retention.
    \item Ultrafiltration: This method concentrates casein micelles and whey proteins, leading to higher cheese yields.
    \item Biological quality: Inflammation of the udder (mastitis) or the growth of proteolytic psychrotrophic bacteria can decrease protein retention.
    \item Fines in the whey: Small particles of precipitated proteins lost during cutting represent a loss of para-casein. 
\end{enumerate}
    
\vspace{1em}
The concept of excluded water is critical for calculating retention; approximately 2.6 g of water per g of cheese protein is unavailable as a solvent for whey proteins, and 0.3 to 0.5 g per g of protein is unavailable for lactose. Retention rates for other milk solids are as follows: \begin{enumerate}
    \item Fat and bacteria: Retention is high, typically between 85 and 95\%, depending on the size of the grains and homogenization.
    \item Lactose: Retention is low (3 to 5\%) as it is dissolved in the water phase.
    \item Ash compounds: Hard and semi-hard rennet cheeses retain 35 to 40\% of ash, whereas acid-precipitated cheeses retain only 10 to 15\% because minerals like calcium and phosphate dissolve as pH decreases. 
    \item Citrate: Retention is approximately 10\%, depending on the moisture content and acidification during syneresis. 
\end{enumerate}

\vspace{1em}
The calculation of cheese yield and composition is based on these retention coefficients applied to the standardized milk solids. For a semi-hard cheese like Danbo or Gouda, the total milk solids in the curd are combined with added NaCl to determine the final dry matter and moisture content. The final weight of the cheese may decrease during ripening due to evaporation unless moisture loss is prevented by waxing or plastic film.


\section{RM 03: BCCP - Chapter 07}
\textbf{Ylva Ardö and E. Waagner Nielsen, 2017}

\vspace{1em}
\textbf{Summary}

The world production of cheese comprises thousands of different varieties that are classified and grouped through various methods, most commonly by fat content, firmness, or processing parameters such as the method of coagulation. Motivation for classification lies in international trading, the protection of names and trademarks, and the standardisation of cheese. The International Dairy Federation (IDF), founded in 1903, developed a vast number of codes and standards that were agreed upon by European governments at the Stresa Convention in 1951. This convention established two levels of protection for cheese names: 
\begin{enumerate} 
    \item Annex A: Protection of four names to be used only on specific cheeses produced in their original countries (Roquefort, Gorgonzola, Parmigiano-Reggiano, and Pecorino Romano). 
    \item Annex B: Mutual permission to use 30 cheese names on domestic and international markets, including Danablu, Danbo, Havarti, Samsø, Gouda, Edam, and Camembert. 
\end{enumerate} 

\vspace{1em}
As a result of this convention, Danish cheeses were renamed from their original descriptors; for example, Danish Schweizer became Samsø and Steppeost became Danbo.

\vspace{1em}
Codex Alimentarius establishes generic cheese standards agreed in international cooperation and administered by the World Trade Organisation (WTO). Standards are defined for 16 cheese types classified according to principal ripening procedure (ripened, mould-ripened, in brine, or unripened) and firmness. Firmness is classified by the moisture in non-fat substance (MNFS\%): 
\begin{enumerate}
    \item soft cheese: > 67\%.
    \item firm cheese: 54-69\%.
    \item semi-soft cheese: 61-69\%.
    \item semi-hard cheese: 54-63\%.
    \item hard cheese: 49-56\%.
    \item extra hard cheese: < 51\%. 
\end{enumerate} 

\vspace{1em}
The classes of semi-soft and semi-hard were merged in recent updates because production variations often cause cheese varieties to fall into both categories.
Traditional cheeses are further protected through geographical designations adopted by the European Commission in 1993, which recognise the specific heritage of a particular region. There are three levels of protection: 
\begin{enumerate}
    \item PDO (Protected Designation of Origin): Foodstuffs which are produced, processed, and prepared in a given geographical area using recognised know-how.
    \item PGI (Protected Geographical Indication): Products where the geographical area has an impact on at least one of the stages of production, processing, or preparation.
    \item TSG (Traditional Speciality Guaranteed): Highlights traditional character in composition or means of production without referring to geographical origin. 
\end{enumerate} 

\vspace{1em}
Additionally, companies or organisations may use registered trademarks, which are national or European brands where the owner decides on branding and marketing programs.


\section{RM 04: Technology of Cheese Making; Chapter 01}
\textbf{Edited by Barry A. Law and A.Y. Tamime, 2010}

\vspace{1em}
\textbf{Summary}

World production of milk in 2008 is estimated at $\approx$ 576 $\times$ $10^6$ tonnes, with cows' milk accounting for $\approx$84.0\% of the total. Milk quality for cheese manufacture is defined as its suitability for conversion into cheese to deliver cheese of the desired quality and yield. Milk consists of protein, lipid, lactose, minerals, minor components, and water. The casein fraction coexists with insoluble minerals as a calcium phosphate-casein complex, while the water and its soluble constituents are referred to as serum. Casein is heterogeneous, comprising four main types—$\alpha_{s1}$ ,$\alpha_{s2}$,$\beta$ and $\kappa$—which represent $\approx$38, 10, 35 and 15 g 100 $\text{g}^{-1}$ of the total casein, respectively. Individual caseins vary in their calcium-binding properties and sensitivity to calcium precipitation.

\vspace{1em}
Different models have been proposed for the structure of the casein micelle based on the location of individual caseins and calcium phosphate: 
\begin{enumerate} 
    \item A sub-micelle model in which sub-micelles are 'cemented' together by colloidal calcium phosphate (CCP). 
    \item A dual bonding model where the interior is composed of $\alpha_{s}$ - and $\beta$-caseins forming a lattice through interactions between hydrophobic regions and hydrophilic phosphoserine clusters. 
    \item A tangled, cross-linked web model comprising rheomorphic casein chains cross-linked by calcium phosphate nanoclusters. 
    \item An interlocked lattice model featuring anchoring calcium phosphate nanoclusters that bind the phosphoserine domains of $\alpha_{s}$ - and $\beta$-caseins. 
\end{enumerate}

\vspace{1em}
The gelation of milk may be induced by selective hydrolysis of $\kappa$-casein at the phenylalanine $10^5$ -methionine $10^6$ peptide bond by rennets, by acidification to a pH close to 4.6, or by a combination of acid and heat. Rennet-induced gelation involves the liberation of caseinomacropeptide into the milk serum and a reduction in the negative surface charge to $\approx$-10 mV. Acid-induced gelation occurs as forces promoting dispersion are overtaken by reductions in negative charge and hydration. Milk quality requirements are defined by safety, compositional, microbiological, sensory, and ethical criteria. Factors affecting quality include composition, microbiology, somatic cell count (SCC), enzymatic activity, and chemical residues. Higher values for casein number, total casein, and calcium are positively correlated with enhanced rennet coagulation properties and cheese yield. The BB genotypes of $\kappa$-casein and $\beta$-lactoglobulin are associated with higher concentrations of casein and superior rennet coagulation properties.

\vspace{1em}
Increasing SCC reduces lactose, fat, and casein contents in milk and results in a marked increase in $\gamma$-caseins due to the hydrolysis of $\beta$- and $\alpha_{s2}$-caseins by plasmin. Native milk contains proteinases from several sources, including indigenous plasmin, lysosomal proteinases of somatic cells, and bacterial proteinases. Excessive proteolytic activity is undesirable as it hydrolyses caseins to water-soluble peptides that are lost in whey. Lipolysis in milk is broadly classified into induced lipolysis, promoted by mechanical damage and temperature alterations, and spontaneous lipolysis.

\vspace{1em}
The key elements of good milk production management are outlined as: 
\begin{enumerate} 
    \item Breeding and selecting for target cheesemaking properties. 
    \item Maintaining a high plane of animal nutrition. 
    \item Minimising bacterial count and SCC in milk. 
    \item Minimising enzymatic activity associated with somatic cells and contaminating bacteria. 
    \item Minimising chemical residues, contaminants, fat damage, and levels of free fatty acids. 
\end{enumerate}


\section{RM 05: Technology of Cheese Making; Chapter 02}
\textbf{M. Johnson and B.A. Law, 2010}

\vspace{1em}
\textbf{Summary}

Cheesemaking is a concentration process that has transitioned from a cottage industry to a high-technology fermentation sector. The global market in 2008 was dominated by the European Union and the United States, which together represented approximately 80\% of world consumption. The primary goals of cheese technology are to establish parameters for desirable flavour, body, and texture, and to develop protocols that routinely reproduce these attributes. The physical and rheological characteristics of the product are governed by interactions between casein molecules, which are influenced by pH, the dissolution of colloidal calcium phosphate (CCP), proteolysis, temperature, and cheese composition.

\vspace{1em}
Milk coagulation is achieved through three primary methods depending on the variety of cheese being produced: 
\begin{enumerate} 
    \item The addition of coagulating enzymes, such as rennets, which destabilise the micelle suspension to form a gel network. 
    \item Acid-induced gelation, where low pH (4.6) reduces repulsive charges between micelles to cause aggregation, used for cottage and cream cheeses. 
    \item Acid-heat precipitation, which uses both acid and high heat to precipitate both casein and whey proteins for varieties such as Ricotta. 
\end{enumerate}

\vspace{1em}
The manufacturing process proceeds through several defined stages to control product consistency. Standardisation of milk composition is necessary to achieve target fat-in-dry matter (FDM) and yield, typically by adjusting the casein-to-fat (C/F) ratio or adding milk solids. After standardisation, milk usually receives a heat treatment, such as pasteurisation at 72\textdegree C for 15 s, which eliminates pathogens while generally requiring no modifications to the manufacturing protocol beyond the potential addition of calcium chloride. Starter cultures are then added to convert lactose to lactic acid, which increases the rate of coagulant activity and syneresis while causing CCP dissolution.

\vspace{1em}
The subsequent stages involve structural and chemical manipulation of the curd: 
\begin{enumerate} 
    \item Coagulation and cutting: The coagulum is cut into small pieces (grains or curd) to facilitate whey removal. Cutting a "fine" or soft coagulum promotes rapid shrinking and "healing" of the curd surface, whereas a "coarse" or firm coagulum heals more slowly and is more prone to breakage. 
    \item Stirring, heating, and syneresis: The combination of mechanical stirring, heating, and acid development causes the casein network to tighten and expel moisture. 
    \item Whey removal and salting: Methods vary by variety, including draining through moulds, "cheddaring" (matting and stacking curd slabs), or the "stirred curd" process. Salt is added via direct addition to the curd, dry surface salting, or brine immersion. 
    \item Pressing: This operation forms the desired shape, forces out remaining whey, and ensures curd particles knit together into a cohesive mass. 
\end{enumerate}

\vspace{1em}
The final ripening or maturation stage involves chemical and enzymatic reactions mediated by the starter culture and adventitious microflora that transform the bland curd into finished cheese. Modern whey technology utilizes membrane filtration—including ultrafiltration (UF), reverse osmosis (RO), nanofiltration (NF), and microfiltration (MF)—to recover functional proteins and lactose.


\section{RM 06: Milk Protein Gels; Chapter 16 - Milk Proteins: From Expression to Food}
\textbf{John A. Lucey, Wisconsin Center for Dairy Research, University of Wisconsin-Madison, 2020}

\vspace{1em}
\textbf{Summary}

The gelation of milk proteins forms the structural basis for cheese and fermented products, achieved through heat, rennet enzymes, or acidification. Casein micelles, the fundamental building blocks for rennet and yogurt gels, are assembled via hydrophobic interactions and the formation of calcium phosphate nanoclusters across phosphoserine clusters. These micelles are protected by $\kappa$-casein, which features a hydrophilic C-terminal "hairy" layer providing steric stabilization and a barrier against association. Gelation methods, including heat treatment, impact the microstructure and digestion rates of the resulting milk gels.

\vspace{1em}
Rennet-induced coagulation proceeds in two overlapping stages. In the primary enzymatic phase, the C-terminal part of the $\kappa$-casein molecule is hydrolyzed, diffusing into the serum phase as caseinomacropeptide (CMP) or glycomacropeptide (GMP). This proteolysis typically obeys first-order kinetics and leads to a $\approx$ 50\% reduction in zeta ($\zeta$)potential, which decreases electrostatic repulsion between micelles. Aggregation begins during the secondary phase once the removal of protective $\kappa$-casein hairs reaches approximately 70\%, though true system-spanning network formation requires at least $\approx$ 87\% hydrolysis. These gels are viscoelastic and are characterized by: 
\begin{enumerate} 
    \item The elastic or storage modulus (G'), measuring stored energy per oscillation. 
    \item The viscous or loss modulus (G''), measuring energy dissipated as heat. 
    \item The loss tangent, representing the ratio of viscous to elastic properties. 
\end{enumerate}

\vspace{1em}
Syneresis in rennet-induced gels is a dehydration process where moisture is lost through the contraction of the casein network. This one-dimensional shrinkage is governed by Darcy's equation, written as equation \ref{eq:rm_05_darcys_equation}:
\begin{equation}
    v=\frac{B}{\eta} \times \frac{p}{x}
    \label{eq:rm_05_darcys_equation}
\end{equation}
where the superficial flow velocity (v) is determined by: 
\begin{enumerate} 
    \item B: the permeability coefficient. 
    \item $\eta$: the viscosity of the liquid. 
    \item p: the pressure acting on the liquid. 
    \item x: the distance over which the liquid flows. 
\end{enumerate} 

\vspace{1em}
In commercial practice, cheesemakers enhance syneresis by cutting curd into smaller grains, increasing stirring speeds, or raising the cooking temperature.

\vspace{1em}
Acid-induced gels, such as yogurt, form as the pH of milk is reduced toward the isoelectric point ($\approx$ 4.6), causing the dissolution of colloidal calcium phosphate (CCP). This process involves three distinct pH regions: \begin{enumerate} \item pH 6.7 to $\approx$ 6.0: Reduction in net negative charge with minimal CCP dissolution. \item pH $\approx$ 6.0 to $\approx$ 5.0: Shrinkage of $\kappa$-casein hairs and complete dissolution of CCP. \item pH < 5.0: Declining net negative charge and increased hydrophobic interactions. \end{enumerate} Factors influencing yogurt texture include fortification levels, fat content, homogenization conditions, and the specific starter culture used.

\vspace{1em}
Whey protein gels are formed by heating protein solutions (usually $\geq$ 6\%) to induce irreversible denaturation and unfolding, which exposes hydrophobic residues. The resulting network type depends on the balance of attractive and repulsive forces; fine-stranded transparent gels form at neutral pH and low ionic strength, while opaque particulate gels form at high ionic strength or pH values near 5. Properties are further influenced by the ratio of $\beta$-lactoglobulin to $\alpha$-lactalbumin and the concentration of divalent cations like $\text{Ca}^{2+}$. Mixed gels, combining rennet and acid action, undergo complex rearrangements and demineralization that often result in a higher final storage modulus than acid-only gels.


\section{RM 07: Technology of Cheesemaking; Chapter 10.4 - Cheese Structure and Eye Formation}
\textbf{Edited by Barry A. Law and A.Y. Tamime, 2010}

\vspace{1em}
\textbf{Summary}

Cheese structure and eye formation in Swiss-type varieties are fundamentally dependent on a close-knit texture and appropriate cheese cohesion. Cohesion is determined by specific mechanical properties, including elongational viscosity and fracture stress, which are influenced by the chemical composition of the matrix and proteolysis changes during ripening. A crucial factor in this development is the pH at one day, which affects the structural state of the protein; eye development is specifically promoted within a pH range of 5.15-5.45. Consistency that is too soft (associated with pH below this range) results in blind cheese, while consistency that is too tough (above this range) increases the risk of fracture and slit formation.

\vspace{1em}
The study of eye formation involves various technological and rheological measurements: 
\begin{enumerate} 
    \item X-ray tomography and transmission to monitor the growth of identified holes. 
    \item Scanning electron microscopy (SEM) to observe the nucleation of eyes and isotropic deformation of curd particles. 
    \item Nuclear magnetic resonance (NMR) imaging for the high-resolution detection of eye features, air pockets, and micro-bubbles. 
    \item Compression and relaxation testing to evaluate fracture stress, fracture strain, and elastic properties at ripening temperatures. 
    \item Ultrasonic techniques for non-destructive monitoring of the gas-solid structure of the cheese. 
\end{enumerate}

\vspace{1em}
Proteolysis is essential for texture development, with the presence of native caseins in the matrix contributing to the firmness and deformability necessary for openness. Intensified proteolysis, particularly reflected by higher levels of phosphotungstic-acid soluble nitrogen (PTASN), has been significantly correlated with the intensity of slit defects in varieties like Comté. Furthermore, the fat content affects mechanical resistance; higher fat levels increase elasticity and decrease resistance, though excessive fat may reduce propionic fermentation. During the warm-room ripening stage, the cheese paste reaches its lowest firmness and highest deformability, providing the most appropriate mechanical conditions for eye formation.

\vspace{1em}
Slit development, which occurs primarily during the cold storage stage, is a consequence of increased internal pressure combined with low fracture resistance. A slit forms instead of a round eye if the local overpressure is higher than the local cheese fracture stress. Fracture patterns in slits often exhibit concentric folds and radial tracks, and these defects are typically oriented perpendicular to the pressing axis. Factors driving this late-stage defect include secondary fermentation by specific propionibacteria strains and changes in carbon dioxide solubility and matrix rigidity induced by the temperature decrease. In summary, successful eye formation requires the precise synchronization of gas production with the optimal mechanical and biochemical properties of the cheese matrix.


\section{RM 08: Technology of Cheesemaking; Chapter 11.7 - Cheese Defects}
\textbf{Edited by Barry A. Law and A.Y. Tamime, 2010}

\vspace{1em}
\textbf{Summary}

Microbiological contamination is a major cause of quality defects in cheese, which may originate from raw milk, other ingredients, or post-heat-process sources within the dairy. Because fresh curd may be matured for several weeks, months, or years, spoilage problems often manifest long after the cheese was produced. Early blowing caused by gas production from coliforms is now rare due to pasteurization and effective post-pasteurization control. While bacteriophage attack has been considered a problem of the past, its reemergence is a concern in artisan and farmhouse environments characterized by poor management of airflow or extended propagation of starter mother cultures. The inhibition of citrate-fermenting starter cultures by bacteriophages or antibiotics allows residual levels of citric acid to remain in the curd; heterofermentative non-starter lactic acid bacteria (NSLAB) are then capable of fermenting citrate to produce carbon dioxide, resulting in gassing, splits, and fissures.

\vspace{1em}
The involvement of yeasts in cheese maturation and spoilage is complex, with dominant species belonging to the genera \textit{Kluyveromyces} and \textit{Debaryomyces}. Excessive growth is implicated in defects such as softening, early blowing of Parmesan, and various forms of discoloration or slime formation. While a level of $10^4$ cfu $\text{g}^{-1}$ may be considered normal in traditionally produced hard cheeses with a natural rind, an initial level of $10^2$ cfu $\text{g}^{-1}$ or below in vacuum-packaged blocks can cause loosening of the cheese bags during maturation due to the production of metabolic carbon dioxide.

\vspace{1em}
Mould growth is the primary cause of defects in modern cheese production. Cheese possesses a specific associated mycoflora, and identification of the species present is an essential prerequisite for the control of problems. In a study of hard, semi-hard, and semi-soft cheeses, 91\% of fungal isolates were \textit{Penicillium} spp., with \textit{P. commune} occurring in 42\% of samples. The specific associated mycoflora and related defects include: 
\begin{enumerate} 
    \item \textit{Penicillium commune}: The wild type of the white Camembert mould and the most frequent contaminant. 
    \item \textit{Penicillium roqueforti}: Used in blue cheese manufacture but undesirable as a cross-contaminant. 
    \item \textit{Cladosporium} spp.: Common where cheese surfaces have become damp. 
    \item \textit{Phoma}-type moulds: Associated with ‘thread-mould’ defect in vacuum-packed block Cheddar. 
    \item \textit{Aspergillus versicolor}: Found on some Dutch cheeses. 
    \item \textit{Penicillium discolor}: Significantly resistant to the antifungal agent natamycin. 
    \item \textit{Penicillium verrucosum}: The only potential ochratoxin A producer found in the cheese environment. 
\end{enumerate}

\vspace{1em}
Analysis of cheese has failed to demonstrate significant or persistent quantities of mycotoxins beneath the rind of correctly manufactured cheese without fissures. Spoilage moulds do not produce the antibiotic penicillin; significant quantities are only produced by \textit{Penicillium chrysogenum}, which is rare as a food spoilage agent and is not used in mould fermentations.


\section{RM 09: Fundamentals of Cheese Science; Chapter 07}
\textbf{Patrick F. Fox, Timothy P. Guinee, Timothy M. Cogan, Paul L. H. McSweeney, 2017}

\vspace{1em}
\textbf{Summary}

The enzymatic (rennet-induced) coagulation of milk is a two-stage process consisting of a primary enzymatic phase and a secondary non-enzymatic phase. The primary phase involves the specific hydrolysis of $\kappa$-casein at the $\text{Phe}_{105}-\text{Met}_{106}$ bond. This action destroys the micelle-stabilizing properties of $\kappa$-casein, producing para-$\kappa$-casein, which remains with the micelle, and caseinomacropeptide (CMP), which is released into the serum. The secondary phase involves the aggregation of these rennet-altered (para-casein) micelles into a three-dimensional gel network in the presence of $\text{Ca}^{2+}$ at temperatures above 20\textdegree C.

\textbf{Summary}
Hydrolysis of $\kappa$-casein is influenced by several environmental factors: 
\begin{enumerate} 
    \item pH: The optimum pH for the first stage in milk is $\approx$ 6.0 at 30\textdegree C. 
    \item Temperature: Coagulation has a broad minimum at 40--45\textdegree C, but is typically performed at 31\textdegree C to optimize starter growth. 
    \item Heat Treatment: Severe heating (> 72\textdegree C) results in denatured $\beta$-lactoglobulin complexing with $\kappa$-casein, which inhibits both the primary and secondary phases. 
\end{enumerate}

\vspace{1em}
The secondary phase occurs when the zeta potential of the micelles is reduced from -10/-20 to -5/-7 mV and approximately 85\% of the total $\kappa$-casein has been hydrolyzed. Aggregation is driven by calcium bridges and hydrophobic interactions and is highly temperature-dependent, with a $\text{Q}_{10}$ of $\approx$ 16. Methods for measuring rennet coagulation properties include: 
\begin{enumerate} 
    \item Measurement of rennet coagulation time (RCT) using rotating bottles or tubes to detect the onset of visual coagulation. 
    \item Dynamic gel firmness testing using the Lattodinamografo to determine parameters such as rennet coagulation time (r), time to reach a firmness of 20 mm ($\text{k}_{20}$), and curd firmness at specific times ($\text{a}_{t}$). 
    \item Low-amplitude strain oscillation rheometry to compute storage modulus (G'), loss modulus (G''), and phase angle ($\delta$). 
    \item On-line sensors, such as hot wire probes or near-infrared (NIR) diffuse reflectance probes, to monitor gel firmness in the cheese vat. 
\end{enumerate}

\vspace{1em}  
Rennets include traditional animal extracts, primarily containing chymosin, and rennet substitutes. Chymosin is a single-chain polypeptide with a molecular mass of 35,600 Da and two catalytically active aspartyl residues. Acceptable rennet substitutes include: 
\begin{enumerate} 
    \item Bovine, porcine, and chicken pepsins. 
    \item Acid proteinases from \textit{Rhizomucor miehei}, \textit{R. pusillus}, and \textit{Cryphonectria parasitica}. 
    \item Fermentation-produced chymosin (FPC) cloned in organisms such as \textit{Aspergillus niger} or \textit{Kluyveromyces marxianus} var. \textit{lactis}. 
\end{enumerate} 

\vspace{1em}
Efficiency is maximized by using rennets with a high ratio of milk-clotting activity to proteolytic activity, such as calf chymosin or FPC, which minimize non-specific proteolysis and loss of fat and protein in the whey.


\section{RM 10: Fundamentals of Cheese Science; Chapter 08}
\textbf{Patrick F. Fox, Timothy P. Guinee, Timothy M. Cogan, Paul L. H. McSweeney, 2017}

\vspace{1em}
\textbf{Summary}

Following rennet-induced gelation, the coagulum is subjected to a series of treatments designed to encourage syneresis, which is the contraction of the para-casein matrix and the expulsion of the aqueous whey phase. This process effectively concentrates the fat and casein of milk approximately tenfold by removing moisture and soluble milk constituents. Methods used to measure the rate and extent of syneresis include quantifying the volume of expressed whey, monitoring changes in curd mass or density, using tracers to measure whey volume, and observing changes in electrical conductivity.

\vspace{1em}
Syneresis is significantly influenced by compositional and processing variables. Fat tends to reduce syneresis, while casein concentration is directly related to it. The rate of syneresis is inversely related to pH, reaching optimality at the isoelectric point of casein ($\approx$ 4.6), and is promoted by the addition of $\text{CaCl}_{2}$. Major processing factors that regulate dehydration include: 
\begin{enumerate} 
    \item Curd Particle Size: Smaller pieces accelerate syneresis due to the increased surface area available for the loss of whey. 
    \item Cook Temperature: Heating the curd-whey mixture promotes syneresis, with temperatures ranging from 31\textdegree C for soft varieties to 52--55\textdegree C for extra-hard varieties. 
    \item Rate of Acid Development: Lower pH values increase the rate and extent of syneresis by reducing inter-protein repulsion. 
    \item Stirring: Agitation facilitates heat transfer, prevents matting, and promotes syneresis through collisions between curd pieces. 
    \item Pressing: External pressure applied to the curd mass contributes significantly to the removal of remaining whey. 
\end{enumerate}

\vspace{1em}
Specific cheese varieties undergo texturization during manufacture to induce characteristic physical properties. Cheddaring involves piling and turning blocks of curd, causing the curd to flow under its own weight and results in the fusion of curd particles. This is accompanied by the solubilization of micellar calcium and an increase in para-casein hydration. In \textit{pasta filata} varieties, the curd is heated to $\approx$ 58--60\textdegree C and mechanically stretched, linearizing the para-casein matrix into fibers with fat trapped in elongated pools.

\vspace{1em}
Final operations include moulding, pressing, and packaging. Moulding allows the curd to form a continuous mass, with high pressure applied to low-moisture curds to ensure adequate fusion. Packaging protects the finished cheese from physical or microbial contamination, reduces moisture loss, and prevents physical deformation. Internal bacterially-ripened cheeses are often vacuum-packed in plastic film or coated with wax to maintain stability during ripening.