\setcounter{chapter}{0}
\setcounter{section}{0}
\chapter{Lecture Notes}
\setlength{\headheight}{12.71342pt}
\addtolength{\topmargin}{-0.71342pt}

\section{Introduction}
The following lecture notes have been compiled by combining personal notes taken during class, personal highlights from the lecture slides, and targeted use of generative AI. The AI was provided with my notes and the highlighted slides, together with carefully designed prompts, to ensure that it focused on the most relevant and interesting aspects of each lecture. This approach aims to produce a coherent and focused summary that reflects both the course content and my individual learning perspective.


\section{Lecture 01: Introduction - 17/11-2025}
\textbf{Dairy Product Technology 2}

\textbf{Anni Bygvrå Hougaard, Associate professor FOOD, 2025/2026}

\vspace{1em}
\textbf{Summary}

The course Dairy Product Technology 2 is designed to provide students with detailed knowledge of the theoretical and practical aspects of cheese and fermented milk production, characterisation, technology, and biochemistry. The areas of competence include the chemistry, biochemistry, microbiology, physics, and technology required for production, including the influences of milk quality and treatment. Students are expected to understand the characteristics and basic differences between various cheese groups and varieties, as well as the technology behind different fermented milk types. This involves the ability to use and evaluate scientific information regarding all steps of production, ripening, and packaging.

\vspace{1em}
The curriculum covers a wide range of scientific topics, including principles of manufacture, cheese yield, starter cultures, pH, microflora, and gas production. Further technical lectures focus on cheese proteolysis, amino acid catabolism, cheese flavour, and analytical methods such as peptide, amino acid, and aroma analysis. The course also addresses cheese defects, ripening engineering, curd formation, coagulants, and specific products such as cheese powder and processed cheese. The weekly structure consists of lectures and theoretical workshops on Mondays, followed by pilot plant and laboratory exercises on Wednesdays.

\vspace{1em}
Practicals and workshops are central to the course and consist of several structured components: 

\begin{enumerate} 
    \item Laboratory and/or pilot plant work. 
    \item Critical evaluation of results. 
    \item Calculations. 
    \item Discussions and conclusion about reliability of results. 
    \item Discussion of results in relation to literature and results of the other groups. 
    \item Report writing. 
\end{enumerate}

\vspace{1em}
In the dairy pilot plant, students produce cheddar cheese using both standard recipes and modified versions involving reduced fat, thermophilic cultures, or changed cutting techniques. Yoghurt production involves six samples with varying fat and protein content or post-fermentation processing, which are later analysed for viscosity, water binding, pH, and sensory properties. Cheese analysis practicals require groups to work with specific varieties, such as Feta, Danbo, Havarti, or Grana Padano, to perform analyses on moisture, ashes, salt, fat, protein, pH, protein degradation, lipolysis, and glycolysis.

\vspace{1em}
The course maintains a strict code of conduct, requiring students to be on time, follow technician instructions, and wear lab coats and safety glasses at all times in the laboratory. Students must read guidelines before starting work and perform calculations continuously, as an analysis is not considered finished until results are evaluated and interpreted. Reports for the cheese analysis practical must be approximately 20 pages and include the following contents: 

\begin{enumerate} 
    \item Title. 
    \item Name of all authors \& date. 
    \item Introduction with short background and objectives. 
    \item Material and Methods. 
    \item Results and Discussion. 
    \item Conclusion. 
    \item Reference list. 
\end{enumerate}

\vspace{1em}
The final examination is a 20-minute oral exam where a student randomly draws a question from a list distributed three weeks prior. The session includes a 7-8 minute presentation and a discussion of the question and the laboratory reports. Participation in practicals and the approval of reports are mandatory prerequisites for attending the exam.


\section{Lecture 02: Principles of Cheese Manufacture - 17/11-2025}
\textbf{Anni Bygvrå Hougaard, Associate professor FOOD, 2025/2026}

\vspace{1em}
\textbf{Summary}

Cheese is defined as coagulated milk and the concentration of milk nutrients, serving to preserve milk nutrients through a low water activity, a low pH, a high salt content, and microbial competition. In 2023, total milk production in the EU-27 was 160.8 million tonnes, of which 96\% was cows' milk. In the same year, Denmark produced 517,900 tonnes of cheese, corresponding to about 2\% of global production. The value of Danish cheese exports reached 14.74 billion DKK, accounting for 56\% of total dairy exports, with per capita consumption in Denmark estimated at 25 - 30 kg/year.

\vspace{1em}
Cheeses are grouped by coagulation method into those coagulated by renneting, those coagulated by acidification, and those made from cheese whey. Rennet-coagulated cheeses are further classified by firmness (MNFS \%) and minimum pH, including extra hard varieties like Grana (MNFS < 51\%), hard varieties like Emmental and Cheddar, semi-hard varieties such as Gouda, Danbo, and Havarti, and soft varieties like Brie and Feta. Acid-coagulated cheeses include Tvorog, Quarg, and Skyr, while whey cheeses are produced by precipitating proteins through heating and acidification (e.g., Ricotta) or by concentration through evaporation (e.g., Brunost).

\vspace{1em}
The principles of cheese manufacture follow a specific sequence of technical steps: 
\begin{enumerate} 
    \item Cheese milk treatment. 
    \item Coagulation - Setting. 
    \item Syneresis - Cutting/Stirring/Washing/Scalding. 
    \item Moulding \& Pressing. 
    \item Salting. 
    \item Surface treatment. 
    \item Ripening. 
\end{enumerate}

\vspace{1em}
Cheese milk treatment involves cold storage (< 5\textdegree C), which can lead to the loss of $\beta$-casein from micelles and poor rennetability. Standardisation is used to achieve a fat to protein ratio that ensures the most economical use of milk components. Heat treatment, commonly 72\textdegree C for 15 seconds, destroys pathogenic microorganisms but may cause whey protein denaturation and the transfer of soluble calcium and phosphate to the colloidal phase. Centrifugal bactofugation can remove 98 - 99.5\% of anaerobic spores.

\vspace{1em}
During syneresis, the curd is cut into cubes and stimulated by mixing and heating. Moulding and pressing techniques influence texture; pre-pressing under whey results in round-eyed cheese, while air between grains creates an open texture. Salting stimulates syneresis, lowers water activity, and provides preservation. Methods include salt on grains, dry salting, and salting in brine, with brine times ranging from 0.5 - 3 hours for soft cheeses to 3 weeks for extra hard cheeses.

\vspace{1em}
Surface treatments include plastic film, waxed surfaces, smeared surfaces, or mould growth. Ripening involves the growth and metabolism of bacteria, cell lysis, and enzyme activity in the cheese matrix. In semi-hard cheese ripening, 100\% of lactose and citrate are decomposed, while 25 - 30\% of casein and less than 1\% of milk fat are broken down. This process is performed by the coagulant, milk enzymes, starter cultures, and non-starter microorganisms.


\section{Lecture 03: Cheese Yield - 17/11-2025}
\textbf{Anni Bygvrå Hougaard, Associate professor FOOD, 2025/2026}

\vspace{1em}
\textbf{Summary}

Cheese yield is fundamentally defined by the recovery of milk components, particularly the maximum recovery of casein and fat while minimizing losses in the whey. Scientifically, yield is expressed as the weight of cheese (in kg) of a specific dry matter content produced from a defined quantity of milk with known protein and fat content. Two primary measures are utilized: Actual cheese yield (Ya), which is the weight of the cheese divided by the total weight of milk, starter, and salt, and Moisture-adjusted cheese yield (MACY), which adjusts the actual yield to a reference moisture content to allow for theoretical comparisons between different batches.

\vspace{1em}
Predictive formulae are employed to anticipate labour requirements, equipment needs, and profitability. These formulae are cheese-type dependent, ranging from the Babcock "rule of thumb" (Yield=1.1$\times$\% fat+2.5×\% casein) to the more complex Van Slyke and Publow formulae. The modified Van Slyke and Publow formula for all cheeses accounts for fat recovery (Kf), milk fat (F), milk casein (C), moisture (M), salt (SC), and whey solids (WS). Retention of milk solids is critical to these calculations; for instance, approximately 85-95\% of fat globules are retained in the casein network, while only 3-5\% of lactose is typically retained, most of which is converted into lactic acid. Protein retention is approximately 76.3\%, including para-casein and small amounts of denatured whey protein from heat treatments or starter cultures.

\vspace{1em}
Several biological and handling factors significantly influence the final yield: 
\begin{enumerate} 
    \item Protein genotypes: The BB genotype of $\kappa$-casein is associated with higher casein concentrations, superior renneting properties, and a 3-8\% increase in MACY. 
    \item Somatic cell count (SCC): High SCC indicates poor health status and is associated with casein-degrading proteinases that lead to soluble peptide losses in the whey. 
    \item Cold storage: Extended cooling (< 5\textdegree C) causes the solubilisation of micellar caseins, especially $\beta$-casein, making them susceptible to hydrolysis. 
    \item Milk handling: Excessive pumping or shearing damages the milk fat globule membrane (MFGM), leading to fat hydrolysis and lower recovery. 
    \item Microbial quality: High levels of psychrotrophic bacteria can produce lipases and proteinases that decrease yield. 
\end{enumerate}

\vspace{1em}
To improve production efficiency, several methods to increase cheese yield are implemented: 
\begin{enumerate} 
    \item Ultrafiltration (UF): Used at low, medium, or high concentrations to increase whey protein incorporation and improve fat and casein retention. 
    \item Microparticulated whey protein: Techniques like Leancreme utilize high heat and shear force to create particles similar to fat globules, resulting in a yield increase of 6-10\%. 
    \item Low proteolytic coagulants: Using camel chymosin (CHY-MAX M) provides a high clotting-to-proteolytic (C/P) ratio, reducing protein loss compared to microbial coagulants. 
    item Transglutaminase: This enzyme creates iso-peptide bonds between proteins, increasing moisture retention and yield. 
    \item Phospholipase: YieldMAX (PLA1) modifies the MFGM by hydrolyzing phospholipids to lysophospholipids, which act as emulsifiers to prevent globule rupture, particularly in pasta filata cheese, increasing yield by $\geq$ 1\%. 
\end{enumerate}

