\chapter{Abbreviations and Explanations}
\setlength{\headheight}{12.71342pt}
\addtolength{\topmargin}{-0.71342pt}

\rowcolors{2}{white}{gray!7}
\begin{longtable}{ p{5cm} | p{2cm} | p{7.5cm} }
    
    \textbf{Topic} & \textbf{Abb.} & \textbf{Description} \\ 
    \hline
    \endfirsthead
    
    
    \textbf{Topic} & \textbf{Abb.} & \textbf{Description} \\ 
    \hline
    \endhead
    
    \multicolumn{3}{r}{\textit{Continued on next page}} \\ 
    \endfoot
    
    \endlastfoot
        
        \textbf{AA-genotype} & \textbf{n.a.} & \textit{A genetic variant of $\kappa$-casein associated with poorer milk coagulation properties, resulting in softer curds and lower cheese yields compared to other genotypes such as BB or AB.} \\

        \textbf{Agitation} & \textbf{n.a.} & \textit{The mechanical stirring or movement of curd grains during cheese making, which affects syneresis by preventing matting and sedimentation, and by applying external pressure that aids whey expulsion.} \\

        \textbf{Aminopeptidase } (\textit{PepN, PepC}) & \textbf{n.a.} & \textit{Intracellular enzymes in LAB that cleave amino acids sequentially from the N-terminus of peptides, contributing to the breakdown of oligopeptides into free amino acids during cheese ripening.} \\

        \textbf{Aspartyl} & \textbf{n.a.} & \textit{A proteolytic enzyme whose active site contains an aspartic acid residue, such as chymosin, which cleaves specific peptide bonds in casein proteins during early cheese ripening.} \\

        \textbf{BB-genotype} & \textbf{n.a.} & \textit{A genetic variant of $\kappa$-casein associated with improved milk coagulation properties, leading to firmer curds and higher cheese yields compared to other genotypes such as AA or AB.} \\

        \textbf{Blowing} & \textbf{n.a.} & \textit{The formation of unwanted gas pockets or holes in cheese during ripening, caused by the activity of gas-producing bacteria that ferment lactose or lactate into gases such as $\text{H}_2$ and $\text{CO}_2$.} \\

        \textbf{Buffering capacity} & \textbf{n.a.} & \textit{The ability of milk or cheese to resist changes in pH upon the addition of acids or bases, primarily due to the presence of caseins and colloidal calcium phosphate, which helps maintain a stable environment during cheese making and ripening.} \\

        \textbf{Capillary electrophoresis} & \textbf{CE} & \textit{An analytical technique that separates peptides and proteins based on their size-to-charge ratio by applying an electric field in a capillary tube, commonly used in cheese research to monitor proteolysis during ripening.} \\

        \textbf{$\alpha-casein$} & \textbf{n.a.} & \textit{A major casein protein in milk that is a primary substrate for chymosin during cheese ripening, where its hydrolysis leads to texture development. Two major $\alpha$-casein groups are $\alpha_s1$-casein and $\alpha_s2$-casein.} \\

        \textbf{$\beta$-casein} & \textbf{n.a.} & \textit{A major surface-active  casein protein in milk that is more extensively hydrolysed by microbial coagulants during cheese ripening, contributing to flavour and texture development.} \\

        \textbf{$\gamma$-caseins} (\textit{gamma-casein}) & \textbf{n.a.} & \textit{Peptide fragments produced from the proteolytic cleavage of $\beta$-casein during cheese ripening, contributing to flavour and texture development.} \\

        \textbf{$\kappa$-casein} & \textbf{n.a.} & \textit{The surface-active casein protein that stabilises casein micelles and whose specific cleavage by coagulants initiates the coagulation process in cheese making.} \\

        \textbf{Cell-envelope proteinase} & \textbf{CEP} & \textit{A proteolytic enzyme anchored to the cell wall of LAB, such as lactocepin, that initiates the breakdown of casein proteins into oligopeptides during cheese ripening.} \\

        \textbf{Chymosin} & \textbf{n.a.} & \textit{A proteolytic enzyme derived from rennet (in the stomachs of young ruminants), primarily responsible for the specific cleavage of $\kappa$-casein at the $\text{Phe}_{105}-\text{Met}_{106}$ bond during cheese making, leading to milk coagulation and curd formation.} \\

        \textbf{Clostridial cell} & \textbf{n.a.} & \textit{Refers to bacterial cells belonging to the genus \textit{Clostridium}, which are anaerobic, spore-forming bacteria that can cause late blowing defects in cheese by fermenting lactate into butyric acid and gases such as $\text{H}_2$ and $\text{CO}_2$.} \\

        \textbf{Clotting activity} (\textit{C}) & \textbf{n.a.} & \textit{The ability of a coagulant enzyme to specifically cleave $\kappa$-casein and induce milk coagulation, leading to curd formation during cheese making.} \\

        \textbf{Colloidal calcium phosphate} & \textbf{CCP} & \textit{The insoluble, inorganic mineral complex that acts as a structural "cementing" agent within casein micelles to maintain their integrity, constitutes approximately 66\% of the total calcium in milk, and serves as the primary buffering agent during the acidification process in cheesemaking} \\

        \textbf{C/P-activity} & \textbf{n.a.} & \textit{The ratio of clotting activity to proteolytic activity of a coagulant enzyme, indicating its specificity for $\kappa$-casein cleavage relative to its overall proteolytic potential, which influences cheese quality and flavour development.} \\

        \textbf{D-culture} & \textbf{n.a.} & \textit{A mesophilic starter containing standard acid-producing \textit{lactococci} and the citrate-fermenting \textit{Lactococcus lactis} subsp. \textit{lactis} biovar \textit{diacetylactis} ($\text{Cit}^+$ \textit{Lactococcus}), which produces $\text{CO}_2$ and diacetyl from citrate, contributing to the characteristic flavour and eye formation in certain cheeses like Gouda and Edam.} \\

        \textbf{DL-culture} & \textbf{n.a.} & \textit{A mesophilic mixed-strain starter containing acid-producing \textit{lactococci} and both citrate-fermenting \textit{Lactococcus} ($\text{Cit}^+$ \textit{Lactococcus}) and \textit{Leuconostoc} species ($\text{Cit}^+$ \textit{Leuconostoc}) as aroma-producing components.} \\

        \textbf{Decarboxylation} & \textbf{n.a.} & \textit{The enzymatic removal of a carboxyl group from amino acids, resulting in the production of amines or gases such as $\text{CO}_2$, which can lead to defects like late blowing in cheese.} \\

        \textbf{Dissociation} & \textbf{n.a.} & \textit{The process by which casein micelles disintegrate into smaller subunits or individual casein molecules, often induced by changes in pH, temperature, or ionic strength during cheese making.} \\

        \textbf{Dynamic rearrangement, fusion, and tightening} & \textbf{n.a.} & \textit{The ongoing structural changes within the casein network during cheese ripening, leading to a denser and more compact matrix by the formation of new bonds and fusion of junction zones.} \\

        \textbf{Endogenous} & \textbf{n.a.} & \textit{Originating from within the system, such as the internal mechanisms of the casein network that lead to spontaneous contraction and whey expulsion during cheese making.} \\

        \textbf{Endogenous syneresis pressure} & \textbf{n.a.} & \textit{The internal pressure generated within the cheese curd due to spontaneous contraction of the casein network, which drives the expulsion of whey.} \\
        
        \textbf{Endopeptidase} (\textit{PepO, PepF}) & \textbf{n.a.} & \textit{Intracellular peptidases of lactic acid bacteria that cleave internal peptide bonds within imported oligopeptides, generating shorter peptides that can be further degraded to free amino acids during cheese ripening.} \\

        \textbf{Entrapped serum} & \textbf{n.a.} & \textit{The aqueous whey phase that is entrapped within the cheese matrix after the initial expulsion phase, often found in the microstructure of the cheese.} \\

        \textbf{Exogenous lipases} & \textbf{n.a.} & \textit{Lipolytic enzymes that are added to milk or cheese from external sources, such as microbial or fungal lipases, to enhance flavour development through targeted hydrolysis of milk fat during cheese ripening. Also known to shorten ripening time.} \\

        \textbf{Expulsion} (\textit{syneresis}) & \textbf{n.a.} & \textit{The process by which whey is expelled from the curd matrix during cheese making, influenced by factors such as pH, temperature, and mechanical pressure.} \\

        \textbf{F1-23} & \textbf{n.a.} & \textit{The peptide fragment consisting of the first 23 amino acids of $\alpha_{s1}$-casein, released upon cleavage by chymosin during cheese making, which is further degraded by the proteolytic system of lactic acid bacteria during ripening.} \\

        \textbf{F24-199} & \textbf{n.a.} & \textit{The peptide fragment consisting of amino acids 24 to 199 of $\alpha_{s1}$-casein, remaining after the initial cleavage by chymosin, which undergoes further proteolysis during cheese ripening.} \\

        \textbf{F29-209} & \textbf{n.a.} & \textit{The C-terminal peptide fragment of $\beta$-casein, consisting of amino acids 29 to 209, remaining after initial cleavage by chymosin, which is further degraded during cheese ripening.} \\

        \textbf{F106-209} & \textbf{n.a.} & \textit{The peptide fragment of $\beta$-casein consisting of amino acids 106 to 209, produced after initial cleavage by chymosin, which undergoes further proteolysis during cheese ripening.} \\

        \textbf{F108-209} & \textbf{n.a.} & \textit{The peptide fragment of $\beta$-casein consisting of amino acids 108 to 209, produced after initial cleavage by chymosin, which undergoes further proteolysis during cheese ripening.} \\

        \textbf{Glycomacropeptide} & \textbf{GMP} & \textit{The hydrophilic C-terminal fragment of $\kappa$-casein released upon enzymatic cleavage by chymosin, which remains soluble in the whey fraction during cheese making.} \\

        \textbf{Hairy layer} & \textbf{n.a.} & \textit{The outer layer of $\kappa$-casein molecules that extend from the surface of casein micelles into the surrounding serum, providing steric stabilization and preventing micelle aggregation prior to coagulation.} \\

        \textbf{Hydrolysis} & \textbf{n.a.} & \textit{The cleavage of chemical bonds, e.g. peptide bonds in proteins, through the addition of water, typically catalysed by enzymes such as proteases during cheese making and ripening.} \\

        \textbf{L-culture} & \textbf{n.a.} & \textit{A mesophilic mixed-strain starter containing acid-producing \textit{lactococci} and citrate-fermenting \textit{Leuconostoc} species ($\text{Cit}^+$ \textit{Leuconostoc}) as the aroma-producing component.} \\

        \textbf{Lactate metabolism} & \textbf{n.a.} & \textit{The microbial conversion of lactic acid in cheese into other compounds such as $\text{CO}_2$, volatile acids, and neutral metabolites, leading to pH increase and driving secondary ripening processes.} \\

        \textbf{Lactocepin} & \textbf{n.a.} & \textit{A cell-envelope associated proteinase (CEP) anchored to the cell wall of LAB, responsible for the initial hydrolysis of casein proteins into oligopeptides during cheese ripening.} \\

        \textbf{Lipoprotein lipase} & \textbf{LPL} & \textit{An indigenous milk enzyme which is associated with the surface of fat globules and catalyses the hydrolysis of triglycerides into free fatty acids and glycerol, contributing to flavour development in cheese.} \\

        \textbf{Low-amplitude strain/stress oscillation rheometry} & \textbf{LASOR} & \textit{A rheological measurement technique in which small, non-destructive oscillatory strains or stresses are applied to a fermented milk gel to probe its linear viscoelastic properties, thereby characterising structure formation without disrupting the protein network.} \\

        \textbf{Lysis} & \textbf{n.a.} & \textit{The breakdown or rupture of bacterial cells, leading to the release of intracellular enzymes that contribute to proteolysis and flavour development during cheese ripening.} \\

        \makecell[l]{\textbf{Lysozyme-sensitive} / \\ \textbf{lyse-sensitive}} & \textbf{n.a.} & \textit{Refers to bacterial cells that can be lysed (broken down) by the enzyme lysozyme, which hydrolyses the peptidoglycan layer of bacterial cell walls, leading to the release of intracellular enzymes that contribute to cheese ripening.} \\ 

        \textbf{Micellar calcium} & \textbf{n.a.} & \textit{The fraction of calcium that is bound within the casein micelles in milk, contributing to their structural integrity, stability, and buffering capacity during cheese making.} \\

        \textbf{Muslin bags} & \textbf{n.a.} & \textit{Porous cloth bags used in cheese making to contain curd while allowing whey to drain, facilitating syneresis and curd consolidation.} \\

        \textbf{Nitrogen fractionation} & \textbf{n.a.} & \textit{The separation of cheese nitrogen into different soluble and insoluble fractions using chemical reagents such as TCA and PTA, allowing for the assessment of proteolysis during cheese ripening.} \\

        \textbf{Non-start lactic acid bacteria} & \textbf{NSLAB} & \textit{A group of lactic acid bacteria that are not part of the initial starter culture used in cheese making, but can grow during ripening and contribute to flavour development, proteolysis, and potential defects such as gas production.} \\

        \textbf{Nozzle-separator} & \textbf{n.a.} & \textit{A high-capacity centrifugal separator that concentrates fermented milk by expelling whey through fixed nozzles while continuously discharging the protein-rich phase, enabling controlled concentration with limited mechanical disruption of the gel.} \\

        \textbf{Nuclei} (primer) & \textbf{n.a.} & \textit{Small heterogeneities in the cheese matrix that act as initiation sites for the formation of larger gas bubbles during blowing defects, often originating from entrapped air or particulate matter.} \\

        \textbf{Oligopeptide transport system} & \textbf{Opp} & \textit{A membrane transport system in LAB that facilitates the uptake of oligopeptides generated from casein proteolysis into the bacterial cell for further intracellular degradation.} \\

        \textbf{Para-casein} & \textbf{n.a.} & \textit{The aggregated form of casein micelles that results from the enzymatic cleavage of $\kappa$-casein during cheese making, leading to gelation and curd formation.} \\

        \textbf{Para-Casein matrix} & \textbf{n.a.} & \textit{The protein network formed by casein micelles in cheese after the enzymatic cleavage of $\kappa$-casein, leading to aggregation and gelation.} \\

        \textbf{Peptide} & \textbf{n.a.} & \textit{Short chains of amino acids that are produced during the proteolytic breakdown of proteins, such as caseins, during cheese ripening, some of which can contribute to bitterness if they accumulate excessively.} \\
    
        \textbf{$\text{Phe}_{23}-\text{Phe}_{24}$} & \textbf{n.a.} & \textit{The specific peptide bond in $\alpha$-casein that is cleaved by chymosin during cheese ripening, contributing to the softening of cheese texture.} \\ 

        \textbf{$\text{Phe}_{105}-\text{Met}_{106}$} & \textbf{n.a.} & \textit{The specific peptide bond in $\kappa$-casein that is cleaved by chymosin and other coagulants during cheese making, leading to the destabilization of casein micelles and initiation of gelation.} \\

        \textbf{Phosphotungstic acid} & \textbf{PTA} & \textit{A chemical reagent used in cheese analysis to precipitate larger peptides, enabling the assessment of secondary proteolysis by measuring smaller peptides and free amino acids in the soluble nitrogen fraction.} \\

        \textbf{Piquancy} & \textbf{n.a.} & \textit{A sharp, pungent sensory impression in cheese arising primarily from short- and medium-chain free fatty acids and their secondary catabolic products generated during lipolysis, contributing to a spicy, biting flavour perception characteristic of certain ripened cheeses.} \\

        \textbf{Plasmin} & \textbf{n.a.} & \textit{An endogenous milk protease that primarily hydrolyses $\beta$-casein during cheese ripening, contributing to texture and flavour development.} \\

        \textbf{Plasminogen} & \textbf{n.a.} & \textit{A heat-stable, inactive zymogen precursor of the indigenous milk proteinase plasmin that is primarily associated with casein micelles and converted into its active proteolytic form by specific plasminogen activators.} \\

        \textbf{Plasticiser} & \textbf{n.a.} & \textit{A substance that increases the plasticity or fluidity of a material, such as fat in cheese, which disrupts the protein matrix and enhances creaminess and mouthfeel.} \\

        \textbf{Porous matrix} & \textbf{n.a.} & \textit{A three-dimensional network of aggregated casein or para-casein micelles that forms the structural foundation of cheese and fermented milk products by physically entrapping fat globules and the aqueous serum phase within its interconnected interstitial pores} \\

        \textbf{PP5} & \textbf{n.a.} & \textit{A plasmin-derived protease in the group of proteose-peptones that specifically cleaves $\beta$-casein during cheese ripening, contributing to texture and flavour development.} \\

        \textbf{PP8-fast} & \textbf{n.a.} & \textit{A plasmin-derived protease in the group of proteose-peptones that is rapidly formed during early cheese ripening due to the fast action of endogenous plasmin on intact $\beta$-casein, contributing to texture and flavour development.} \\

        \textbf{PP8-slow} & \textbf{n.a.} & \textit{A plasmin-derived protease in the group of proteose-peptones that gradually is formed during cheese ripening due to the slow action of endogenous plasmin on intact $\beta$-casein, contributing to texture and flavour development.} \\

        \textbf{Pregrastric lipase} & \textbf{PGE} & \textit{A lipolytic enzyme derived from the stomachs of young ruminants that specifically hydrolyses short- and medium-chain triglycerides, releasing free fatty acids that contribute to flavour development in cheese.} \\

        \textbf{Primary proteolysis} & \textbf{n.a.} & \textit{The initial stage of protein breakdown in cheese caused mainly by residual coagulant and indigenous milk enzymes, resulting in large and intermediate peptides from intact caseins.} \\

        \textbf{Proline-specific peptidase} (\textit{PepX, PepQ, PepR}) & \textbf{n.a.} & \textit{Intracellular enzymes in LAB that hydrolyse peptide bonds adjacent to proline residues, aiding in the breakdown of proline-rich peptides, which generally are resistant to other peptidases, during cheese ripening.} \\

        \textbf{Propionic acid bacteria} & \textbf{PAB} & \textit{A group of bacteria, such as \textit{Propionibacterium freudenreichii}, that ferment lactate into propionic acid, acetic acid, and $\text{CO}_2$, contributing to the characteristic flavour and eye formation in Swiss-type cheeses.} \\

        \textbf{Proteolytic activity} (\textit{P}) & \textbf{n.a.} & \textit{The ability of a coagulant enzyme to hydrolyse various casein proteins beyond the primary $\kappa$-casein cleavage, affecting cheese texture and flavour development during ripening.} \\

        \textbf{Proteolysis} & \textbf{n.a.} & \textit{The enzymatic breakdown of proteins into smaller peptides and amino acids, a key process during cheese ripening that influences texture and flavour development.} \\

        \textbf{PrtP} & \textbf{n.a.} & \textit{A specific type of serine CEP found in certain LAB strains, involved in the initial proteolysis of casein during cheese ripening.} \\

        \textbf{Psychrotrophic bacteria} & \textbf{n.a.} & \textit{Bacteria that can grow at low temperatures, such as those found in refrigerated milk, which can produce lipases and proteases that contribute to spoilage and off-flavours in cheese.} \\

        \textbf{Quarg-separator} & \textbf{n.a.} & \textit{A continuous centrifugal separation device that concentrates fermented milk by mechanically removing whey from the acidified protein gel, thereby increasing total solids while controlling texture through defined shear and residence time.} \\

        \textbf{Rennet paste} & \textbf{n.a.} & \textit{A traditional form of rennet preparation made by drying and grinding the stomach lining of young ruminants, containing chymosin and other proteolytic enzymes used in cheese making.} \\

        \makecell[l]{\textbf{Reverd-Phase High-Performance} \\ \textbf{Liquid Chromatography}} & \textbf{RP-HPLC} & \textit{An analytical technique used to separate and quantify peptides and proteins based on their hydrophobicity, commonly employed in cheese research to monitor proteolysis during ripening.} \\

        \textbf{Secondary proteolysis} & \textbf{n.a.} & \textit{The subsequent stage of proteolysis during cheese ripening, involving further breakdown of peptides into smaller peptides and free amino acids by various proteases, contributing to flavour and texture development.} \\

        \textbf{$\text{Ser}_{104}-\text{Phe}_{105}$} & \textbf{n.a.} & \textit{The specific peptide bond in $\kappa$-casein that is cleaved by the coagulant from \textit{C. parasitica}, leading to excessive proteolysis and destabilization of casein micelles during cheese making.} \\

        \textbf{Somatic cell count} & \textbf{SCC} & \textit{A measure of the number of somatic cells (mainly white blood cells) present in milk, used as an indicator of milk quality and udder health, with higher counts often associated with mastitis. The cheese yield is typically reduced with higher SCC due to increased proteolytic activity.} \\

        \textbf{Spontaneous contraction} & \textbf{n.a.} & \textit{Internal mechanisms by which the casein network contracts, resulting in generated pressure, leading to the expulsion of whey from the gel matrix.} \\

        \textbf{Steric} & \textbf{n.a.} & \textit{Stabilisation of particles in suspension, such as casein micelles, through physical hindrance provided by surface-bound molecules (like the "hairy layer" of $\kappa$-casein), preventing close approach and aggregation.} \\

        \textbf{Steric barrier} & \textbf{n.a.} & \textit{The spatial hindrance created by the hydrated $\kappa$-casein surface layer that prevents close approach and aggregation of casein micelles by maintaining a physical separation between their protein surfaces.} \\

        \textbf{Trichloroacetic acid} & \textbf{TCA} & \textit{A strong organic acid used in cheese analysis to precipitate proteins, allowing for the measurement of soluble nitrogen fractions that indicate the extent of proteolysis during ripening.} \\

        \textbf{Van der Walls forces} & \textbf{n.a.} & \textit{Weak attractive forces between molecules or particles, such as casein micelles, that can contribute to their aggregation when other stabilizing forces (like electrostatic repulsion) are reduced during cheese making.} \\

        \textbf{$\zeta$-potential} (\textit{zeta-potential}) & \textbf{n.a.} & \textit{The electrostatic potential at the slipping plane (sorunding area which is electrically charged) of particles in suspension, such as casein micelles, which influences their stability and aggregation behaviour during cheese making.} \\
        
    \end{longtable}

