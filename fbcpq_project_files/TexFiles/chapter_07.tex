\chapter{Abbreviations and Explanations}
\setlength{\headheight}{12.71342pt}
\addtolength{\topmargin}{-0.71342pt}

\rowcolors{2}{white}{gray!7}
\begin{longtable}{ p{5cm} | p{2cm} | p{7.5cm} }
    
    \textbf{Topic} & \textbf{Abb.} & \textbf{Description} \\ 
    \hline
    \endfirsthead
    
    
    \textbf{Topic} & \textbf{Abb.} & \textbf{Description} \\ 
    \hline
    \endhead
    
    \multicolumn{3}{r}{\textit{Continued on next page}} \\ 
    \endfoot
    
    \endlastfoot
    
    \textbf{Spontaneous contraction} & \textbf{n.a.} & \textit{Internal mechanisms by which the casein network contracts, resulting in generated pressure, leading to the expulsion of whey from the gel matrix.} \\

    \textbf{Para-Casein matrix} & \textbf{n.a.} & \textit{The protein network formed by casein micelles in cheese after the enzymatic cleavage of $\kappa$-casein, leading to aggregation and gelation.} \\

    \textbf{Expulsion} (\textit{syneresis}) & \textbf{n.a.} & \textit{The process by which whey is expelled from the curd matrix during cheese making, influenced by factors such as pH, temperature, and mechanical pressure.} \\

    \textbf{Dynamic rearrangement, fusion, and tightening} & \textbf{n.a.} & \textit{The ongoing structural changes within the casein network during cheese ripening, leading to a denser and more compact matrix by the formation of new bonds and fusion of junction zones.} \\

    \textbf{Endogenous syneresis pressure} & \textbf{n.a.} & \textit{The internal pressure generated within the cheese curd due to spontaneous contraction of the casein network, which drives the expulsion of whey.} \\

    \textbf{Entrapped serum} & \textbf{n.a.} & \textit{The aqueous whey phase that is entrapped within the cheese matrix after the initial expulsion phase, often found in the microstructure of the cheese.} \\

    \textbf{Porous matrix} & \textbf{n.a.} & \textit{A three-dimensional network of aggregated casein or para-casein micelles that forms the structural foundation of cheese and fermented milk products by physically entrapping fat globules and the aqueous serum phase within its interconnected interstitial pores} \\

    \textbf{Agitation} & \textbf{n.a.} & \textit{The mechanical stirring or movement of curd grains during cheese making, which affects syneresis by preventing matting and sedimentation, and by applying external pressure that aids whey expulsion.} \\

    \textbf{Endogenous} & \textbf{n.a.} & \textit{Originating from within the system, such as the internal mechanisms of the casein network that lead to spontaneous contraction and whey expulsion during cheese making.} \\

    \textbf{Colloidal calcium phosphate} & \textbf{CCP} & \textit{The insoluble, inorganic mineral complex that acts as a structural "cementing" agent within casein micelles to maintain their integrity, constitutes approximately 66\% of the total calcium in milk, and serves as the primary buffering agent during the acidification process in cheesemaking} \\


    
    
    \end{longtable}

