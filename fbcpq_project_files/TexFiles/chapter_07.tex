\chapter{Abbreviations and Explanations}
\setlength{\headheight}{12.71342pt}
\addtolength{\topmargin}{-0.71342pt}

\rowcolors{2}{white}{gray!7}
\begin{longtable}{ p{5cm} | p{2cm} | p{7.5cm} }
    
    \textbf{Topic} & \textbf{Abb.} & \textbf{Description} \\ 
    \hline
    \endfirsthead
    
    
    \textbf{Topic} & \textbf{Abb.} & \textbf{Description} \\ 
    \hline
    \endhead
    
    \multicolumn{3}{r}{\textit{Continued on next page}} \\ 
    \endfoot
    
    \endlastfoot
    
    \textbf{Spontaneous contraction} & \textbf{n.a.} & \textit{Internal mechanisms by which the casein network contracts, resulting in generated pressure, leading to the expulsion of whey from the gel matrix.} \\

    \textbf{Para-Casein matrix} & \textbf{n.a.} & \textit{The protein network formed by casein micelles in cheese after the enzymatic cleavage of $\kappa$-casein, leading to aggregation and gelation.} \\

    \textbf{Expulsion} (\textit{syneresis}) & \textbf{n.a.} & \textit{The process by which whey is expelled from the curd matrix during cheese making, influenced by factors such as pH, temperature, and mechanical pressure.} \\

    \textbf{Dynamic rearrangement, fusion, and tightening} & \textbf{n.a.} & \textit{The ongoing structural changes within the casein network during cheese ripening, leading to a denser and more compact matrix by the formation of new bonds and fusion of junction zones.} \\

    \textbf{Endogenous syneresis pressure} & \textbf{n.a.} & \textit{The internal pressure generated within the cheese curd due to spontaneous contraction of the casein network, which drives the expulsion of whey.} \\

    \textbf{Entrapped serum} & \textbf{n.a.} & \textit{The aqueous whey phase that is entrapped within the cheese matrix after the initial expulsion phase, often found in the microstructure of the cheese.} \\

    \textbf{Porous matrix} & \textbf{n.a.} & \textit{A three-dimensional network of aggregated casein or para-casein micelles that forms the structural foundation of cheese and fermented milk products by physically entrapping fat globules and the aqueous serum phase within its interconnected interstitial pores} \\

    \textbf{Agitation} & \textbf{n.a.} & \textit{The mechanical stirring or movement of curd grains during cheese making, which affects syneresis by preventing matting and sedimentation, and by applying external pressure that aids whey expulsion.} \\

    \textbf{Endogenous} & \textbf{n.a.} & \textit{Originating from within the system, such as the internal mechanisms of the casein network that lead to spontaneous contraction and whey expulsion during cheese making.} \\

    \textbf{Colloidal calcium phosphate} & \textbf{CCP} & \textit{The insoluble, inorganic mineral complex that acts as a structural "cementing" agent within casein micelles to maintain their integrity, constitutes approximately 66\% of the total calcium in milk, and serves as the primary buffering agent during the acidification process in cheesemaking} \\

    \textbf{$\text{Phe}_{105}-\text{Met}_{106}$} & \textbf{n.a.} & \textit{The specific peptide bond in $\kappa$-casein that is cleaved by chymosin and other coagulants during cheese making, leading to the destabilization of casein micelles and initiation of gelation.} \\

    \textbf{$\kappa$-casein} & \textbf{n.a.} & \textit{The surface-active casein protein that stabilises casein micelles and whose specific cleavage by coagulants initiates the coagulation process in cheese making.} \\

    \textbf{$\text{Ser}_{104}-\text{Phe}_{105}$} & \textbf{n.a.} & \textit{The specific peptide bond in $\kappa$-casein that is cleaved by the coagulant from \textit{C. parasitica}, leading to excessive proteolysis and destabilization of casein micelles during cheese making.} \\

    \textbf{$\zeta$-potential} (\textit{zeta-potential}) & \textbf{n.a.} & \textit{The electrostatic potential at the slipping plane (sorunding area which is electrically charged) of particles in suspension, such as casein micelles, which influences their stability and aggregation behaviour during cheese making.} \\

    \textbf{Steric} & \textbf{n.a.} & \textit{Stabilisation of particles in suspension, such as casein micelles, through physical hindrance provided by surface-bound molecules (like the "hairy layer" of $\kappa$-casein), preventing close approach and aggregation.} \\
    
    \textbf{$\text{Phe}_{23}-\text{Phe}_{24}$} & \textbf{n.a.} & \textit{The specific peptide bond in $\alpha$-casein that is cleaved by chymosin during cheese ripening, contributing to the softening of cheese texture.} \\

    \textbf{$\alpha-casein$} & \textbf{n.a.} & \textit{A major casein protein in milk that is a primary substrate for chymosin during cheese ripening, where its hydrolysis leads to texture development. Two major $\alpha$-casein groups are $\alpha_s1$-casein and $\alpha_s2$-casein.} \\

    \textbf{$\beta$-casein} & \textbf{n.a.} & \textit{A major surface-active  casein protein in milk that is more extensively hydrolysed by microbial coagulants during cheese ripening, contributing to flavour and texture development.} \\

    \textbf{Peptide} & \textbf{n.a.} & \textit{Short chains of amino acids that are produced during the proteolytic breakdown of proteins, such as caseins, during cheese ripening, some of which can contribute to bitterness if they accumulate excessively.} \\

    \textbf{Chymosin} & \textbf{n.a.} & \textit{A proteolytic enzyme derived from rennet (in the stomachs of young ruminants), primarily responsible for the specific cleavage of $\kappa$-casein at the $\text{Phe}_{105}-\text{Met}_{106}$ bond during cheese making, leading to milk coagulation and curd formation.} \\

    \textbf{Plasmin} & \textbf{n.a.} & \textit{An endogenous milk protease that primarily hydrolyses $\beta$-casein during cheese ripening, contributing to texture and flavour development.} \\

    \textbf{Clotting activity} (\textit{C}) & \textbf{n.a.} & \textit{The ability of a coagulant enzyme to specifically cleave $\kappa$-casein and induce milk coagulation, leading to curd formation during cheese making.} \\

    \textbf{Proteolytic activity} (\textit{P}) & \textbf{n.a.} & \textit{The ability of a coagulant enzyme to hydrolyse various casein proteins beyond the primary $\kappa$-casein cleavage, affecting cheese texture and flavour development during ripening.} \\

    \textbf{C/P-activity} & \textbf{n.a.} & \textit{The ratio of clotting activity to proteolytic activity of a coagulant enzyme, indicating its specificity for $\kappa$-casein cleavage relative to its overall proteolytic potential, which influences cheese quality and flavour development.} \\

    \textbf{Hydrolysis} & \textbf{n.a.} & \textit{The cleavage of chemical bonds, e.g. peptide bonds in proteins, through the addition of water, typically catalysed by enzymes such as proteases during cheese making and ripening.} \\

    \textbf{Proteolysis} & \textbf{n.a.} & \textit{The enzymatic breakdown of proteins into smaller peptides and amino acids, a key process during cheese ripening that influences texture and flavour development.} \\

    \textbf{Primary proteolysis} & \textbf{n.a.} & \textit{The initial stage of protein breakdown in cheese caused mainly by residual coagulant and indigenous milk enzymes, resulting in large and intermediate peptides from intact caseins.} \\

    \textbf{Secondary proteolysis} & \textbf{n.a.} & \textit{The subsequent stage of proteolysis during cheese ripening, involving further breakdown of peptides into smaller peptides and free amino acids by various proteases, contributing to flavour and texture development.} \\

    \textbf{Trichloroacetic acid} & \textbf{TCA} & \textit{A strong organic acid used in cheese analysis to precipitate proteins, allowing for the measurement of soluble nitrogen fractions that indicate the extent of proteolysis during ripening.} \\

    \textbf{Phosphotungstic acid} & \textbf{PTA} & \textit{A chemical reagent used in cheese analysis to precipitate larger peptides, enabling the assessment of secondary proteolysis by measuring smaller peptides and free amino acids in the soluble nitrogen fraction.} \\

    \textbf{$\gamma$-caseins} (\textit{gamma-casein}) & \textbf{n.a.} & \textit{Peptide fragments produced from the proteolytic cleavage of $\beta$-casein during cheese ripening, contributing to flavour and texture development.} \\

    \textbf{Reversed-Phase High-Performance Liquid Chromatography} & \textbf{RP-HPLC} & \textit{An analytical technique used to separate and quantify peptides and proteins based on their hydrophobicity, commonly employed in cheese research to monitor proteolysis during ripening.} \\

    



    \end{longtable}

