\chapter*{Preface}
\setlength{\headheight}{12.71342pt}
\addtolength{\topmargin}{-0.71342pt}
These course notes have been prepared as part of the NFOK16006U course Dairy Product Technology 2 at the University of Copenhagen, covering the period from November 2025 to February 2026.

\vspace{1em}
The notes compile material and reflections relevant to the course and are intended as a resource to enhance the learning experience for students. The content is shared freely and may be used as study material or as a template for structuring individual notes.

\vspace{1em}
All information is provided without responsibility for its correctness, and users are encouraged to verify data, formulas, and interpretations with the original sources and course materials.

\vspace{1em}
Please enjoy reading these notes, and feel free to reach out if you have any questions.


\chapter*{Course Description}
\setlength{\headheight}{12.71342pt}
\addtolength{\topmargin}{-0.71342pt}

\section*{Education}
MSc Programme in Agriculture


\section*{Content}
The aim of the course is to give students a detailed knowledge of both theoretical and practical aspects of cheese and fermented milk production, characterisation, technology and biochemistry.


\section*{Learning Outcome}
The course is targeted to students interested in plant science (Horticulture and Agriculture) and food science students who are particularly interested in fruit and berry crops and the quality and use of the raw materials/food products these crops provide.
\subsection*{Knowledge}

\begin{itemize}
    \item Understand chemistry, biochemistry, microbiology, physics and technology in the production of cheese and fermented milks.
    \item Understand the characteristics of, and basic differences between, cheese groups and varieties.
    \item Understand the use and application of dairy enzymes in cheese and fermented milks
\end{itemize}

\subsection*{Skills}

\begin{itemize}
    \item Apply critical analysis of cheese and fermented milk quality and characteristics.
\end{itemize}

\subsection*{Competences}

\begin{itemize}
    \item Ability to use and evaluate scientific information and knowledge concerning cheese and fermented milks, incl. all steps of production and ripening.
    \item Ability to comprehend the technology behind different cheese varieties and fermented milk types.
    \item Capacity to critically evaluate laboratory results and pilot plant trials with regard to the possible development of innovative cheese and fermented milks and/or processes.
\end{itemize}


%\section*{Literature}
%See Absalon for a list of course literature.


%\section*{Recommended Academic Qualifications}
%Qualifications corresponding to Dairy Internship and within the field of milk processing is recommended.

Academic qualifications equivalent to a BSc degree is recommended.


\section*{Teaching and Learning Methods}
The course involves lectures and laboratory practicals, workshops and oral presentations, literature studies, patent searches and report writing. The student must obtain their own laboratory results and evaluate them in relation to theory given at lectures and found in literature, including critical use of electronic data bases and internet. Results from all students' practicals and literature studies will be merged during oral presentations and discussions in workshops.

\section*{Workload}

\begin{table}[h]
    \centering
    \caption{A table with an overview over the workload for the course.}
    \label{tab:workload}
    \rowcolors{2}{white}{gray!7}
    \begin{tabular}{ l | c}
        \textbf{Category} & \textbf{Hours} \\ 
        \hline
        Lectures & 32 \\ 

        Class Instruction & 15 \\

        Preparation & 75 \\ 

        Practical exercises & 75 \\

        Project work & 8 \\

        Exam & 1 \\

        \hline
        Total & 206 \\ 
    \end{tabular}
\end{table}

\section*{Exam}

\begin{table}[h]
    \centering
    \caption{A table with an overview over the elaborated description of the course}
    \label{tab:elaborated_description}
    \rowcolors{2}{white}{gray!7}
    \begin{tabular}{ l | p{10cm} }
        Credit & 7.5 ECTS \\ 
        
        Type of assessment & Oral examination, 20 minutter \\ 

        Type of assessment details & 3 weeks before the examination, all examination questions (covering the essential issues dealt with in the course) will be published on Absalon. The students are expected to use all of these questions to be fully prepared for the subsequent oral exam. At the oral exam itself, one of the questions will be drawn by the student, and the examination immediately proceeds. For the question drawn, all aids are allowed during the exam (approx. 15 minutes). The last 5 minutes of the exam will be used for questions in relation to the reports prepared from the laboratory practicals. \\

        Examination prerequisites & Attendance at all laboratory practicals is compulsory. All laboratory reports must have been approved. \\

        Aid & All aids allowed 

        \href{https://kunet.ku.dk/study/food-science-technology-ma/Pages/topic.aspx?topicid=20fc0507-633c-455d-85ae-eb53a44d4072}{Read about how to use Generative AI on KuNet}\\

        Marking scale & 7-point grading scale \\

        Censorship form &   \begin{itemize}
                                \item No external censorship
                                \item Several internal examiner
                            \end{itemize} \\

        Re-exam &   Same as ordinary exam. 
        Non-approved reports must be handed in three weeks prior to the re-exam and approved 2 weeks prior to the re-exam. \\ 
    \end{tabular}
\end{table}


\newpage