\chapter{Exam Questions and Answers}
\setlength{\headheight}{12.71342pt}
\addtolength{\topmargin}{-0.71342pt}

This chapter of the course notes compiles the exam questions for the course held in February 2026, along with their respective answers prepared by me. The purpose of this section is twofold: firstly, to provide a reflective exercise that consolidates understanding of the course material; and secondly, to document my comprehension of the course topics as assessed through the exam questions.

\vspace{1em}
To ensure citation accuracy and academic transparency, NotebookLM has been employed as the primary generative AI platform. Its use has focused on verifying that all citations accurately reference the uploaded course materials and lecture slides provided by the professors. Beyond citation control, this section also represents an ongoing exploration of prompt engineering - refining interaction design to optimise AI output quality, precision, and academic reliability. Through this approach, the work aims to maintain a high academic standard while enhancing clarity, structure, and depth in written responses.

\vspace{1em}
There are a total of 10 questions in the exam, each consisting of multiple structured sub-questions. The questions are designed to assess both theoretical understanding and applied knowledge within dairy product technology. The numbering of sections corresponds directly to the numbering of the exam questions, ensuring a clear and consistent structure throughout. Questions 1-8 primarily address cheese manufacture and ripening, while questions 9-10 focus on fermented milk products, including structure formation and production processes. Each question is presented as a comprehensive topic intended for oral presentation, followed by examiner-led follow-up questions and discussion.

\section*{Examination Details}
Exam is oral (total time 20 minutes)

\vspace{0.5em}
At the oral exam a question will be randomly drawn by the student.

\vspace{0.5em}
For this question, all aids are allowed (15 minutes oral exam). You should present your answer within 7-8 min, and you will then be asked some follow-up questions for the remainder of the time.

\vspace{0.5em}
For the final part of the exam (approx. 5 minutes), topics related to the report from the practical part of the course will be discussed.

\vspace{0.5em}
It is recommended to bring written answers to the questions in a more or less ready form, which may be used as an aid for answering the drawn question. It may also be recommended to bring a printed copy of the report for the final part of the exam. 


\newpage
\section{Controlling the Moisture Content of Cheese}
Water is expelled from the milk gel during production of cheese, and this is an important processing step for regulating the moisture content of the final cheese.

\subsection*{1.1}
\vspace{1em}
Give the name of this process and describe the mechanism (also on a microstructural level) behind how water is being expelled from the milk gel.

\subsubsection*{Answer}
Syneresis is the process by which water is expelled from the milk gel during cheese production. It involves the spontaneous contraction of the para-casein matrix, leading to the expulsion of the aqueous whey phase that contains water and water-soluble milk constituents.

\vspace{1em}
On a microstructural level, the mechanism is driven by the dynamic rearrangement, fusion, and tightening of the casein network. This internal restructuring, often referred to as microsyneresis, involves the formation of new bonds between casein strands and the fusion of junction zones. As these rearrangements occur, the casein particles move into a more compact structure, which creates an endogenous syneresis pressure (stress) within the viscoelastic network. This pressure effectively "squeezes" the physically entrapped serum out of the porous matrix toward the surface of the curd grains.

\vspace{1em}
The local transport of whey through the gel is governed by Darcy’s equation, given in equation \ref{eq:ex_answer_darcys_equation}:
\begin{equation}
v = \frac{B}{\eta} \times \frac{p}{l}
\label{eq:ex_answer_darcys_equation}
\end{equation}

This states that the flow velocity (v) depends on the factors described in table \ref{tab:exam_question_1.1_table}.
\begin{table}[h]
    \centering
    \caption{Parameters affecting whey flow velocity during syneresis according to Darcy's equation.}
    \label{tab:exam_question_1.1_table}
    \rowcolors{2}{white}{gray!7}
    \begin{tabular}{ l | p{10cm} }
        \textbf{Parameter} & \textbf{Description} \\ 
        \hline
        \textbf{B}: (Permeability coefficient) & Average size and number of pores in the gel network \\ 

        \textbf{p}: (Pressure) & Sum of endogenous syneresis pressure, external pressure, and gravitational pressure \\

        \textbf{$\eta$}: (Viscosity) & Viscosity of the expelling whey \\ 

        \textbf{l}: (Distance) & Length of the path over which the liquid must flow to reach a surface \\
    \end{tabular}
\end{table}

\vspace{1em}
As syneresis progresses, the pore size and permeability of the gel decrease because the network strands become thicker and the overall structure becomes more constrained, which eventually slows the rate of whey expulsion.

\subsection*{1.2}
\vspace{1em}
Explain how cutting the milk gel, stirring of the curd grains and temperature changes in the cheese vat effect the amount of water that is expelled. Elaborate on how pH at whey drainage affects calcium level in the cheese.

\subsubsection*{Answer}

\begin{enumerate} 
    \item \textbf{Cutting the Milk Gel:} Cutting the coagulum into smaller grains accelerates syneresis by significantly increasing the surface area available for whey escape and reducing the distance (l) the whey must travel to reach the grain surface. Finer cuts lead to higher flow velocities (v) and lower moisture content in the final cheese. See figure \ref{fig:exam_question_1.2_figure} for a schematic illustration of the cutting effect.
    \item \textbf{Stirring of the Curd Grains:} Agitation prevents the grains from matting or sedimenting, which would otherwise impede syneresis. Stirring also applies external pressure ($\text{p}_{ex}$) through collisions between curd grains and against vat walls, increasing the total pressure (p) driving the whey out. 
    \item \textbf{Temperature Changes:} Increasing the temperature of the curd-whey mixture (cooking/scalding) promotes syneresis by increasing hydrophobic interactions between casein molecules. This leads to faster network rearrangements, a higher endogenous syneresis pressure, and a more rapid increase in the permeability of the gel. 
\end{enumerate}

\vspace{1em}
Effect of pH at Whey Drainage on Calcium Levels The pH level at the time of whey drainage is the primary determinant of the total calcium-to-casein ratio in the final cheese. In native milk, a large portion of calcium exists as colloidal calcium phosphate (CCP), which acts as a structural cross-link within the micelles. As the pH decreases during manufacture due to the fermentation of lactose to lactic acid, this CCP gradually solubilizes into the serum phase.

\vspace{1em}
Because the whey removed during drainage represents the majority of the liquid lost during the process, any calcium that has been solubilized is permanently removed from the curd. Consequently, a lower pH at drainage results in lower retained calcium and a lower buffering capacity in the finished cheese. High-calcium cheeses (drained at high pH, e.g., Emmental) tend to be more elastic, while low-calcium cheeses (drained at lower pH, e.g., Feta or Cheddar) are typically more short or brittle.

\vspace{0.5em}
\begin{figure}[h]
    \centering
    \resizebox{0.9\textwidth}{!}{%
    \begin{tikzpicture}[line width=2pt]
            
    % --- Labels
    \node at (2,6,0) {\textbf{Milk gel before cutting}};
    \node at (2,5.5,0) {Surface area = $6 \times 4^2 = 96 cm^2$};
    \node at (9,6,0) {\textbf{Milk gel after cutting}};
    \node at (9,5.5,0) {Surface area = $8 \times 6 \times 2^2 = 192 cm^2$};
    
    % --- Big cube: 4x4x4 (scale factor 2)
    \begin{scope}[scale=2]
      \perfectcube
    \end{scope}
    \draw[->, line width=1pt]
    (1,-1,2) -- (3.2,-1,2)
    node[midway, below] {$l_{big_cube}=2cm$};
    
    % --- Four small cubes: 2x2x2 (aligned in height with big cube)
 
    % top row
    \begin{scope}[shift={(7,2,0)}]
      \perfectcube
    \end{scope}
        \draw[->, line width=1pt]
        (8,-1.2,2) -- (9,-1.2,2)
        node[midway, below] {$l_{small_cubes}=1cm$};
    \begin{scope}[shift={(10,2,0)}]
      \perfectcube
    \end{scope}
    
    % bottom row
    \begin{scope}[shift={(7,-1,0)}]
      \perfectcube
    \end{scope}
    \begin{scope}[shift={(10,-1,0)}]
      \perfectcube
    \end{scope}

    \end{tikzpicture}}
    
    \caption{Schematic illustration of the cutting effect of curd grains on syneresis. The left cube is uncut, while the right sides shows the smaller cubes. "l" is the distance whey must travel from the interior to reach the surface.}
    \label{fig:exam_question_1.2_figure}
\end{figure}


\newpage
\section{Influence of Type of Coagulants on Cheese Manufacture and Ripening}
\subsection*{2.1}
Coagulants from different sources (animals, microorganisms, plants) can be used in the production of cheese. Give examples of different types of commonly used coagulant preparations and describe differences in their primary action on casein in the cheese vat as well as what kind of differences the different coagulants may cause during ripening?

\subsubsection*{Answer A}
Common Types of Coagulant Preparations Coagulants are categorized by their origin and have transitioned from traditional animal extracts to highly specified recombinant enzymes: 
\begin{enumerate} 
    \item \textbf{Animal Rennet:} Traditionally extracted from the abomasum of neonatal mammals (calves, lambs, or kids). It primarily consists of \textbf{chymosin} (80-95\%) and \textbf{bovine pepsin} (5-20\%). 
    \item \textbf{Microbial/Fungal Coagulants:} Derived from fungi such as \textit{Rhizomucor miehei}, \textit{Rhizomucor pusillus}, and \textit{Cryphonectria parasitica}. These are widely used due to lower costs but vary significantly in heat stability and proteolytic specificity. 
    \item \textbf{Fermentation-Produced Chymosin (FPC):} Produced by genetically modified organisms (e.g., \textit{Aspergillus niger} or \textit{Kluyveromyces lactis}) into which the bovine or camel chymosin gene has been inserted. FPC is considered the "ideal" milk-clotting enzyme due to its high purity. 
    \item \textbf{Vegetable Coagulants:} Extracted from plants, most successfully the flowers of the cardoon thistle (\textit{Cynara cardunculus}), used for specific ewe’s milk cheeses. 
\end{enumerate}


\subsubsection*{Answer B}
Primary Action in the Cheese Vat The primary role of a coagulant is to destabilize the casein micelle. In the cheese vat, most coagulants (chymosin, FPC, and \textit{Rhizomucor} spp.) act by \textbf{selectively hydrolysing the $\text{Phe}_{105} - \text{Met}_{106}$ peptide bond of $\kappa$-casein}. (Note: \textit{C. parasitica} is an exception, cleaving the $\text{Ser}_{104} - \text{Phe}_{105}$ bond). This cleavage releases the hydrophilic \textbf{caseinomacropeptide (CMP/GMP)} into the whey, removing the micelle's "hairy layer" and reducing the zeta potential. This loss of steric and electrostatic repulsion allows the sensitized micelles to aggregate into a 3D viscoelastic gel network.

\subsubsection*{Answer C}
Differences During Ripening Coagulant activity continues long after the curd is formed, as 0-15\% (up to 30\% for chymosin at low pH) of the enzyme is retained in the curd. 
\begin{enumerate} 
    \item \textbf{Substrate Specificity:} Chymosin is responsible for the \textbf{initial hydrolysis of $\alpha$-casein} (specifically at $\text{Phe}_{23} - \text{Phe}_{24}$), which is vital for softening the cheese texture. Microbial coagulants generally exhibit higher activity on $\beta$-casein. 
    \item \textbf{Bitterness Risk:} Coagulants with a low ratio of clotting-to-proteolytic activity (e.g., some microbial and vegetable rennets) carry a high risk of developing bitter off-flavours due to the excessive or non-specific accumulation of bitter peptides. 
    \item \textbf{Camel Chymosin:} Exhibits the highest known specificity, significantly reducing bitterness during ripening compared to bovine chymosin. 
    \item \textbf{High-Cook Cheeses:} In varieties like Emmental or Parmesan, the high cooking temperatures partially or fully inactivate chymosin, making \textbf{plasmin} the dominant agent for primary proteolysis. 
\end{enumerate}


\subsection*{2.2}
\vspace{1em}
Describe what is meant by clotting activity versus proteolytic activity of a coagulant. What is the optimal proteolytic activity of a coagulant?

\subsubsection*{Answer}
Clotting (C) vs. Proteolytic (P) Activity 
\begin{enumerate} 
    \item \textbf{Clotting Activity (C):} Refers to the enzyme's specific ability to hydrolyse $\kappa$-casein to induce coagulation. It is measured in \textbf{IMCU} (International Milk Clotting Units). 
    \item \textbf{Proteolytic Activity (P):} Refers to non-specific hydrolysis of other bonds in caseins (e.g., $\alpha_{s1}$ - and $\beta$-casein). 
    \item \textbf{Optimal Activity:} The optimal proteolytic activity is represented by a high C/P ratio. A high ratio ensures a firm coagulum and high yield while protecting the cheese from flavour and texture defects. Chymosin is the benchmark for this high specificity. 
\end{enumerate}

\subsection*{2.3}
\vspace{1em}
How could the different activity of coagulants during cheese ripening be investigated?

\subsubsection*{Answer}
Investigating Coagulant Activity During Ripening Different activity patterns can be investigated through biochemical and physical analysis: 
\begin{enumerate} 
    \item \textbf{Nitrogen Fractionation:} Measuring Nitrogen content in fractions such as pH 4.6 soluble N (index of primary proteolysis) or 12\% TCA and 5\% PTA soluble N (index of secondary proteolysis and amino acids). 
    \item \textbf{Urea-PAGE and Capillary Electrophoresis (CE):} These techniques allow for the visualization and quantification of the degradation of intact $\alpha_{s1}$ - and $\beta$-caseins into their respective breakdown products (e.g., $\alpha_{s1}$ -I and $\gamma$-caseins). 
    \item \textbf{RP-HPLC and LC-MS:} Used to create peptide profiles and identify specific peptides produced by different coagulants. 
    \item \textbf{Sensory/Rheological Assessment:} Correlating bitterness scores with peptide accumulation or monitoring fracture stress/firmness as the protein matrix degrades. 
\end{enumerate}


\newpage
\section{Cheese Yield}
The cheese yield varies due to a number of cheese making parameters, and it varies between different cheese types.

\subsection*{3.1}
\vspace{1em}
Describe which major milk components of the cheese milk that are retained, and which are not retained in the cheese, and describe why they are retained or not retained. Elaborate on the difference between Cheese Yield and Moisture Adjusted Cheese Yield (MACY).

\subsubsection*{Answer A}
\textbf{Retention of Milk Components} Cheese manufacture is a dehydration process where milk nutrients are concentrated approximately tenfold. The retention of components is governed by their physical state and solubility: 
\begin{enumerate} 
    \item \textbf{Casein:} Typically \textasciitilde 76\% is retained. As the structural basis of the gel, insoluble para-casein remains in the matrix while the hydrophilic glycomacropeptide (GMP) is released into the whey after renneting. 
    \item \textbf{Fat:} Retained at high levels (85-95\%) because fat globules (0.5-10 $\mu$m) are physically entrapped within the narrowing pores of the casein network. Losses occur if globules are mechanically damaged ("free fat") or located on the cut surfaces of curd grains. 
    \item \textbf{Whey Proteins:} Predominantly not retained as they are soluble at cheesemaking pH and temperatures. However, they can be included if denatured via heat, causing them to complex with $\kappa$-casein. 
    \item \textbf{Lactose:} Only 3-5\% is retained. Being fully soluble, it remains in the whey; the small portion remaining in the curd is eventually fermented into lactic acid. 
    \item \textbf{Minerals and Citrate:} Ash retention is \textasciitilde 35-40\% in rennet cheeses. Colloidal calcium phosphate remains associated with the casein, while soluble minerals partition into the whey. 
\end{enumerate}

\subsubsection*{Answer B}
\textbf{Actual Yield vs. Moisture Adjusted Cheese Yield (MACY)} Actual cheese yield (Y$_a$) is the quantity of cheese produced from a defined weight of milk, often expressed as kg cheese per 100 kg milk. While useful for immediate production weights, Y$_a$ is limited because it does not account for variations in moisture or milk composition.
Moisture Adjusted Cheese Yield (MACY) is a theoretical value that normalizes the yield to a reference moisture content (e.g., 37\% for Cheddar). This allows for a meaningful comparison of efficiency across different batches or factories where moisture levels may fluctuate. The formula is given in equation \ref{eq:ex_03_adjustet_moisture}: 
\begin{equation}
    \text{Y}_{ma}=\text{Y}_{a} \times \frac{100-\text{Actual Moisture} \%}{100-\text{Reference Moisture} \%}
    \label{eq:ex_03_adjustet_moisture}
\end{equation}


\subsection*{3.2}
\vspace{1em}
Describe how milk quality parameters can affect cheese yield. Describe industrial methods/ingredients which may be used to increase cheese yield.

\subsubsection*{Answer A}
\textbf{Effect of Milk Quality Parameters} 
\begin{enumerate} 
    \item \textbf{Somatic Cell Count (SCC):} Elevated SCC (above 100,000 cells/mL) significantly reduces yield. High SCC milk contains indigenous proteases (plasmin) that degrade casein into soluble peptides that are lost in the whey. 
    \item \textbf{Protein Genotypes:} The BB genotype of $\kappa$-casein is associated with higher casein levels and superior renneting properties, increasing MACY by 3-8\% compared to the AA variant. 
    \item \textbf{Cold Storage:} Storing raw milk below 5\textdegree C can lead to the dissociation of $\beta$-casein from micelles and the growth of psychrotrophs. These bacteria produce heat-stable proteases and lipases that reduce component recovery and yield. 
\end{enumerate}

\subsubsection*{Answer B}
\textbf{Industrial Methods to Increase Yield} 
\begin{enumerate} 
    \item \textbf{Ultrafiltration (UF):} UF concentrates all milk proteins, including whey proteins, which are then retained in the curd. It also creates a firmer coagulum, improving fat and casein retention. 
    \item \textbf{Microparticulated Whey Protein (MWP):} Commercial products like LeanCreme\texttrademark\ use high heat and shear to create whey particles similar in size to fat globules. This can increase yield by 6-10\% through protein incorporation and increased water binding. 
    \item \textbf{Enzymatic Additives:} Phospholipase (e.g., YieldMAX) improves fat recovery by modifying the fat globule membrane. Transglutaminase can be used to cross-link proteins, increasing moisture retention. 
    \item \textbf{Heat Treatment:} Heating milk above 72\textdegree C denatures whey proteins, causing them to complex with the casein matrix and increasing total protein recovery. 
\end{enumerate}


\newpage
\section{Importance of pH Development in Cheese Ripening}
During the first day(s), pH in cheese decreases to a minimum and then it increases at different rates depending on cheese variety.

\subsection*{4.1}
\vspace{1em}
Describe how different pH minima (the lowest pH obtained during cheese making) may be obtained, and give examples on cheese varieties with low, medium and high minimum-pH.

\subsubsection*{Answer}

The pH minimum of a cheese is primarily determined by the ratio of residual lactose content to the buffering capacity of the curd, which consists mainly of proteins and inorganic phosphate. This acidification process occurs in two distinct phases: Phase 1 takes place before whey separation, where lactose fermented into lactic acid is continuously replaced by lactose diffusing from the surrounding whey into the curd grains. Phase 2 begins after whey separation, and the further decrease in pH is determined solely by the amount of lactose remaining in the curd relative to the buffering substances. Cheesemakers can obtain a higher pH minimum by using curd washing, which involves removing a portion of the whey and replacing it with warm water to dilute the lactose concentration in the curd. Additionally, high cooking temperatures can be used to inhibit the starter bacteria, slowing the rate of acid production and resulting in a higher minimum pH. Examples of minimum-pH levels include: 
\begin{enumerate} 
    \item \textbf{Low Minimum-pH ($\approx$ 4.6-4.7):} Feta, Camembert, and Danablu. 
    \item \textbf{Medium Minimum-pH ($\approx$ 4.8-5.2):} Cheddar (4.75-4.95), Gouda (5.20-5.25), and Danbo (5.20-5.25). 
    \item \textbf{High Minimum-pH ($\approx$ 5.3+):} Emmental (5.2-5.30) and Halloumi ($\approx$ 6.1). 
\end{enumerate}

\subsection*{4.2}
\vspace{1em}
Give examples on cheese varieties with low, medium and high final pH (pH of the ripened cheese), respectively, and describe which microbial/enzymatic activities that are involved.

\subsubsection*{Answer}
The final pH of a ripened cheese depends on the specific microbial and enzymatic pathways utilized by the microflora during maturation. Proteolysis contributes to a general increase in pH across most varieties by increasing the buffering capacity of the cheese matrix. Specific examples include: 
\begin{enumerate} 
    \item \textbf{Low Final pH (Cheddar, $\approx$ 5.5):} In Cheddar, non-starter lactic acid bacteria (NSLAB) often convert L-lactate to D-lactate, but the low moisture and storage temperatures generally limit extensive pH increases. 
    \item \textbf{Medium Final pH (Danbo, $\approx$ 6.2--6.3):} Varieties like Danbo and Gouda see a gradual rise in pH as proteolysis releases peptides and amino acids, and the slow metabolism of lactate occurs. 
    \item \textbf{High Final pH (Camembert/Brie, $\approx$ 7.0--8.0):} In these varieties, the oxidative deamination of amino acids by surface moulds and yeasts produces ammonia ($NH_3$), which significantly neutralizes the acidity. Blue cheeses also reach high final pH levels ($\approx$ 6.5) due to similar ammonia production from mould activity. 
\end{enumerate}


\subsection*{4.3}
\vspace{1em}
Differences in pH between the interior and the surface of cheeses are most pronounced in cheeses with surface ripening. Mention two different cheese varieties with different kind of surface ripening. Explain how pH gradients develop during ripening and how they influence ripening using one of the cheese varieties as an example.

\subsubsection*{Answer}
Two distinct varieties of surface-ripened cheeses include white-mould cheeses, such as \textbf{Camembert} or \textbf{Brie}, and bacterial smear-ripened cheeses, such as \textbf{Limburger} or \textbf{Danbo} with smear. In Camembert, a steep pH gradient develops during ripening because the surface microflora (initially yeasts like \textit{Kluyveromyces lactis} and the mould \textit{Penicillium camemberti}) metabolize lactic acid to $CO_2$ and $H_2O$. As the surface lactate is exhausted, the mould begins the oxidative deamination of amino acids, producing ammonia ($NH_3$) which diffuses from the surface into the cheese interior. This process increases the surface pH to $\approx$ 7.0-8.0 while the centre remains more acidic ($\approx$ 4.5-5.0). These gradients influence ripening in the following ways: 
\begin{enumerate} 
    \item \textbf{Mineral Migration:} The neutralization of the surface causes a concentration gradient that makes lactic acid migrate to the surface and calcium phosphate migrate from the interior to precipitate at the surface. 
    \item \textbf{Enzyme Stimulation:} The increase in pH stimulates the activity of the indigenous milk enzyme \textbf{plasmin}, which is most active at neutral pH, thereby accelerating primary proteolysis. 
    \item \textbf{Texture Softening:} The combination of pH increase, the loss of micellar calcium from the interior, and increased proteolysis results in the characteristic extensive softening of the cheese body from the surface toward the centre. 
\end{enumerate}


\vspace{2em} 
\textbf{Analogy for Presentation:} You can think of the pH minimum as a "battery charge" set during production; varieties like Feta are fully charged with acid (low pH), while Gouda is only partially charged due to curd washing. During ripening, the surface microflora acts like an "acid-eater," neutralizing the surface. In a Brie, this neutral zone acts like a magnet, pulling structural minerals (calcium) to the surface and letting enzymes "wake up" to turn the firm interior into a soft, spreadable paste.


\newpage
\section{Gas Production in Cheese}
\subsection*{5.1}
Gas may be produced from the metabolism of lactose and citrate. Which are the main microorganisms involved in this metabolism, and in which type of cheeses is this relevant?

\subsubsection*{Answer}

\subsection*{5.2}
\vspace{1em}
Gas production in cheese is not always desirable and may be considered as a serious defect. Mention three different types of gas producing reactions (substrate and the organisms involved) in which undesired gas production may occur. What measures are available to the cheese producer to control undesired gas production in cheese?

\subsubsection*{Answer}

\subsection*{5.3}
\vspace{1em}
In Swiss cheese, eye formation is an important cheese quality parameter. What factors (microbial and physio-chemical) are necessary to ensure that satisfactory eye formation develops?

\subsubsection*{Answer}


\newpage
\section{Cheese Proteolysis}
\subsection*{6.1}
Proteolysis is a fundamental process in cheese ripening. List enzymes that hydrolyse peptide bonds in intact casein molecules in rennet coagulated semi-hard cheeses, describe the specificity of these enzymes and give the first peptides that are produced.

\subsubsection*{Answer}

\subsection*{6.2}
\vspace{1em}
Explain how the first peptide released from $\alpha_{S1}$-casein by chymosin in cheese ($\alpha_{S1}$-casein (f1-23)), may be further hydrolysed all the way to amino acids, and mention which enzymes that are involved and what the origin of these enzymes may be.

\subsubsection*{Answer}

\subsection*{6.3}
\vspace{1em}
Give examples of two cheese types with different primary proteolysis (first attack on intact casein) and explain how the differences are related to cheese manufacture. Elaborate on suitable laboratory methods for measuring proteolysis in cheese.

\subsubsection*{Answer}

\subsection*{6.4}
\vspace{1em}
What approaches can be used to accelerate proteolysis and what are the advantages/disadvantages of each approach?

\subsubsection*{Answer}


\newpage
\section{Amino Acids Catabolism and Cheese Flavour Formation}
\subsection*{7.1}
Amino acids are released from casein derived peptides in cheese during ripening and used to different extents by different microorganisms. Describe the enzymes that are directly involved in amino acid release incl. their name, specificity and origin.

\subsubsection*{Answer}

\subsection*{7.2}
\vspace{1em}
Amino acids may be catabolized to aroma compounds in cheese. What are the main pathways (and the enzymes involved) in the catabolism of amino acids?

\subsubsection*{Answer}

\subsection*{7.3}
\vspace{1em}
Which are the four groups of amino acids, which breakdown products may contribute to the cheese flavours 

\begin{enumerate}
    \item Malty
    \item Floral
    \item Cooked cabbage/garlic
    \item Buttery
\end{enumerate}

respectively?

\subsubsection*{Answer:}
\begin{table}[h]
    \centering
    \caption{A table with an overview over the workload for the course.}
    \label{tab:exam_question_7.3_table}
    \rowcolors{2}{white}{gray!7}
    \begin{tabular}{ l | c}
        \textbf{Flavour} & \textbf{Amino Acid group} \\ 
        \hline
        Malty & X \\ 

        Floral & XX \\

        Cooked cabbage/garlic & XXX \\ 

        Buttery & XXXX \\

    \end{tabular}
\end{table}

\subsection*{7.4}
\vspace{1em}
Which analytical method can be used for identification of aroma compounds in cheese?

\subsubsection*{Answer}


\newpage
\section{Importance of Milk Fat in Cheese Flavour}
Low-fat cheese commonly lacks the complex and balanced flavour of normal-fat cheese.

\subsection*{8.1}
\vspace{1em}
Describe three different roles that milk fat may have in cheese flavour and describe how the off-flavour bitterness is related to fat content in cheese.

\subsubsection*{Answer}

\subsection*{8.2}
\vspace{1em}
During ripening, fat is degraded to different extent in different cheese varieties from less than 1\% to about 10\% or more. Which lipolytic enzymes are involved in different kinds of cheeses, and where do they origin from? Which kind of aroma compounds may be produced from lipolysis and further catabolism of the fatty acids released, and what flavour notes do each of these aroma compounds introduce.

\subsubsection*{Answer}

\subsection*{8.3}
\vspace{1em}
Give examples of compounds from milk fat catabolism that contribute to the characteristic flavour of the two cheeses, Parmesan and Blue cheese.
How can lipolysis be enhanced in cheese during ripening?

\subsubsection*{Answer}


\newpage
\section{Structure Formation in Fermented Milk Products}
During acidification of milk, e.g. during fermentation using a standard culture of lactic acid bacteria for yoghurt production, several changes occur in the milk components, leading to formation of the final structure of fermented milk products.

\subsection*{9.1}
\vspace{1em}
Describe the changes occurring in the casein micelle as a result of acidification.

\subsubsection*{Answer}

\subsection*{9.2}
\vspace{1em}
What are the main factors affecting structure formation in fermented milk products, and how can these be used to control the structural properties of the final product?
How can you measure the structure formation?

\subsubsection*{Answer}


\newpage
\section{Production of Fermented Milk Products}
A number of basic processing steps are included in most production processes for fermented milk products.

\subsection*{10.1}
\vspace{1em}
Present an outline of a standard production process for yoghurt manufacture (set or stirred), including a short description of the aim and outcome of the selected parameters for each processing step.

\subsubsection*{Answer}

\subsection*{10.2}
\vspace{1em}
How can concentration for production of fermented milk products like Greek yoghurt, ymer and skyr be obtained? Both traditional methods and modern technologies should be addressed.

\subsubsection*{Answer}

\subsection*{10.3}
\vspace{1em}
Describe which molecules (apart from lactic acid) that are largely responsible for the flavour of plain yoghurt and describe how they are formed.

\subsubsection*{Answer}



