\chapter{Exam Questions and Answers}
\setlength{\headheight}{12.71342pt}
\addtolength{\topmargin}{-0.71342pt}

This chapter of the course notes compiles the exam questions for the course held in February 2026, along with their respective answers prepared by me. The purpose of this section is twofold: firstly, to provide a reflective exercise that consolidates understanding of the course material; and secondly, to document my comprehension of the course topics as assessed through the exam questions.

\vspace{1em}
To ensure citation accuracy and academic transparency, NotebookLM has been employed as the primary generative AI platform. Its use has focused on verifying that all citations accurately reference the uploaded course materials and lecture slides provided by the professors. Beyond citation control, this section also represents an ongoing exploration of prompt engineering - refining interaction design to optimise AI output quality, precision, and academic reliability. Through this approach, the work aims to maintain a high academic standard while enhancing clarity, structure, and depth in written responses.

\vspace{1em}
There are a total of 10 questions in the exam, each consisting of multiple structured sub-questions. The questions are designed to assess both theoretical understanding and applied knowledge within dairy product technology. The numbering of sections corresponds directly to the numbering of the exam questions, ensuring a clear and consistent structure throughout. Questions 1-8 primarily address cheese manufacture and ripening, while questions 9-10 focus on fermented milk products, including structure formation and production processes. Each question is presented as a comprehensive topic intended for oral presentation, followed by examiner-led follow-up questions and discussion.

\section*{Examination Details}
Exam is oral (total time 20 minutes)

\vspace{0.5em}
At the oral exam a question will be randomly drawn by the student.

\vspace{0.5em}
For this question, all aids are allowed (15 minutes oral exam). You should present your answer within 7-8 min, and you will then be asked some follow-up questions for the remainder of the time.

\vspace{0.5em}
For the final part of the exam (approx. 5 minutes), topics related to the report from the practical part of the course will be discussed.

\vspace{0.5em}
It is recommended to bring written answers to the questions in a more or less ready form, which may be used as an aid for answering the drawn question. It may also be recommended to bring a printed copy of the report for the final part of the exam. 


\newpage
\section{Controlling the Moisture Content of Cheese}
Water is expelled from the milk gel during production of cheese, and this is an important processing step for regulating the moisture content of the final cheese.

\subsection*{1.1}
\vspace{1em}
Give the name of this process and describe the mechanism (also on a microstructural level) behind how water is being expelled from the milk gel.

\subsubsection*{Answer}
Syneresis is the process by which water is expelled from the milk gel during cheese production. It involves the spontaneous contraction of the para-casein matrix, leading to the expulsion of the aqueous whey phase that contains water and water-soluble milk constituents.

\vspace{1em}
On a microstructural level, the mechanism is driven by the dynamic rearrangement, fusion, and tightening of the casein network. This internal restructuring, often referred to as microsyneresis, involves the formation of new bonds between casein strands and the fusion of junction zones. As these rearrangements occur, the casein particles move into a more compact structure, which creates an endogenous syneresis pressure (stress) within the viscoelastic network. This pressure effectively "squeezes" the physically entrapped serum out of the porous matrix toward the surface of the curd grains.

\vspace{1em}
The local transport of whey through the gel is governed by Darcy’s equation, given in equation \ref{eq:ex_answer_darcys_equation}:
\begin{equation}
v = \frac{B}{\eta} \times \frac{p}{l}
\label{eq:ex_answer_darcys_equation}
\end{equation}

This states that the flow velocity (v) depends on the factors described in table \ref{tab:exam_question_1.1_table}.
\begin{table}[h]
    \centering
    \caption{Parameters affecting whey flow velocity during syneresis according to Darcy's equation.}
    \label{tab:exam_question_1.1_table}
    \rowcolors{2}{white}{gray!7}
    \begin{tabular}{ p{4cm} | p{10cm} }
        \textbf{Parameter} & \textbf{Description} \\ 
        \hline
        \textbf{B}: (Permeability coefficient) 
        & Average size and number of pores in the gel network \\ 

        \textbf{p}: (Pressure) 
        & \makecell[l]{Sum of endogenous syneresis pressure, external pressure, and gravita- 
        \\tional pressure} \\

        \textbf{$\eta$}: (Viscosity) 
        & Viscosity of the expelling whey \\ 

        \textbf{l}: (Distance) 
        & Length of the path over which the liquid must flow to reach a surface \\
    \end{tabular}
\end{table}



\vspace{1em}
As syneresis progresses, the pore size and permeability of the gel decrease because the network strands become thicker and the overall structure becomes more constrained, which eventually slows the rate of whey expulsion.

\subsection*{1.2}
\vspace{1em}
Explain how cutting the milk gel, stirring of the curd grains and temperature changes in the cheese vat effect the amount of water that is expelled. Elaborate on how pH at whey drainage affects calcium level in the cheese.

\subsubsection*{Answer A}

\begin{enumerate} 
    \item \textbf{Cutting the Milk Gel:} Cutting the coagulum into smaller grains accelerates syneresis by significantly increasing the surface area available for whey escape and reducing the distance (l) the whey must travel to reach the grain surface, see equation \ref{eq:ex_answer_darcys_equation}. Finer cuts lead to higher flow velocities (v) and lower moisture content in the final cheese. See figure \ref{fig:exam_question_1.2_figure} for a schematic illustration of the cutting effect.
    \item \textbf{Stirring of the Curd Grains:} Agitation prevents the grains from matting or sedimenting, which would otherwise impede syneresis. Stirring also applies external pressure ($\text{p}_{ex}$) through collisions between curd grains and against vat walls, increasing the total pressure (p) driving the whey out. 
    \item \textbf{Temperature Changes:} Increasing the temperature of the curd-whey mixture (cooking/scalding) promotes syneresis by increasing hydrophobic interactions between casein molecules. This leads to faster network rearrangements, a higher endogenous syneresis pressure, and a more rapid increase in the permeability of the gel. 
\end{enumerate}

\subsubsection*{Answer B}
The pH level at the time of whey drainage is the primary determinant of the total calcium-to-casein ratio in the final cheese. In native milk, a large portion of calcium exists as colloidal calcium phosphate (CCP), which acts as a structural cross-link within the micelles. As the pH decreases during manufacture due to the fermentation of lactose to lactic acid, this CCP gradually solubilizes into the serum phase.

\vspace{1em}
Because the whey removed during drainage represents the majority of the liquid lost during the process, any calcium that has been solubilized is permanently removed from the curd. Consequently, a lower pH at drainage results in lower retained calcium and a lower buffering capacity in the finished cheese. High-calcium cheeses (drained at high pH, e.g., Emmental) tend to be more elastic, while low-calcium cheeses (drained at lower pH, e.g., Feta or Cheddar) are typically more short or brittle.

\vspace{0.5em}
\begin{figure}[h]
    \centering
    \resizebox{0.85\textwidth}{!}{%
    \begin{tikzpicture}[line width=2pt]
            
    % --- Labels
    \node at (2,6,0) {\textbf{Milk gel before cutting}};
    \node at (2,5.5,0) {Surface area = $6 \times 4^2 = 96 cm^2$};
    \node at (9,6,0) {\textbf{Milk gel after cutting}};
    \node at (9,5.5,0) {Surface area = $8 \times 6 \times 2^2 = 192 cm^2$};
    
    % --- Big cube: 4x4x4 (scale factor 2)
    \begin{scope}[scale=2]
      \perfectcube
    \end{scope}
    \draw[->, line width=1pt]
    (1,-1,2) -- (3.2,-1,2)
    node[midway, below] {$l_{big_cube}=2cm$};
    
    % --- Four small cubes: 2x2x2 (aligned in height with big cube)
 
    % top row
    \begin{scope}[shift={(7,2,0)}]
      \perfectcube
    \end{scope}
        \draw[->, line width=1pt]
        (8,-1.2,2) -- (9,-1.2,2)
        node[midway, below] {$l_{small_cubes}=1cm$};
    \begin{scope}[shift={(10,2,0)}]
      \perfectcube
    \end{scope}
    
    % bottom row
    \begin{scope}[shift={(7,-1,0)}]
      \perfectcube
    \end{scope}
    \begin{scope}[shift={(10,-1,0)}]
      \perfectcube
    \end{scope}

    \end{tikzpicture}}
    
    \caption{Schematic illustration of the cutting effect of curd grains on syneresis. The left cube is uncut, while the right sides shows the smaller cubes. "l" is the distance whey must travel from the interior to reach the surface.}
    \label{fig:exam_question_1.2_figure}
\end{figure}


\newpage
\section{Influence of Type of Coagulants on Cheese Manufacture and Ripening}
\subsection*{2.1}
Coagulants from different sources (animals, microorganisms, plants) can be used in the production of cheese. \textbf{A)} Give examples of different types of commonly used coagulant preparations and \textbf{B)} describe differences in their primary action on casein in the cheese vat as well as \textbf{C)} what kind of differences the different coagulants may cause during ripening?

\subsubsection*{Answer A}
\textbf{Common Types of Coagulant Preparations}

Coagulants are categorized by their origin and have transitioned from traditional animal extracts to highly specified recombinant enzymes: 

\begin{enumerate} 
    \item \textbf{Animal Rennet:} Traditionally extracted from the abomasum of neonatal mammals (calves, lambs, or kids). It primarily consists of \textbf{chymosin} (80-95\%) and \textbf{bovine pepsin} (5-20\%). 
    \item \textbf{Microbial/Fungal Coagulants:} Derived from fungi such as \textit{Rhizomucor miehei}, \textit{Rhizomucor pusillus}, and \textit{Cryphonectria parasitica}. These are widely used due to lower costs but vary significantly in heat stability and proteolytic specificity. 
    \item \textbf{Fermentation-Produced Chymosin (FPC):} Produced by genetically modified organisms (e.g., \textit{Aspergillus niger} or \textit{Kluyveromyces lactis}) into which the bovine or camel chymosin gene has been inserted. FPC is considered the "ideal" milk-clotting enzyme due to its high purity. 
    \item \textbf{Vegetable Coagulants:} Extracted from plants, most successfully the flowers of the cardoon thistle (\textit{Cynara cardunculus}), used for specific ewe’s milk cheeses. 
\end{enumerate}

\vspace{1em}
For a schematic overview of coagulants, see table \ref{tab:exam_question_2.1_table}.
\begin{table}[h]
    \centering
    \caption{Classification of coagulants used in cheesemaking based on origin, enzymatic composition, and technological properties.}
    \label{tab:exam_question_2.1_table}
    \rowcolors{2}{white}{gray!7}
    \begin{tabular}{ p{2.5cm} | p{4cm} | p{3cm} | p{4.5cm} }
        \textbf{Coagulant type} 
        & \textbf{Biological origin / source} 
        & \textbf{Main enzyme(s)} 
        & \textbf{Key characteristics / typical use} \\ 
        \hline

        Animal rennet
        & \makecell[l]{Extracted from the aboma- \\ sum of neonatal mammals \\ (calves, lambs, kids).}
        & \makecell[l]{Chymosin (80-95\%) \\ Bovine pepsin - \\ (5-20\%).}
        & \makecell[l]{High milk-clotting specificity \\ with low general proteolysis. \\ Considered the traditional \\ reference coagulant.} \\

        \makecell[l]{Microbial / \\ fungal coagulants}
        & \makecell[l]{Produced by fungi such as \\ \textit{Rhizomucor miehei}, \\ \textit{Rhizomucor pusillus}, \\ \textit{Cryphonectria parasitica}.}
        & \makecell[l]{Aspartic proteinases \\ of microbial origin.}
        & \makecell[l]{Lower cost alternatives to \\ animal rennet; wider variation \\ in heat stability and proteolytic \\ specificity.} \\

        \makecell[l]{Fermentation- \\ produced \\ chymosin (FPC)}
        & \makecell[l]{Produced by genetically \\ modified microorganisms \\ (e.g. \textit{Aspergillus niger}, \\ \textit{Kluyveromyces lactis}).}
        & \makecell[l]{Bovine or camel \\ chymosin.}
        & \makecell[l]{High purity and consistent \\ activity; regarded as the \\ ideal milk-clotting enzyme \\ in modern cheesemaking.} \\

        Vegetable coagulants
        & \makecell[l]{Extracted from plants, \\ mainly cardoon thistle \\ (\textit{Cynara cardunculus}).}
        & \makecell[l]{Plant-derived \\ aspartic proteinases.}
        & \makecell[l]{Used in specific traditional \\ cheeses, especially ewe's \\ milk cheeses; often associated \\ with higher proteolysis.} \\

    \end{tabular}
\end{table}


\subsubsection*{Answer B}
\textbf{Primary Action on Casein in the Cheese Vat}

The primary role of a coagulant is to destabilize the casein micelle. In the cheese vat, most coagulants (chymosin, FPC, and \textit{Rhizomucor} spp.) act by \textbf{selectively hydrolysing the $\text{Phe}_{105} - \text{Met}_{106}$ peptide bond of $\kappa$-casein}. (Note: \textit{C. parasitica} is an exception, cleaving the $\text{Ser}_{104} - \text{Phe}_{105}$ bond). This cleavage releases the hydrophilic \textbf{caseinomacropeptide (CMP/GMP)} into the whey, removing the micelle's "hairy layer" and reducing the zeta potential. This loss of steric and electrostatic repulsion allows the sensitized micelles to aggregate into a 3D viscoelastic gel network.

\subsubsection*{Answer C}
\textbf{Differences During Ripening}

Coagulant activity continues long after the curd is formed, as 0-15\% (up to 30\% for chymosin at low pH) of the enzyme is retained in the curd. 
\begin{enumerate} 
    \item \textbf{Substrate Specificity:} Chymosin is responsible for the \textbf{initial hydrolysis of $\alpha$-casein} (specifically at $\text{Phe}_{23} - \text{Phe}_{24}$), which is vital for softening the cheese texture. Microbial coagulants generally exhibit higher activity on $\beta$-casein. 
    \item \textbf{Bitterness Risk:} Coagulants with a low ratio of clotting-to-proteolytic activity (e.g., some microbial and vegetable rennets) carry a high risk of developing bitter off-flavours due to the excessive or non-specific accumulation of bitter peptides. 
    \item \textbf{Camel Chymosin:} Exhibits the highest known specificity, significantly reducing bitterness during ripening compared to bovine chymosin. 
    \item \textbf{High-Cook Cheeses:} In varieties like Emmental or Parmesan, the high cooking temperatures partially or fully inactivate chymosin, making \textbf{plasmin} the dominant agent for primary proteolysis. 
\end{enumerate}


\subsection*{2.2}
\vspace{1em}
Describe what is meant by clotting activity versus proteolytic activity of a coagulant. What is the optimal proteolytic activity of a coagulant?

\subsubsection*{Answer}
Clotting (C) vs. Proteolytic (P) Activity 
\begin{enumerate} 
    \item \textbf{Clotting Activity (C):} Refers to the enzyme's specific ability to hydrolyse $\kappa$-casein to induce coagulation. It is measured in \textbf{IMCU} (International Milk Clotting Units). 
    \item \textbf{Proteolytic Activity (P):} Refers to non-specific hydrolysis of other bonds in caseins (e.g., $\alpha_{s1}$ - and $\beta$-casein). 
    \item \textbf{Optimal Activity:} The optimal proteolytic activity is represented by a high C/P ratio. A high ratio ensures a firm coagulum and high yield while protecting the cheese from flavour and texture defects. Chymosin is the benchmark for this high specificity. 
\end{enumerate}

\subsection*{2.3}
\vspace{1em}
How could the different activity of coagulants during cheese ripening be investigated?

\subsubsection*{Answer}
\textbf{Investigating Coagulant Activity During Ripening}

Different activity patterns can be investigated through biochemical and physical analysis: 
\begin{enumerate} 
    \item \textbf{Nitrogen Fractionation:} Measuring Nitrogen content in fractions such as pH 4.6 soluble N (index of primary proteolysis) or 12\% TCA and 5\% PTA soluble N (index of secondary proteolysis and amino acids). 
    \item \textbf{Urea-PAGE and Capillary Electrophoresis (CE):} These techniques allow for the visualization and quantification of the degradation of intact $\alpha_{s1}$ - and $\beta$-caseins into their respective breakdown products (e.g., $\alpha_{s1}$ -I and $\gamma$-caseins). 
    \item \textbf{RP-HPLC and LC-MS:} Used to create peptide profiles and identify specific peptides produced by different coagulants. 
    \item \textbf{Sensory/Rheological Assessment:} Correlating bitterness scores with peptide accumulation or monitoring fracture stress/firmness as the protein matrix degrades. 
\end{enumerate}


\newpage
\section{Cheese Yield}
The cheese yield varies due to a number of cheese making parameters, and it varies between different cheese types.

\subsection*{3.1}
\vspace{1em}
A) Describe which major milk components of the cheese milk that are retained, and which are not retained in the cheese, and describe why they are retained or not retained. B) Elaborate on the difference between Cheese Yield and Moisture Adjusted Cheese Yield (MACY).

\subsubsection*{Answer A}
\textbf{Retention of Milk Components} 

Cheese manufacture is a dehydration process where milk nutrients are concentrated approximately tenfold. The retention of components is governed by their physical state and solubility: 
\begin{enumerate} 
    \item \textbf{Casein:} Typically \textasciitilde 76\% is retained. As the structural basis of the gel, insoluble para-casein remains in the matrix while the hydrophilic glycomacropeptide (GMP) is released into the whey after renneting. 
    \item \textbf{Fat:} Retained at high levels (85-95\%) because fat globules (0.5-10 $\mu$m) are physically entrapped within the narrowing pores of the casein network. Losses occur if globules are mechanically damaged ("free fat") or located on the cut surfaces of curd grains. 
    \item \textbf{Whey Proteins:} Predominantly not retained as they are soluble at cheesemaking pH and temperatures. However, they can be included if denatured via heat, causing them to complex with $\kappa$-casein. 
    \item \textbf{Lactose:} Only 3-5\% is retained. Being fully soluble, it remains in the whey; the small portion remaining in the curd is eventually fermented into lactic acid. 
    \item \textbf{Minerals and Citrate:} Ash retention is \textasciitilde 35-40\% in rennet cheeses. Colloidal calcium phosphate remains associated with the casein, while soluble minerals partition into the whey. 
\end{enumerate}

\subsubsection*{Answer B}
\textbf{Actual Yield vs. Moisture Adjusted Cheese Yield (MACY)} 

Actual cheese yield (Y$_a$) is the quantity of cheese produced from a defined weight of milk, often expressed as kg cheese per 100 kg milk. While useful for immediate production weights, Y$_a$ is limited because it does not account for variations in moisture or milk composition.
Moisture Adjusted Cheese Yield (MACY) is a theoretical value that normalizes the yield to a reference moisture content (e.g., 37\% for Cheddar). This allows for a meaningful comparison of efficiency across different batches or factories where moisture levels may fluctuate. The formula is given in equation \ref{eq:ex_03_adjustet_moisture}: 
\begin{equation}
    \text{Y}_{ma}=\text{Y}_{a} \times \frac{100-\text{Actual Moisture} \%}{100-\text{Reference Moisture} \%}
    \label{eq:ex_03_adjustet_moisture}
\end{equation}


\subsection*{3.2}
\vspace{1em}
Describe how milk quality parameters can affect cheese yield. Describe industrial methods/ingredients which may be used to increase cheese yield.

\subsubsection*{Answer A}
\textbf{Effect of Milk Quality Parameters} 
\begin{enumerate} 
    \item \textbf{Somatic Cell Count (SCC):} Elevated SCC (above 100,000 cells/mL) significantly reduces yield. High SCC milk contains endogenous proteases (plasmin) that degrade casein into soluble peptides that are lost in the whey. 
    \item \textbf{Protein Genotypes:} The BB genotype of $\kappa$-casein is associated with higher casein levels and superior renneting properties, increasing MACY by 3-8\% compared to the AA variant. 
    \item \textbf{Cold Storage:} Storing raw milk below 5\textdegree C can lead to the dissociation of $\beta$-casein from micelles and the growth of psychrotrophs. These bacteria produce heat-stable proteases and lipases that reduce component recovery and yield. 
\end{enumerate}

\subsubsection*{Answer B}
\textbf{Industrial Methods to Increase Yield} 
\begin{enumerate} 
    \item \textbf{Ultrafiltration (UF):} UF concentrates all milk proteins, including whey proteins, which are then retained in the curd. It also creates a firmer coagulum, improving fat and casein retention. 
    \item \textbf{Microparticulated Whey Protein (MWP):} Commercial products like LeanCreme\texttrademark\ use high heat and shear to create whey particles similar in size to fat globules. This can increase yield by 6-10\% through protein incorporation and increased water binding. 
    \item \textbf{Enzymatic Additives:} Phospholipase (e.g., YieldMAX) improves fat recovery by modifying the fat globule membrane. Transglutaminase can be used to cross-link proteins, increasing moisture retention. 
    \item \textbf{Heat Treatment:} Heating milk above 72\textdegree C denatures whey proteins, causing them to complex with the casein matrix and increasing total protein recovery. 
\end{enumerate}


\newpage
\section{Importance of pH Development in Cheese Ripening}
During the first day(s), pH in cheese decreases to a minimum and then it increases at different rates depending on cheese variety.

\subsection*{4.1}
\vspace{1em}
Describe how different pH minima (the lowest pH obtained during cheese making) may be obtained, and give examples on cheese varieties with low, medium and high minimum-pH.

\subsubsection*{Answer}

The pH minimum of a cheese is primarily determined by the ratio of residual lactose content to the buffering capacity of the curd, which consists mainly of proteins and inorganic phosphate. This acidification process occurs in two distinct phases:
\begin{itemize}
    \item \textbf{Phase 1:} Occurs before whey separation, where lactose is continuously fermented into lactic acid by starter bacteria. The pH drops rapidly due to the high availability of lactose diffusing from the surrounding whey into the curd grains.
    \item \textbf{Phase 2:} Begins after whey separation, where the rate of pH decrease slows down significantly. The further drop in pH is determined solely by the amount of lactose remaining in the curd relative to its buffering substances.
\end{itemize}

Additionally, high cooking temperatures can be used to inhibit the starter bacteria, slowing the rate of acid production and resulting in a higher minimum pH. Examples of minimum-pH levels include: 
\begin{enumerate} 
    \item \textbf{Low Minimum-pH ($\approx$ 4.6-4.7):} Feta, Camembert, and Danablu. 
    \item \textbf{Medium Minimum-pH ($\approx$ 4.8-5.2):} Cheddar (4.75-4.95), Gouda (5.20-5.25), and Danbo (5.20-5.25). 
    \item \textbf{High Minimum-pH ($\approx$ 5.3+):} Emmental (5.2-5.30) and Halloumi ($\approx$ 6.1). 
\end{enumerate}

\subsection*{4.2}
\vspace{1em}
Give examples on cheese varieties with low, medium and high final pH (pH of the ripened cheese), respectively, and describe which microbial/enzymatic activities that are involved.

\subsubsection*{Answer}
The final pH of a ripened cheese depends on the specific microbial and enzymatic pathways utilized by the microflora during maturation. Proteolysis contributes to a general increase in pH across most varieties by increasing the buffering capacity of the cheese matrix. Specific examples include: 
\begin{enumerate} 
    \item \textbf{Low Final pH (Cheddar, $\approx$ 5.5):} In Cheddar, NSLAB often convert L-lactate to D-lactate, but the low moisture and storage temperatures generally limit extensive pH increases. 
    \item \textbf{Medium Final pH (Danbo, $\approx$ 6.2-6.3):} Varieties like Danbo and Gouda see a gradual rise in pH as proteolysis releases peptides and amino acids, and the slow metabolism of lactate occurs. 
    \item \textbf{High Final pH (Camembert/Brie, $\approx$ 7.0-8.0):} In these varieties, the oxidative deamination of amino acids by surface moulds and yeasts produces ammonia ($\text{NH}_3$), which significantly neutralizes the acidity. Blue cheeses also reach high final pH levels ($\approx$ 6.5) due to similar ammonia production from mould activity. 
\end{enumerate}


\subsection*{4.3}
\vspace{1em}
Differences in pH between the interior and the surface of cheeses are most pronounced in cheeses with surface ripening. Mention two different cheese varieties with different kind of surface ripening. Explain how pH gradients develop during ripening and how they influence ripening using one of the cheese varieties as an example.

\subsubsection*{Answer}
Two distinct varieties of surface-ripened cheeses include white-mould cheeses, such as \textbf{Camembert} or \textbf{Brie}, and bacterial smear-ripened cheeses, such as \textbf{Limburger} or \textbf{Danbo} with smear. In Camembert, a steep pH gradient develops during ripening because the surface microflora (initially yeasts like \textit{Kluyveromyces lactis} and the mould \textit{Penicillium camemberti}) metabolize lactic acid to $\text{CO}_2$ and $\text{H}_2$. As the surface lactate is exhausted, the mould begins the oxidative deamination of amino acids, producing ammonia ($\text{NH}_3$) which diffuses from the surface into the cheese interior. This process increases the surface pH to $\approx$ 7.0-8.0 while the centre remains more acidic ($\approx$ 4.5-5.0). These gradients influence ripening in the following ways: 
\begin{enumerate} 
    \item \textbf{Mineral Migration:} The neutralization of the surface causes a concentration gradient that makes lactic acid migrate to the surface and calcium phosphate migrate from the interior to precipitate at the surface. 
    \item \textbf{Enzyme Stimulation:} The increase in pH stimulates the activity of the endogenous milk enzyme \textbf{plasmin}, which is most active at neutral pH, thereby accelerating primary proteolysis. 
    \item \textbf{Texture Softening:} The combination of pH increase, the loss of micellar calcium from the interior, and increased proteolysis results in the characteristic extensive softening of the cheese body from the surface toward the centre. 
\end{enumerate}


\newpage
\section{Gas Production in Cheese}
\subsection*{5.1}
Gas may be produced from the metabolism of lactose and citrate. Which are the main microorganisms involved in this metabolism, and in which type of cheeses is this relevant?

\subsubsection*{Answer}
The main microorganisms involved in the production of gas from citrate metabolism are citrate-fermenting \textbf{meso- philic LAB}, specifically \textit{Lactococcus lactis} subsp. \textit{lactis} biovar \textit{diacetylactis} (D-strains) and \textbf{\textit{Leuconostoc}} species such as \textit{L. mesenteroides}. These organisms utilize citrate to produce $\text{CO}_2$, along with flavour compounds like diacetyl and acetate. This process is primarily relevant in \textbf{Dutch-type cheeses} (e.g., Gouda, Edam, Danbo) where the $\text{CO}_2$ is responsible for the formation of characteristic small eyes. It is also relevant in certain \textbf{soft and fresh cheeses} (e.g., Quarg, Cottage cheese, and Fromage frais) where the metabolic by-products contribute to the buttery aroma and specific texture. Gas production from lactose metabolism can occur through the activity of \textbf{coliform bacteria} (e.g., \textit{Escherichia coli}) or \textbf{yeasts}, which produce $\text{CO}_2$ and $\text{H}_2$. This is generally undesired in most ripened varieties but is part of the natural flora in some traditional artisanal cheeses.

\subsection*{5.2}
\vspace{1em}
Gas production in cheese is not always desirable and may be considered as a serious defect. Mention three different types of gas producing reactions (substrate and the organisms involved) in which undesired gas production may occur. What measures are available to the cheese producer to control undesired gas production in cheese?

\subsubsection*{Answer}
Undesired gas production, often referred to as "blowing," can manifest as three distinct types of defects: 
\begin{enumerate} 
    \item \textbf{Early Blowing:} This occurs within 1-2 days of manufacture when \textbf{coliform bacteria} (e.g., \textit{Escherichia coli}) ferment \textbf{lactose} into $\text{H}_2$  and $\text{CO}_2$  gases. 
    \item \textbf{Late Blowing:} This appears later in ripening (weeks or months) and is caused by \textbf{anaerobic spore-forming bacteria}, specifically \textit{Clostridium tyrobutyricum} or \textit{C. butyricum}, which ferment \textbf{lactate} into butyric acid, $\text{H}_2$ , and $\text{CO}_2$ . 
    \item \textbf{Decarboxylation of Amino Acids:} Certain NSLAB or \textit{Lactobacillus} species can decarboxylate \textbf{amino acids} (such as glutamic acid) to produce $\text{CO}_2$ , which leads to late-stage cracks and slits in hard or semi-hard cheeses. 
\end{enumerate}

\vspace{1em}
To control undesired gas production, cheese producers can employ several measures: 
\begin{enumerate} 
    \item \textbf{Milk Pre-treatment:} 
        \begin{itemize}
            \item Using \textbf{pasteurization} effectively eliminates vegetative coliform cells.
            \item \textbf{Bactofugation} or \textbf{microfiltration} of the milk can remove up to 99\% of clostridial spores.
        \end{itemize}     

    \item \textbf{Chemical Additives:} 
        \begin{itemize}
            \item Adding \textbf{potassium or sodium nitrate} ($\text{KNO}_3 / \text{NaNO}_3$) suppresses the growth of coliforms and clostridia by acting as an alternative electron acceptor or through the formation of nitrite. 
            \item \textbf{Lysozyme} can also be added to milk to lyse sensitive clostridial cells.
        \end{itemize}

    \item \textbf{Process Control:} 
        \begin{itemize}
            \item Implementing high hygienic standards at the farm (e.g., avoiding silage feed) and in the factory minimizes initial contamination.
            \item Using active starter cultures ensures a \textbf{rapid pH drop}, which inhibits many spoilage organisms.
            \item Maintaining low ripening temperatures also retards the growth of gas-producers like propionic acid bacteria or clostridia in non-target varieties. 
        \end{itemize}
      
\end{enumerate}

\subsection*{5.3}
\vspace{1em}
In Swiss cheese, eye formation is an important cheese quality parameter. What factors (microbial and physio-chemical) are necessary to ensure that satisfactory eye formation develops?

\subsubsection*{Answer}
Satisfactory eye formation in Swiss-type cheeses (e.g., Emmental) requires a precise combination of microbial and physio-chemical factors: 
\begin{enumerate} 
    \item \textbf{Microbial Factors:} The primary driver is the growth of \textbf{\textit{Propionibacterium freudenreichii}}, which ferments the lactate produced by the starter into propionate, acetate, and large amounts of \textbf{$\text{CO}_2$ gas}. The rate and quantity of gas production must be high enough (e.g., exceeding 2 L/day in an 80 kg Emmental) to reach the necessary internal overpressure to expand the eyes. 
    \item \textbf{Nuclei (Primers):} For eyes to form, the cheese must contain \textbf{nuclei} or "weak points," which are typically microscopic air bubbles or irregularities between curd particles where the gas can collect and grow. 
    \item \textbf{Texture and Cohesion:} The cheese matrix must have appropriate \textbf{elasticity and elongational viscosity} to allow the holes to expand into round shapes without splitting. This is achieved through high cooking temperatures ($\approx$ 55\textdegree C) and a high pH at whey drainage ($\approx$ 6.4), which \textbf{retains colloidal calcium phosphate}, acting as a buffering agent and providing a rubbery, elastic structure. 
    \item \textbf{Environmental Conditions:} Ripening in a \textbf{warm room} (18-24\textdegree C) is necessary to stimulate the PAB growth while the cheese is still flexible. 
    \item \textbf{Physical Mass:} The cheese must be \textbf{large enough} (e.g., 80-100 kg wheels) to prevent excessive $\text{CO}_2$ from diffusing out too quickly, thereby maintaining sufficient internal pressure for eye growth. 
\end{enumerate}


\newpage
\section{Cheese Proteolysis}
\subsection*{6.1}
Proteolysis is a fundamental process in cheese ripening. List enzymes that hydrolyse peptide bonds in intact casein molecules in rennet coagulated semi-hard cheeses, describe the specificity of these enzymes and give the first peptides that are produced.

\subsubsection*{Answer}
In rennet-coagulated semi-hard cheeses, two principal enzymes are responsible for the hydrolysis of intact casein molecules: chymosin (from the rennet) and plasmin (an endogenous milk enzyme). 
\begin{enumerate} 
    \item \textbf{Chymosin (EC 3.4.23.4):} This is an aspartyl proteinase with a low pH optimum. In the ripening of semi-hard cheese, its primary action is the hydrolysis of \textbf{$\alpha_{s1}$-casein}, specifically at the \textbf{$\text{Phe}_{23}-\text{Phe}_{24}$} peptide bond. 
    \item \textbf{Plasmin (EC 3.4.21.7):} This is an alkaline serine proteinase with a neutral pH optimum that shows high specificity for peptide bonds following the basic amino acids \textbf{Lysine} and \textbf{Arginine}. Its preferred substrate is \textbf{$\beta$-casein}. 
\end{enumerate}

\vspace{1em}
The first peptides produced by these enzymes are: 
\begin{itemize} 
    \item \textbf{From Chymosin:} $\alpha_{s1}$-casein (f1-23) and $\alpha_{s1}$-casein (f24-199). 
    \item \textbf{From Plasmin:} $\gamma$-caseins ($\gamma_1$ f29-209, $\gamma_2$ f106-209, $\gamma_3$ f108-209) and various proteose-peptones (e.g., PP5, PP8-slow, and PP8-fast). 
\end{itemize}

\subsection*{6.2}
\vspace{1em}
Explain how the first peptide released from $\alpha_{S1}$-casein by chymosin in cheese ($\alpha_{S1}$-casein (f1-23)), may be further hydrolysed all the way to amino acids, and mention which enzymes that are involved and what the origin of these enzymes may be.

\subsubsection*{Answer}
The peptide \textbf{$\alpha_{s1}$-casein (f1-23)} is further degraded into free amino acids through the coordinated action of the \textbf{starter LAB} proteolytic system. 
\begin{enumerate} 
    \item \textbf{Initial Hydrolysis (Extracellular):} The \textbf{cell-envelope associated proteinase (CEP)} (also known as \textbf{lactocepin} or \textbf{PrtP}) anchored to the starter LAB cell wall hydrolyses the f1-23 peptide into smaller oligopeptides. Depending on the strain specificity, typical products include fragments such as f1-13, f1-14, f1-16, f1-8, and f1-9. 
    \item \textbf{Transport:} These oligopeptides (typically up to ~8 amino acids long) are then taken up by the bacterial cell through the \textbf{oligopeptide transport system (Opp)}. 
    \item \textbf{Final Hydrolysis (Intracellular):} Once inside the cytoplasm, a battery of \textbf{intracellular peptidases} acts in concert to hydrolyse the peptides into free amino acids. Key enzymes include \textbf{aminopeptidases} (PepN, PepC), \textbf{endopeptidases} (PepO, PepF), and \textbf{proline-specific peptidases} (PepX, PepQ, PepR) which are essential due to the high proline content of casein. 
\end{enumerate} 

\vspace{1em}
The origin of these enzymes is the \textbf{starter culture} (e.g., \textit{Lactococcus lactis}, \textit{Lactobacillus helveticus}) and NSLAB that release these enzymes upon cell \textbf{lysis}.

\subsection*{6.3}
\vspace{1em}
Give examples of two cheese types with different primary proteolysis (first attack on intact casein) and explain how the differences are related to cheese manufacture. Elaborate on suitable laboratory methods for measuring proteolysis in cheese.

\subsubsection*{Answer}
Two examples of cheeses with different primary proteolysis are \textbf{Cheddar} and \textbf{Emmental}: 
\begin{itemize} 
    \item \textbf{Cheddar:} Primary proteolysis is dominated by \textbf{chymosin} activity on $\alpha_{s1}$-casein. This occurs because Cheddar is cooked at a moderate temperature ($\sim$39-40\textdegree C), which does not inactivate the rennet. 
    \item \textbf{Emmental:} Primary proteolysis is dominated by \textbf{plasmin} activity on $\beta$-casein. In this variety, high cooking temperatures ($\sim$53-55\textdegree C) partially or completely inactivate chymosin, while plasmin is heat-stable and its activity is actually stimulated by the activation of plasminogen at these temperatures. 
\end{itemize}

\vspace{1em}
Laboratory methods for measuring proteolysis include: 
\begin{itemize} 
    \item \textbf{Nitrogen Fractionation:} Measuring Nitrogen in different fractions such as \textbf{pH 4.6 soluble N} (general index of ripening), \textbf{12\% TCA soluble N} (small peptides), and \textbf{5\% PTA soluble N} (free amino acids). 
    \item \textbf{Electrophoresis (Urea-PAGE):} Used to visualize the breakdown of intact $\alpha_{s1}$- and $\beta$-caseins into their primary products. 
    \item \textbf{Capillary Electrophoresis (CE):} Provides quantitative analysis of intact caseins and large degradation products. 
    \item \textbf{HPLC and LC-MS:} Used to create peptide profiles and identify specific peptides or quantify individual free amino acids. 
\end{itemize}

\subsection*{6.4}
\vspace{1em}
What approaches can be used to accelerate proteolysis and what are the advantages/disadvantages of each approach?

\subsubsection*{Answer}

Several approaches can be used to accelerate proteolysis in cheese: \begin{enumerate} 
    \item \textbf{Exogenous General Proteinases:} Adding fungal or bacterial proteinases (e.g., from \textit{Aspergillus} or \textit{Bacillus}) to the milk or curd. 
        \begin{itemize} 
            \item \textit{Advantage:} Rapidly accelerates the initial breakdown of casein. 
            \item \textit{Disadvantage:} High risk of unbalanced flavour, over-ripening, and the development of bitterness. 
        \end{itemize} 
    
    \item \textbf{Attenuated Starter/Adjunct Cultures:} Adding LAB cells that have been treated (e.g., by heat or freeze-shocking) to prevent acid production while allowing them to lyse and release peptidases early. 
        \begin{itemize} 
            \item \textit{Advantage:} Enhances flavour and reduces bitterness by converting peptides to amino acids without the risk of over-softening the texture. 
            \item \textit{Disadvantage:} Adds cost to the manufacturing process. 
        \end{itemize} 
    
    \item \textbf{Elevated Ripening Temperatures:} Increasing the storage temperature for a period during maturation. 
        \begin{itemize} 
            \item \textit{Advantage:} Simple and inexpensive way to speed up all biochemical reactions. 
            \item \textit{Disadvantage:} Can lead to the growth of undesirable non-starter bacteria and defects in eye formation or texture. 
        \end{itemize} 
    
    \item \textbf{High-Pressure (HP) Treatment:} Subjecting young cheese to high pressure. 
        \begin{itemize} 
            \item \textit{Advantage:} Induces bacterial lysis and modifies the casein matrix to make it more accessible to enzymes. 
            \item \textit{Disadvantage:} Requires expensive equipment and can alter the desired texture. 
        \end{itemize} 
\end{enumerate}


\newpage
\section{Amino Acids Catabolism and Cheese Flavour Formation}
\subsection*{7.1}
Amino acids are released from casein derived peptides in cheese during ripening and used to different extents by different microorganisms. Describe the enzymes that are directly involved in amino acid release incl. their name, specificity and origin.

\subsubsection*{Answer}
Amino acid release in cheese is a multi-stage process involving extracellular and intracellular enzymes originating primarily from \textbf{starter and non-starter LAB}. 
\begin{enumerate} 
    \item \textbf{Lactocepin (PrtP):} A \textbf{cell-envelope associated proteinase} anchored to the cell wall of LAB. 
    \begin{itemize} 
        \item \textbf{Origin:} Starter and non-starter LAB. 
        \item \textbf{Specificity:} It is the only extracellular enzyme capable of hydrolysing large peptides (produced initially by the coagulant or plasmin) into \textbf{oligopeptides}, typically containing 4-30 residues. For example, it degrades the $\alpha_{s1}$-casein (f1-23) fragment produced by chymosin into smaller pieces. 
    \end{itemize}
    \item \textbf{Intracellular Peptidases:} These enzymes are released into the cheese matrix following \textbf{bacterial cell lysis}. They act in concert to reduce small oligopeptides into free amino acids. 
    \item Specific types include: 
    \begin{itemize} 
        \item \textbf{Aminopeptidases (e.g., PepN, PepC, PepA):} These have a broad specificity for releasing single amino acids from the N-terminal of peptides. 
        \item \textbf{Proline-specific peptidases (e.g., PepX, PepQ, PepR, PepI, PepP):} These are essential for \textbf{de-bittering} cheese because casein is rich in proline; they hydrolyse specific bonds involving proline that other enzymes cannot. 
        \item \textbf{Di- and Tripeptidases (e.g., PepV, PepT):} These are specific for the final cleavage of di- and tri-peptides into free amino acids. 
    \end{itemize} 
\end{enumerate}

\subsection*{7.2}
\vspace{1em}
Amino acids may be catabolized to aroma compounds in cheese. What are the main pathways (and the enzymes involved) in the catabolism of amino acids?

\subsubsection*{Answer}
The catabolism of amino acids is considered the \textbf{rate-limiting step} in cheese flavour formation. The main pathways and enzymes include: 
\begin{enumerate} 
\item \textbf{Transamination (Primary Pathway):} This is the initial step for most amino acids in cheese. Enzymes called \textbf{aminotransferases (AT)} transfer the amino group from an amino acid to an acceptor molecule (principally $\alpha$-ketoglutarate) to produce an \textbf{$\alpha$-keto acid} and glutamate. 
    \item \textbf{Decarboxylation:} Performed by \textbf{decarboxylases}, this pathway converts amino acids into \textbf{amines} and $\text{CO}_2$. While some amines contribute to putrid off-flavours, others like GABA (from glutamate) are produced by LAB to regulate internal pH. 
    \item \textbf{Deamination:} Performed by \textbf{deaminases}, resulting in the production of \textbf{ammonia ($\text{NH}_3$)} and keto acids. This is prominent in surface-ripened cheeses where ammonia increases pH and influences texture. 
    \item \textbf{Elimination/Lyase Activity:} \textbf{Lyases} (e.g., methionine-$\gamma$-lyase) are particularly important for methionine catabolism, producing methanethiol, which leads to potent \textbf{sulphur compounds}. 
    \item \textbf{Dehydration:} \textbf{Dehydratases} act anaerobically on amino acids containing an -OH or -SH group (such as Serine or Threonine) to produce pyruvate and ammonia. 
\end{enumerate}

\subsection*{7.3}
\vspace{1em}
Which are the four groups of amino acids, which breakdown products may contribute to the cheese flavours 

\begin{enumerate}
    \item Malty
    \item Floral
    \item Cooked cabbage/garlic
    \item Buttery
\end{enumerate}

respectively?

\subsubsection*{Answer:}

\begin{table}[h]
    \centering
    \caption{Amino acid groups and typical flavour-active breakdown products in cheese.}
    \label{tab:exam_question_7.3_table}
    \rowcolors{2}{white}{gray!7}
    \begin{tabular}{ p{3cm} | p{5cm} | p{6cm} }
        \textbf{Flavour} & \textbf{Amino acid group} & \textbf{Key breakdown products} \\ 
        \hline
        Malty 
        & \makecell[l]{Branched-chain amino acids \\ (Leu, Ile, Val)}
        & \makecell[l]{Aldehydes, alcohols and acids \\ (e.g. 3-methylbutanal)} \\

        Floral 
        & \makecell[l]{Aromatic amino acids \\ (Phe, Tyr, Trp)} 
        & Aromatic aldehydes, alcohols and esters \\

        \makecell[l]{Cooked cabbage / \\ garlic}
        & \makecell[l]{Sulphur-containing amino acids \\ (Met, Cys)} 
        & \makecell[l]{Volatile sulphur compounds \\ (e.g. methanethiol, dimethyl disulphide)} \\

        Buttery 
        & \makecell[l]{Acidic amino acids \\ (Asp, Glu)} 
        & Diacetyl and related carbonyl compounds \\
    \end{tabular}
\end{table}

\subsection*{7.4}
\vspace{1em}
Which analytical method can be used for identification of aroma compounds in cheese?

\subsubsection*{Answer}
The standard analytical technique for the identification of aroma compounds is \textbf{Gas Chromatography-Mass Spectrometry (GC-MS)}. Because aroma compounds are volatile and present in very low concentrations (ppb to ppm), they must first be isolated and concentrated using techniques such as \textbf{Solid Phase Micro Extraction (SPME)} or \textbf{Dynamic Headspace Sampling (DHS)}.

\vspace{1em}
To determine which of these thousands of volatiles are actually \textbf{odour-active}, \textbf{Gas Chromatography-Olfactometry (GC-O)} is used. In GC-O, the human nose acts as the detector at a sniffing port to record and describe odours as they elute from the GC column.



\newpage
\section{Importance of Milk Fat in Cheese Flavour}
Low-fat cheese commonly lacks the complex and balanced flavour of normal-fat cheese.

\subsection*{8.1}
\vspace{1em}
A) Describe three different roles that milk fat may have in cheese flavour and B) describe how the off-flavour bitterness is related to fat content in cheese.

\subsubsection*{Answer A}
Milk fat is not merely a structural component; it is a vital contributor to the sensory profile of cheese through three primary roles: 
\begin{enumerate} 
    \item \textbf{Source of Flavour Precursors:} Milk fat (triglycerides) undergoes lipolysis to release free fatty acids (FFAs). These FFAs either contribute directly to flavour (pungency, piquancy) or act as precursors for other potent aroma compounds like methyl ketones, lactones, and esters. 
    \item \textbf{Solvent and Masking Agent:} Fat acts as a solvent for lipophilic aroma compounds produced from other metabolic pathways, such as proteolysis and glycolysis. By acting as a solvent, it regulates the concentration and release of these aromas during consumption. 
    \item \textbf{Textural Plasticiser:} Fat functions as a plasticiser in the cheese matrix. It interrupts the dense para-casein network, providing a creamy mouthfeel and smoothness that influences how flavours are perceived. 
\end{enumerate}

\subsubsection*{Answer B}
Bitterness in cheese is caused by the accumulation of hydrophobic peptides. In normal-fat cheese, the fat has a masking ability that shields the taste buds from these bitter compounds. In low-fat cheese, the reduction in fat leads to a more compact protein matrix with higher protein-to-fat ratios, often resulting in a higher concentration of bitter peptides that are no longer masked, making the defect much more prominent.

\subsection*{8.2}
\vspace{1em}
During ripening, fat is degraded to different extent in different cheese varieties from less than 1\% to about 10\% or more. A) Which lipolytic enzymes are involved in different kinds of cheeses, and where do they origin from? B) Which kind of aroma compounds may be produced from lipolysis and further catabolism of the fatty acids released, and what flavour notes do each of these aroma compounds introduce.

\subsubsection*{Answer A}
Lipolytic Enzymes and Origins: The extent of lipolysis depends on the specific enzymes present in the cheese: 
\begin{enumerate} 
    \item \textbf{Lipoprotein Lipase (LPL):} An endogenous milk enzyme. It is significant primarily in raw milk cheeses as it is extensively inactivated by pasteurization. 
    \item \textbf{Pregastric Lipase (PGE):} Originates from rennet paste (macerated stomachs of lambs or kids). It is essential for the sharp, piquant flavour in Italian cheeses like Provolone and Pecorino. 
    \item \textbf{Microbial Lipases/Esterases:} These originate from starter cultures (LAB/PAB), NSLAB, or moulds (\textit{P. roqueforti} and \textit{P. camemberti}). Mould lipases are particularly potent, driving the extensive lipolysis in Blue and Camembert varieties. 
    \item \textbf{Psychrotrophic Lipases:} Heat-stable enzymes from bacteria like \textit{Pseudomonas} that survive pasteurization and can cause unclean off-flavours. 
\end{enumerate}

\subsubsection*{Answer B}
Aroma Compounds and Flavour Notes: 
\begin{itemize} 
    \item \textbf{Short-chain Fatty Acids (e.g., Butyric, Caproic):} Introduce sharp, pungent, and piquant notes. 
    \item \textbf{Methyl Ketones (e.g., 2-heptanone):} Provide the characteristic mouldy/blue cheese flavour. 
    \item \textbf{Lactones:} Introduce caramel or coconut notes. 
    \item \textbf{Ethyl Esters:} Result in fruity or pineapple flavour notes. 
    \item \textbf{Thioesters:} Contribute cooked cabbage or cauliflower notes, common in surface-ripened cheeses. 
\end{itemize}

\subsection*{8.3}
\vspace{1em}
A) Give examples of compounds from milk fat catabolism that contribute to the characteristic flavour of the two cheeses, Parmesan and Blue cheese.
B) How can lipolysis be enhanced in cheese during ripening?

\subsubsection*{Answer A}
Characteristic Compounds:
\begin{itemize}
    \item Parmesan: The flavour is dominated by short- and medium-chain fatty acids (C4:0 to C12:0), specifically butanoic (butyric) and hexanoic acids, which provide its piquant, sharp character. Long-ripened Parmesan also features a high concentration of ethyl esters providing fruity notes.
    \item Blue Cheese: The primary flavour compounds are methyl ketones (especially 2-heptanone and 2-nonanone), which give the classic "blue mould" aroma. It also contains high levels of octanoic and hexanoic acids.
\end{itemize}


\subsubsection*{Answer B}
Methods to Enhance Lipolysis: 
\begin{enumerate} 
    \item \textbf{Homogenisation of Cheese Milk:} This breaks fat globules into smaller sizes, increasing the surface area available for lipases to act at the oil-water interface. 
    \item \textbf{Use of Rennet Paste:} Incorporating rennet paste (rather than liquid extract) adds pregastric lipases directly to the vat. 
    \item \textbf{Reduced Heat Treatment:} Using lower pasteurization temperatures (e.g., 65\textdegree C) preserves more of the endogenous milk lipase (LPL). 
    \item \textbf{Exogenous Lipases or Adjunct Cultures:} Adding commercial lipase preparations or selecting highly lipolytic mould strains (like specific \textit{P. roqueforti} variants). 
\end{enumerate}


\newpage
\section{Structure Formation in Fermented Milk Products}
During acidification of milk, e.g. during fermentation using a standard culture of lactic acid bacteria for yoghurt production, several changes occur in the milk components, leading to formation of the final structure of fermented milk products.

\subsection*{9.1}
\vspace{1em}
Describe the changes occurring in the casein micelle as a result of acidification.

\subsubsection*{Answer}
The transformation of liquid milk into a fermented gel is driven by the destabilization of the casein micelle as the pH drops from 6.7 toward the isoelectric point of casein (pH 4.6). This process involves several distinct physical and chemical changes:
\begin{enumerate} 
    \item \textbf{Solubilization of Colloidal Calcium Phosphate (CCP):} In native milk, approximately 66\% of calcium is bound to micelles as CCP, acting as a structural "cement". As pH decreases, CCP gradually solubilizes into the serum phase. It is completely dissolved by the time the milk reaches pH 5.0-5.3. 
    \item \textbf{Loss of Steric Stabilization:} Casein micelles are natively stabilized by the hydrophilic, negatively charged C-terminal of $\kappa$-casein, often called the "hairy layer". Acidification titrates these negative charges, causing the "hairs" to shrink or collapse as the pH approaches 5.0, which removes the steric barrier preventing micellar contact. 
    \item \textbf{Changes in Micelle Mass and Size:} Initially, the average micellar mass and radius decrease between pH 6.7 and 5.5 due to the dissolution of CCP and the dissociation of individual caseins (primarily $\beta$-casein) into the serum. Below pH 5.5, these dissolved caseins lose their charge, become more hydrophobic, and reassemble/aggregate onto the remaining micellar framework. 
    \item \textbf{Reduction in Zeta Potential:} The negative surface charge (zeta potential) of the micelles decreases numerically as pH drops. When it reaches a critical level (< $\approx$ 4 mV), the repulsive forces are overtaken by attractive van der Waals forces and hydrophobic interactions, leading to aggregation. 
\end{enumerate}

\subsection*{9.2}
\vspace{1em}
What are the main factors affecting structure formation in fermented milk products, and how can these be used to control the structural properties of the final product?

\subsubsection*{Answer}
Structure formation in fermented milk is a highly controllable process governed by several technological variables:
\textbf{Main Factors and Control Mechanisms} 
\begin{enumerate} 
    \item \textbf{Heat Treatment of Milk:} Heating milk to high temperatures (e.g., 85-95 \textdegree C for 5-30 min) is the most critical factor. It causes whey proteins ($\beta$-lactoglobulin) to denature and bind covalently via disulphide bonds to $\kappa$-casein on the micelle surface. This creates a finer, more branched protein network with smaller pores, resulting in higher gel firmness (G') and reduced syneresis (wheying-off). 
    \item \textbf{Protein Content (Dry Matter):} Increasing the total solids through evaporation or the addition of skim milk powder/UF-retentate increases the number of stress-bearing strands in the network. Higher protein concentration leads to a stiffer gel and higher viscosity. \textbf{Homogenisation:} Subjecting milk to high pressure (15-25 MPa) disrupts fat globules, which are then coated with casein to form a "recombined fat globule membrane". These globules act as "active fillers" that integrate into the protein network, significantly improving viscosity and smoothness. 
    \item \textbf{Incubation Temperature and Inoculation Rate:} Lowering the incubation temperature (e.g., to 40 \textdegree C) and using a higher inoculation rate provides a more rigid structure and reduces the risk of whey separation by allowing for slower, more uniform rearrangements of the network. 
    \item \textbf{Exopolysaccharides (EPS):} Using "ropy" starter cultures that produce EPS acts as a natural texturiser, increasing the viscosity and providing a characteristic stringy texture. 
\end{enumerate}


\subsection*{9.3}
How can you measure the structure formation?

\subsubsection*{Answer}

\textbf{Measurement of Structure Formation} 
\begin{enumerate} 
    \item \textbf{Low-Amplitude Strain/Stress Oscillation Rheometry (LASOR):} This fundamental method measures the storage modulus (G') (elasticity/firmness), loss modulus (G'') (viscosity), and phase angle ($\delta$). The gel point is identified when $\delta$ reaches 45\textdegree. 
    \item \textbf{Microscopy:} Confocal Laser Scanning Microscopy (CLSM) or Scanning Electron Microscopy (SEM) is used to visualize the 3D protein network, pore size, and the distribution of fat globules. 
    \item \textbf{Physical Assays:} Viscosity is measured using rotational viscometers, while water-binding capacity is assessed by quantifying syneresis (whey expulsion) after centrifugation or drainage. 
    \item \textbf{Sensory Analysis:} Trained panels evaluate attributes like smoothness, thickness, and mouthfeel, which correlate with instrumental rheology. 
\end{enumerate}


\newpage
\section{Production of Fermented Milk Products}
A number of basic processing steps are included in most production processes for fermented milk products.

\subsection*{10.1}
\vspace{1em}
Present an outline of a standard production process for yoghurt manufacture (set or stirred), including a short description of the aim and outcome of the selected parameters for each processing step.

\subsubsection*{Answer}
A standard production process for yoghurt (both set and stirred) involves several critical stages designed to ensure a stable gel structure and specific sensory profile: 
\begin{enumerate} 
    \item \textbf{Preliminary Treatment and Standardization:} Milk is standardized to the desired fat content (typically <0.5-3.23\%). To improve viscosity and gel firmness, the total solids (TS) are increased by 1-3\% via the addition of skim milk powder (SMP), evaporation, or ultrafiltration. 
    \item \textbf{Homogenisation:} Performed at 60-70 \textdegree C and 15-20 MPa. The aim is to reduce fat globule size and create a recombined fat globule membrane coated with casein; these globules act as "active fillers" that integrate into the protein network, increasing consistency and preventing syneresis. 
    \item \textbf{High Heat Treatment (HHT):} Milk is heated to 80-85 \textdegree C for 30 minutes or 90-95 \textdegree C for 5 minutes. This is critical for denaturing whey proteins (specifically $\beta$-lactoglobulin), which then bind covalently to $\kappa$-casein on the micelle surface. This creates a finer, more stable 3D network that holds water more effectively. 
    \item \textbf{Inoculation:} The milk is cooled to 37-45 \textdegree C and inoculated with a starter culture, usually \textit{Streptococcus thermophilus} and \textit{Lactobacillus delbrueckii} subsp. \textit{bulgaricus}. 
    \item \textbf{Fermentation:} 
        \begin{itemize} 
            \item \textbf{Set-type:} The inoculated milk is filled into retail containers and incubated until the pH reaches $\approx$ 4.6, resulting in a continuous gel formed within the package. 
            \item \textbf{Stirred-type:} The milk is incubated in a large fermentation tank until pH $\approx$ 4.6. The resulting gel is then stirred (broken) and pre-cooled to 20-25\textdegree C before packaging. 
        \end{itemize} 
    \item \textbf{Final Cooling and Storage:} The product is blast-cooled to <5\textdegree C to stop metabolic activity and stabilize the texture. 
\end{enumerate}

\subsection*{10.2}
\vspace{1em}
How can concentration for production of fermented milk products like Greek yoghurt, ymer and skyr be obtained? Both traditional methods and modern technologies should be addressed.

\subsubsection*{Answer}
Concentration methods differentiate these products by increasing dry matter (DM) and protein content through the removal of whey: 

\begin{table}
    \centering
    \caption{Concentration methods for fermented milk products.}
    \label{tab:exam_question_10.2_table}
    \rowcolors{2}{white}{gray!7}
    \begin{tabular}{ p{4cm} | p{5cm} | p{5cm} }
        \textbf{Product} & \textbf{Traditional Method} & \textbf{Modern Method} \\ 
        \hline
        Skyr 
        & \makecell[l]{Acidified skim milk (often with a \\ trace of rennet) is placed in cloth \\ bags and hung in piles to allow the \\  whey to drain by gravity for 24  \\ hours.}  
        & \makecell[l]{Concentration is achieved using a \\ \textbf{quarg-separator} (centrifugal) or \\  \textbf{ultrafiltration} of the fermented \\ milk to reach 16-20\% DM.} \\

        Ymer 
        & \makecell[l]{Skim milk is fermented with a \\ mesophilic DL-culture. The curd is \\ heated to $\approx$50\textdegree C, causing it to float \\  to the top due to $\text{CO}_2$ production \\ by the starter. The underlying whey \\ is drained, and cream is added to \\ reach $\approx$ 3.5\% fat. }        
        & \makecell[l]{Skim milk is concentrated via \textbf{ul-} \\ \textbf{trafiltration} prior to fermentation, \\ ensuring all whey proteins are re- \\ tained, which increases yield and \\ provides a smoother texture.}    \\

        Greek Yoghurt / Labneh 
        & \makecell[l]{Straining the finished yoghurt in \\ muslin bags for several hours until \\ the desired solids content (approx. \\ 25\% DM) is reached.}   
        & \makecell[l]{Using a \textbf{nozzle separator} to con- \\ tinuously remove whey from the \\ stirred curd or concentrating the \\ milk base via \textbf{ultrafiltration} before \\ inoculation.}    \\
    \end{tabular}
\end{table}


\subsection*{10.3}
\vspace{1em}
Describe which molecules (apart from lactic acid) that are largely responsible for the flavour of plain yoghurt and describe how they are formed.

\subsubsection*{Answer}
The characteristic aroma and flavour of plain yoghurt result from specific volatile compounds produced during the symbiotic growth of the starter organisms: 
\begin{enumerate} 
    \item \textbf{Acetaldehyde:} This is the most essential "yoghurt flavour" molecule. It is primarily formed from the amino acid \textbf{threonine} via the enzyme \textbf{threonine aldolase}, which cleaves it into glycine and acetaldehyde. It can also be a product of pyruvate metabolism. \textit{S. thermophilus} generally produces more acetaldehyde than \textit{L. bulgaricus}. 
    \item \textbf{Diacetyl (2,3-butanedione):} Provides a "buttery" aroma. It is formed through \textbf{citrate metabolism}, where $\alpha$-acetolactate undergoes spontaneous decarboxylation into diacetyl. 
    \item \textbf{Acetoin:} Also a product of citrate metabolism, contributing buttery and "sour milk" notes. 
    \item \textbf{Acetic Acid and Ethanol:} Produced via carbohydrate (pyruvate) metabolism, these add sharp and slightly alcoholic notes that round out the flavour profile. 
    \item \textbf{Peptides and Free Amino Acids:} While \textit{L. bulgaricus} is primarily responsible for proteolysis to stimulate the growth of \textit{S. thermophilus}, the resulting small peptides and amino acids (like Valine or Leucine) contribute to background taste notes. 
\end{enumerate}



