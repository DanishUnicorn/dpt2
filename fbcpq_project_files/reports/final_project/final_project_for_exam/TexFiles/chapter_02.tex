\setcounter{chapter}{1}
\setcounter{section}{0}
%\chapter{Introduction}
\setlength{\headheight}{12.71342pt}
\addtolength{\topmargin}{-0.71342pt}

\section{Introduction}
Feta cheese is a traditional Greek cheese that is protected under the Protected Designation of Origin (PDO) scheme, ensuring that authentic feta is produced in specific regions of Greece and from specific milk types. Feta must be produced from sheep's milk or from a blend of sheep's and goat's milk, with goat's milk comprising no more than 30\% of the total milk content (Katsouri et al., 2020)
Most of the PDO-certified feta is produced from pasteurized milk and relies on defined starter cultures to ensure consistent fermentation. Commonly used bacterial strains include, but are not limited to, Streptococcus thermophilus and Lactobacillus delbrueckii subsp. bulgaricus, which dominate commercial feta production however minor variation in microflora between producers can be observed (Katsouri et al., 2020).
The feta cheese analysed in this course was sourced from the brand Änglamark and manufactured in Greece. According to the product information the feta cheese has a energy value of 278 kcal per 100 g, with 23 g fat, 16 g protein, and 5.5 g salt per 100 g.


\subsection{Manufacturing}
The manufacturing process of Greek feta cheese differs from other semi-hard cheeses such as Gouda, Danbo and is illustrated in Figure X . Feta is a brined, non-pressed cheese, and its production is characterized by high moisture content, acidic fermentation, and maturation in brine.
Milk intended for feta production must comply with PDO regulations and consist of 100\% sheep's milk or a mixture of sheep's and goat's milk, with the total amount of goat's milk not exceeding 30\% (Katsouri et al., 2020).
To ensure microbiological safety and consistent quality, the milk is pasteurized most often using high temperature short time (HTST) pasteurization of 72°C for 15 seconds. Pasteurization decreases the microbial load (pathogenic and spoilage microorganisms) in the milk while preserving the functional properties of proteins necessary for cheesemaking (Katsouri et al., 2020).
Following pasteurization, the milk is cooled to inoculation temperature, and thermophilic starter cultures are added. These typically include Streptococcus thermophilus and Lactobacillus delbrueckii subsp. bulgaricus, which contribute with a rapid lactose fermentation and lactic acid production. The resulting decrease in pH is essential for curd formation and contributes to the characteristic acidic flavour of feta cheese. After reaching the targeted pH rennet is introduced, leading to enzymatic cleavage of $\kappa$-casein and the aggregation of casein micelles into a gel.
Once coagulated, the curd is cut into large cubes to facilitate whey expulsion while retaining a high moisture level. Unlike Gouda and Danbo, feta curds are not washed or scalded, as more lactose is needed during early fermentation and ripening. The cut cheese curd is transferred into perforated moulds, where whey drainage occurs under gravity without mechanically added pressure. 
After drainage, the curd is cut into blocks and dry salted for several days. This step increases whey removal, regulates microbial activity and initiates rind formation. After dry salting the blocks are submerged in a salted brine solution and ripening occurs over a minimum of two months as required to be PDO certified. During ripening acidification, proteolysis and salt diffusion continues and contribute to the final texture and flavour of the feta cheese.


\subsection{Materials and Methods}
An Änglamark feta cheese was used for all experiments described in this report. The cheese is organic and certified with Protected Designation of Origin (PDO). It is produced in Hellas, Greece, and made from 70\% sheep's milk and 30\% goat's milk. 
The nutritional information provided on the packaging is shown below and will be references when dicussing the laboratory results.

\begin{table}[h]
    \centering
    \caption{Nutritional information of Änglamark feta cheese per 100 g.}
    \label{tab:feta_nutrition}
    \rowcolors{2}{white}{gray!7}
    \begin{tabular}{ c | c  }
        \textbf{Nærringsværdi} & Per 100 g\\
        \hline
        Indeholder          & 100 g \\
        Energi              & 278 kcal \\
        Fedt                & 23 g \\
        Heraf mættet fedt   & 16 g \\
        Kulhydrater         & 0.7 g\\
        Heraf sukkerarter   & 0.7 g\\
        Protein             & 17 g\\
        Salt                & 2.2 g\\
    \end{tabular}
\end{table}

\subsubsection{Preparing the Cheese for the Lab}
We were given two 150g feta cheeses. Before preparing the cheese sample, a sensory analysis was done outside the laboratory.
The feta was crumbled and divided into two bags with approximately equal amounts in each, with one part being used on the first day in the laboratory and the other part being put in the freezer for day two in the laboratory for later analysis. 
Analysis of the cheese was decided into four areas:

\begin{itemize}
    \item Compositional analysis 
    \item Proteolysis 
    \item Lipolysis
    \item Glycolysis
\end{itemize}

\subsubsection{Compositional Analysis}
The compositional analysis included determination of the total solid content (see Table XX), fat content using the Gerber method, and total nitrogen and protein levels using the DUMAS method. Additionally, the pH, salt, and ash content were measured.
For the total solids content about 5 g of feta was added to a beaker containing pumice and mixed with a spatula. Afterwards it was dried for 16-18 hours at 104°C. The sample was then cooled to room temperature in the desiccator and weighed before it was put back in the oven for 2 hours more. Then it was cooled in the desiccator again and weighed once again to secure that the constant weight was obtained.
Gerber method for fat content
For measurement of fat content with the Gerber method we added 3 g of feta to a rubber stopper, which was fitted with a glass tube and connected to the bottom of a butyrometer. 10 ml of sulfuric acid was added, after which it was heated in a water bath at a temperature of 65-70°C for 10 minutes. Every 10 minutes, a stopper was inserted into the butyrometer, and it was subsequently inverted ten times to ensure thorough mixing, after which the butyrometer was returned to the water bath. This process continued until the cheese was completely dissolved, at ten-minute intervals. 1 ml of amyl alcohol was added to the butyrometer and mixed with the dissolved cheese to release the fat. Sulfuric acid was added until the liquid surface reached the 30 percent mark, after which the butyrometer was inverted 10 times. The butyrometer was left in the water bath for a period of 5 minutes, after which it was transferred to the Gerber centrifuge. It ran at 65 °C, at 1000-1200 RPM for 5 minutes. After the centrifuge, the butyrometer was put back in the water bath for another 5 minutes. The percentage value could then be read from the butyrometer.
For the determination of total nitrogen and protein in feta using DUMAS, 0.5 grams of feta was added to a crucible and covered with parafilm. The crucible was placed in the autosampler on top of the Rapid Max N Exceed and subsequently analyzed using Dumas. The results were subsequently provided to us.
For the NaCl content, we used a Hach TitraLab AT1000 Series to calculate the percentage. 2 grams of feta were weighed and transferred into a 100 mL beaker and mixed with 10 mL of a 0.5M sodium-tri-citrate solution and 40 mL heated deionized water at 50°C. Then placed on an external stirring plate at 50°C for a duration of one hour until the cheese was fully suspended. The Hach TitraLab AT1000 Series calculates the results directly in percentage of NaCl.
To measure the pH of feta, 2-4 g of feta were weighed into a plastic shot glass and mixed with water in a 1:1 ratio to form a paste, after which the pH was measured with a pH meter.
For the ash content 2 g feta was weighed and placed in a pre-weighed crucible, which was then placed in a desiccator and transferred to an oven by the laboratory assistants, where it underwent a dehydration process overnight at a temperature of 104°C. It was then transferred to the muffle furnace at a temperature of about 525°C for 20 hours. The crucible was subsequently weighed to determine its ash content.

For proteolysis, the amount of pH 4.6-soluble nitrogen was determined using DUMAS, along with formol-titratable nitrogen in the pH 4.6 fraction and ammonium nitrogen content.
For pH 4.6-SN - soluble nitrogen, 12.5 grams of feta are placed in a 400 ml beaker and 50 ml of neutralized 0.5 M trisodium citrate solution is added, cover with foil and place in a water bath at 55°C for 30 minutes. Transfer the sample to a 250 ml volumetric flask, cool to room temperature and add deionized water to the 250 ml mark. Return the sample to the beaker and add 28 ml of 1.0 M HCl. The pH is measured and should be between 4.3-4.6.
For the formalin titratable nitrogen, two samples were prepared. 1 cheese sample and 1 blank. For the cheese sample, 20 ml cheese filtrate from the previous experiment is pipetted into a 50 ml beaker. Place on a stirring plate with a magnet in the beaker. Lower the pH electrode into the liquid and add 1 ml 1.0 M NaOH. The titration is carried out with a digital burette. 0.1 M NaOH is added to the desired pH of 8.3. 10 ml neutralized formalin is added, and a new titration is carried out again with 0.1 M NaOH, pH 8.3 again. The amount of 0.1 M NaOH, at each titration is noted as well as the final pH value. For the blank, add 50 ml 0.5 M trisodium citrate and dilute in a 250 ml volumetric flask. The solution is transferred to a 400 ml beaker, and 28 ml 1.0 M HCl is added. 20 ml of the sample is transferred to a 50 ml beaker. Same procedures with magnet, stirring plate and pH electrode. Finally, 1 ml 1.0 M NaOH is added. Same process as before and the amount of 1.0 M NaOH and final pH value are noted. The pH was then adjusted with 1.0 M HCl and the solution filtered into a new 400 ml beaker. The filtrate is then used to determine total nitrogen using the Dumas method and the Formol titratable nitrogen method.
For ammonium N, 2.5 g of feta is weighed into a Kjeldahl digestion tube and mixed with 2 g of barium carbonate and 2-3 drops of anti-foam. A receiving flask is prepared with 50 ml of 1.0\% (w/w) boric acid solution and 4 drops of Kjeldahl indicator. The receiving flask is placed in the Kjeldahl instrument and the digestion tubes containing the cheese samples are fixed in a position so that they are in contact with the rubber stopper. The samples are distilled and the beaker containing the distilled sample is removed from the Kjeldahl instrument. A titration with 0.1 M HCl is performed. A colour change from green to grey is observed and the exact amount of HCl is documented. However, this analysis is not done by us, but by the technicians and the results are uploaded to us for further analysis.
Lipolysis was assessed by titrating the acidity of the fat in the cheese.
Glycolysis was measured by determining the concentrations of D- and L-lactic acid.
For the acidity of fat in cheese, 30 g of feta were mixed with 1.2 g of sodium polyphosphate, 3.0 ml of 1 M NaOH and 50 ml of deionized water. The mixture was stirred as much as possible until a smooth, thin paste is obtained. The cheese paste formed is transferred to a volumetric flask using 50 ml of 40-50°C warm deionized water. The cheese paste is then heated in a water bath at 100 °C for 15-20 minutes until the fat is clearly separated. 15 ml of 1.0 M HCl is added to the volumetric flask and swirled gently and placed on the table for 15 minutes. 50 ml of BDI reagent is added to the volumetric flask, swirled gently and placed in the water bath for a further 10 minutes until the fat phase is clearly visible. When the fat phase was clearly visible, the volumetric flask was filled with deionized water for marking and left on the table for 5 minutes until the fat appeared as a distinct clear, clean fat phase at the top of the volumetric flask. An amount of 0.5-0.8 g of the clear fat phase was transferred to a 25 ml beaker, after which 15 ml of fat-dissolving mixture was added. Subsequently, titrated with KOH until a clear turn to pink was observed. The pH and amount of KOH were noted.
For the enzymatic determination of L- and D-lactic acid, 1 g of feta was transferred to 100 ml of cold measuring flask with 80 ml of deionized water and placed in a 60°C water bath for 15 minutes. The sample was then cooled to room temperature and adjusted to the 100 ml mark in the measuring flask. The sample was transferred to a new beaker via a Whatman 1 filter. Four cuvettes were used for each analysis: a blank, a standard and two samples. Subsequently, two different procedures had to be followed in the exercise manual for D-Lactic Acid and L-Lactic Acid, respectively.
Details regarding the equipment, materials, and methods used are provided in the Manual for Chemical Analysis of Cheese (Hougaard and Danielsen, 2025) 


\section{Results and Calculations}
\subsection{Compositional Analysis Results}
\begin{table}[h]
    \centering
    \caption{Total solids data for feta cheese samples.}
    \label{tab:total_solids}
    \resizebox{\textwidth}{!}{
    \rowcolors{2}{white}{gray!7}
    \begin{tabular}{ c | p{4cm} | p{4cm} | p{4cm} }
        \textbf{Replicate} & \textbf{Weigh of beaker + pumice + spatula: m$_0$ [g]} & \textbf{Weigh of grated sample: m$_1$ [g]} & \textbf{Weigh after drying (beaker + pumice + spatula + sample): m$_2$ [g]} \\
        \hline
        Sample 1 & 65.6194 & 4.9960 & 67.8421 \\
        Sample 2 & 65.1652 & 4.9935 & 68.4598
    \end{tabular} }
\end{table}

Total solids were calculated using the formula in equation \ref{eq:total_solids_01}.

\begin{equation}
    \text{Total Solids [\%]} = \frac{(m_2 - m_0)}{m_1} \times 100
    \label{eq:total_solids_01}
\end{equation}

Where:
\begin{itemize}
    \item $m_0$ = weight of beaker + pumice + spatula [g]
    \item $m_1$ = weight of grated sample before drying [g]
    \item $m_2$ = weight of the beaker, pumice, spatula, and sample after drying [g]
\end{itemize}

After determination of total solids (TS), the moisture content was calculated as:
\begin{equation}
    \text{Moisture [\%]} = 100-\text{Total Solids}
    \label{eq:total_solids_02}
\end{equation}

The results from the total solids content calculations are summarized below in equations \ref{eq:TS_01} and \ref{eq:TS_02}:

\begin{equation}
    \text{Sample 1: $TS_1$} = \frac{67.8421-65.6194}{4.9960}\times 100 = 44.49 \%
    \label{eq:TS_01}
\end{equation}

And,

\begin{equation}
    \text{Sample 1: $TS_2$} = \frac{68.4598-65.1652}{4.9935}\times 100 = 65.98 \%
    \label{eq:TS_02}
\end{equation}

The difference between the two samples has been calculated as seen in equation \ref{eq:TS_diff}:

\begin{equation}
\Delta \mathrm{TS}\,[\%]
= \lvert TS_1 - TS_2 \rvert
= \lvert 44.49 - 65.98 \rvert
= 21.49\%
\label{eq:TS_diff}
\end{equation}

The moisture content for each sample was calculated using equation \ref{eq:total_moisture_01} and \ref{eq:total_moisture_02}:

\begin{equation}
    \text{Sample 1: Moisture$_1$} = 100 - 44.49 = 55.51 \%
    \label{eq:total_moisture_01}
\end{equation}

And, 

\begin{equation}
    \text{Sample 2: Moisture$_2$} = 100 - 65.98 = 34.02 \%
    \label{eq:total_moisture_02}
\end{equation}


\subsection{Fat Content - Gerber Method}
The fat content of the cheese was determined using the Gerber method. The fat percentage was determined by reading the butyrometer after completion of the analysis. The results are summarised in Table \ref{tab:gerber_fat_content}.

\begin{table}[h]
    \centering
    \caption{Fat content results from Gerber method for feta cheese samples.}
    \label{tab:gerber_fat_content}
    \rowcolors{2}{white}{gray!7}
    \begin{tabular}{ c | c | c }
        \textbf{} & \textbf{[g]} & \textbf{\% fat} \\
        \hline
        Sample 1 & 3.05 & 24 \\
        Sample 2 & 3.04 & 22.5 \\
    \end{tabular}
\end{table}

The difference between the duplicates was calculated as seen in equation \ref{eq:fat_diff}:

\begin{equation}
    \Delta \mathrm{Fat}[\%]
    = \lvert \text{Fat}_1 - \text{Fat}_2 \rvert
    = \lvert 24 - 22.5 \rvert
    = 1.5\%
\label{eq:fat_diff}
\end{equation}

Lastly the average fat content was calculated using equation \ref{eq:fat_avg}:

\begin{equation}
    \text{Average Fat [\%]} = \frac{\text{Fat}_1 + \text{Fat}_2}{2} = \frac{24 + 22.5}{2} = 23.25 \%
    \label{eq:fat_avg}
\end{equation}

\subsection{Total Nitrogen and Protein Content - DUMAS}
The total protein content of the feta cheese was calculated from the nitrogen content using a conversion factor of 6.38, which is commonly used for milk proteins. The results from the DUMAS analysis are summarized in Table \ref{tab:dumas_nitrogen_protein}.

\begin{table}[h]
    \centering
    \caption{Total nitrogen and protein content results from DUMAS analysis for feta cheese samples.}
    \label{tab:dumas_nitrogen_protein}
    \rowcolors{2}{white}{gray!7}
    \begin{tabular}{ c | c | c | c }
        \textbf{} & \textbf{[mg]} & \textbf{[N\%]} & \textbf{g protein/100g} \\
        \hline
        Sample 1 & 512.1 & 2.34 & 14.92 \\
        Sample 2 & 516.3 & 2.43 & 15.50 \\
    \end{tabular}
\end{table}

The protein content was calculated using the formula in equation \ref{eq:protein_calculation}:

\begin{equation}
    \text{Protein [\%]} = \text{Nitrogen [\%]} \times 6.38
    \label{eq:protein_calculation}
\end{equation}

The calculations for the protein \% for each sample are shown in equations \ref{eq:protein_01} and \ref{eq:protein_02}:

\begin{equation}
    \text{Sample 1: Protein}_1 = 2.34 \times 6.38 = 14.92 \%
    \label{eq:protein_01}
\end{equation}
And,
\begin{equation}
    \text{Sample 2: Protein}_2 = 2.43 \times 6.38 = 15.50 \%
    \label{eq:protein_02}
\end{equation}

The average protein content was calculated using equation \ref{eq:protein_avg}:
\begin{equation}
    \text{Average Protein [\%]} = \frac{\text{Protein}_1 + \text{Protein}_2}{2} = \frac{14.92 + 15.50}{2} = 15.21 \%
    \label{eq:protein_avg}
\end{equation}


\subsection{NaCl Content}
The salt (NaCl) content of the feta cheese was determined using the method described in the laboratory manual. The salt concentration was given by the analytical equipment. The results are summarised in Table \ref{tab:salt_content}.

\begin{table}[h]
    \centering
    \caption{Salt content results for feta cheese samples.}
    \label{tab:salt_content}
    \rowcolors{2}{white}{gray!7}
    \begin{tabular}{ c | c | c }
        \textbf{} & \textbf{{g}} & \textbf{\% salt} \\
        \hline
        Sample 1 & 2.030 & 3.00 \\
        Sample 2 & 2.002 & 2.95 \\
    \end{tabular}
\end{table}

The difference between the duplicates was calculated as seen in equation \ref{eq:salt_diff}:
\begin{equation}
    \Delta \mathrm{Salt}[\%]
    = \lvert \text{Salt}_1 - \text{Salt}_2 \rvert
    = \lvert 3.00 - 2.95 \rvert
    = 0.05\%
\label{eq:salt_diff}
\end{equation}

The salt content of the feta cheese samples was measured as 3.00\% and 2.95\%. The difference between duplicate measurements (0.05\%) is below the maximum allowed difference of 0.06\% specified in the laboratory manual and is therefore considered accurate and reliable.


\subsection{pH in Cheese}
The pH was measured for the two samples and the results are shown in table \ref{tab:pH_results}.
\begin{table}[h]
    \centering
    \caption{pH results for feta cheese samples.}
    \label{tab:pH_results}
    \rowcolors{2}{white}{gray!7}
    \begin{tabular}{ c | c }
        \textbf{} & \textbf{pH} \\
        \hline
        Sample 1 & 4.28 \\
        Sample 2 & 4.27 \\
    \end{tabular}
\end{table}

The measured pH values of the two feta cheese samples were 4.28 and 4.27. A difference of 0,01 in pH value indicates reliable results.


\subsection{Ash Content}
Ash contents were calculated using the formula in equation \ref{eq:ash_content_01}.

\begin{equation}
    \text{Ash Content [\%]} = \frac{(m_2 - m_0)}{m_1} \times 100
    \label{eq:ash_content_01}
\end{equation}

Where:
\begin{itemize}
    \item $m_0$ = the mass of the crucible without ash [g]
    \item $m_1$ = the mass of the cheese sample before ashing [g]
    \item $m_2$ = the mass of the crucible and ash after ashing [g]
\end{itemize}

The measured masses used in calculations are summarised in Table \ref{tab:ash_content_masses}.

\begin{table}[h]
    \centering
    \caption{Masses used for ash content calculations of feta cheese samples.}
    \label{tab:ash_content_masses}
    \rowcolors{2}{white}{gray!7}
    \begin{tabular}{ c | c | c | c }
        \textbf{} & \textbf{$m_0$ [g]} & \textbf{$m_1$ [g]} & \textbf{$m_2$ [g]} \\
        \hline
        Sample 1 & 12.8282 & 1.9745 & 12.8947 \\
        Sample 2 & 12.4736 & 2.2283 & 12.5468 \\
    \end{tabular}
\end{table}

The results from the ash content calculations are summarized below in equations \ref{eq:ash_01} and \ref{eq:ash_02}:

\begin{equation}
    \text{Sample 1: Ash Content}_1 = \frac{12.8947-12.8282}{1.9745}\times 100 = 3.36 \%
    \label{eq:ash_01}
\end{equation}
And,
\begin{equation}
    \text{Sample 2: Ash Content}_2 = \frac{12.5468-12.4736}{2.2283}\times 100 = 3.28 \%
    \label{eq:ash_02}
\end{equation}


\section{Proteolysis Assessment Results}
\subsection{pH 4.6-Soluble Nitrogen}
Soluble nitrogen content at pH 4.6 was determined according to the method described in the laboratory manual. Soluble nitrogen is expressed as grams of nitrogen per 100 g of product (\% w/w). The nitrogen content of the soluble fraction was measured using DUMAS, see Table \ref{tab:ph46_sn_results}.

\begin{table}[h]
    \centering
    \caption{pH 4.6-soluble nitrogen results for feta cheese samples.}
    \label{tab:ph46_sn_results}
    \rowcolors{2}{white}{gray!7}
    \begin{tabular}{ c | c | c }
        \textbf{} & \textbf{[mg]} & \textbf{[N\%]} \\
        \hline
        Sample 1 & 512.1 & 2.344 \\
        Sample 2 & 516.3 & 2.43 \\
    \end{tabular}
\end{table}

The pH 4.6-soluble nitrogen content was calculated using the formula in equation \ref{eq:ph46_sn_calculation}:

\begin{equation}
    \text{pH 4.6-SN [\%]} = \frac{m_2}{m_1 \times \frac{2 [\text{mL}]}{278 \text{mL}}} \times 100
    \label{eq:ph46_sn_calculation}
\end{equation}

Where:

\begin{itemize}
    \item $m_1$ = the mass of the cheese sample [g]
    \item $m_2$ = the mass of the 2 mL sample used for DUMAS analysis [g]
    \item $N\%$ = nitrogen content in the filtrate determined by DUMAS [N\%]
\end{itemize}

The calculations for the pH 4.6-soluble nitrogen \% for each sample are shown in equations \ref{eq:ph46_sn_01} and \ref{eq:ph46_sn_02}:

\begin{equation}
    \text{Sample 1: pH 4.6-SN}_1 = \frac{0.5121}{12.5 \times \frac{2 [\text{[mL]}]}{278 \text{[mL]}}} \times 100 = 13.35 \%
    \label{eq:ph46_sn_01}
\end{equation}
And,
\begin{equation}
    \text{Sample 2: pH 4.6-SN}_2 = \frac{0.5163}{12.5 \times \frac{2 [\text{[mL]}]}{278 \text{[mL]}}} \times 100 = 13.95 \%
    \label{eq:ph46_sn_02}
\end{equation}


\subsection{Formol-Titratable Nitrogen}
Table \ref{tab:formol_titratable_nitrogen_results} summarizes the results from the formol-titratable nitrogen analysis.
\begin{table}[h]
    \centering
    \caption{Formol-titratable nitrogen results for feta cheese samples.}
    \label{tab:formol_titratable_nitrogen_results}
    \rowcolors{2}{white}{gray!7}
    \begin{tabular}{ c | c | c }
        \textbf{} & \textbf{0.1 M NaOH [mL]} & \textbf{pH} \\
        \hline
        Sample 1 & 0.21 & 8.34 \\
        Sample 2 & 0.22 & 8.29 \\
        Blank 1  & 0.00 & 8.30 \\
        Blank 2  & 0.00 & 8.30 \\
    \end{tabular}
\end{table}

The formol-titratable nitrogen content was calculated using the formula in equation \ref{eq:formol_titratable_nitrogen_calculation}:

\begin{equation}
    \text{Ammonium} = \frac{1.40 \times C \times (V_1 - V_2)}{m \times \frac{20}{278}}
    \label{eq:formol_titratable_nitrogen_calculation}
\end{equation}

Where:
\begin{itemize}
    \item $C$ = concentration of NaOH [mol/L]
    \item $V_1$ = volume of NaOH used for sample titration [mL]
    \item $V_2$ = volume of NaOH used for blank titration [mL]
    \item $m$ = mass of cheese sample [g]
\end{itemize}


The calculations for the formol-titratable nitrogen [mg] for each sample are shown in equations \ref{eq:formol_titratable_nitrogen_01} and \ref{eq:formol_titratable_nitrogen_02}:

\begin{equation}
    \text{Sample 1:} Ammonium_1 = \frac{1.40 \times 0.10 \times (0.21 - 0.00)}{12.5} \times \frac{20}{278} = 0.032
    \label{eq:formol_titratable_nitrogen_01}
\end{equation}
And,
\begin{equation}
    \text{Sample 2:} Ammonium_2 = \frac{1.40 \times 0.10 \times (0.22 - 0.00)}{12.5} \times \frac{20}{278} = 0.034
    \label{eq:formol_titratable_nitrogen_02}
\end{equation}


\subsection{Ammonium - Nitrate}
Table \ref{tab:ammonium_nitrogen_results} summarizes the results from the ammonium nitrogen analysis.
\begin{table}[h]
    \centering
    \caption{Data of the analysed feta cheese and blanks from the exercise, made by Tania, Oliver and Bente.}
    \label{tab:ammonium_nitrogen_results}
    \rowcolors{2}{white}{gray!7}
    \begin{tabular}{ c | c | c }
        \textbf{} & \textbf{g cheese} & \textbf{mL 0.1 M HCl} \\
        \hline
        Sample 1 & 2.5740 & 4.13 \\
        Sample 2 & 2.5782 & 5.2 \\
        Blank 1  & -      & 0.11 \\
        Blank 2  & -      & 0.04 \\
    \end{tabular}
\end{table}

The equation from 2.2. is used to calculate ammonium-nitrogen content with the fraction removed, as seen in equation \ref{eq:ammonium_nitrogen_calculation}:
\begin{equation}
    \text{Ammonium}_1 = \frac{1.40 \times C \times (V_1 - V_2)}{m}
    \label{eq:ammonium_nitrogen_calculation}
\end{equation}

The calculations for the ammonium-nitrogen \% for each sample are shown in equations \ref{eq:ammonium_nitrogen_calculation_01} and \ref{eq:ammonium_nitrogen_calculation_02}:

\begin{equation}
    \text{Ammonium}_1 = \frac{1.40 \times 0.10 \times (4.13 - 0.11)}{2.5740} = 0.2186 N\%
    \label{eq:ammonium_nitrogen_calculation_01}
\end{equation}
And,
\begin{equation}
    \text{Ammonium}_2 = \frac{1.40 \times 0.10 \times (5.2 - 0.04)}{2.5782} = 0.2802 N\%
    \label{eq:ammonium_nitrogen_calculation_02}
\end{equation}

The ammonium percentage for sample 1 is 0.2186 N\% and for sample 2 is 0.2802 N\%.


\section{Assessment of Lipolysis}
\subsection{Acidity of Fat in Cheese}
Table \ref{tab:acidity_of_fat_results} summarizes the results from the acidity of fat in cheese analysis.

\begin{table}[h]
    \centering
    \caption{Acidity of fat in cheese results for feta cheese samples.}
    \label{tab:acidity_of_fat_results}
    \rowcolors{2}{white}{gray!7}
    \begin{tabular}{ c | c | c | c }
        \textbf{} & \textbf{Fat [g]} & \textbf{KOH [mL]} & \textbf{Degree of acidity [\%]} \\
        \hline
        Sample 1 & 0.55885 & 2.57 & 4.59 \\
        Sample 2 & 0.55380 & 2.18 & 3.94 \\
    \end{tabular}
\end{table}


The degree of acidity was calculated using the formula in equation \ref{eq:degree_of_acidity_calculation}:

\begin{equation}
    \text{Degree of Acidity [\%]} = \frac{C \times V}{m} \times 100
    \label{eq:degree_of_acidity_calculation}
\end{equation}

Where:
\begin{itemize}
    \item $C$ = concentration of KOH [mol/L]
    \item $V$ = volume of KOH used for titration [mL]
    \item $m$ = mass of fat in the sample [g]
\end{itemize}

The calculations for the degree of acidity \% for each sample are shown in equations \ref{eq:degree_of_acidity_01} and \ref{eq:degree_of_acidity_02}:

\begin{equation}
    \text{Sample 1: Degree of Acidity}_1 = \frac{0.1 \times 2.57}{0.55885} \times 100 = 4.59 \%
    \label{eq:degree_of_acidity_01}
\end{equation}
And,
\begin{equation}
    \text{Sample 2: Degree of Acidity}_2 = \frac{0.1 \times 2.18}{0.55380} \times 100 = 3.94 \%
    \label{eq:degree_of_acidity_02}
\end{equation}


\section{Glycolysis Analysis}
\subsection{Enzymatic Determination of L- and D-Lactic Acids}
The table \ref{tab:D_lactic_acid_results} summarizes the results from the D-lactic acid analysis.
\begin{table}[h]
    \centering
    \caption{D-lactic acid results for feta cheese samples.}
    \label{tab:D_lactic_acid_results}
    \rowcolors{2}{white}{gray!7}
    \begin{tabular}{ c | c | p{1.55cm} | c | c | c | c | c }
        & \multicolumn{1}{c|}{\textbf{Before D-LDH}}
        & \multicolumn{6}{c}{\textbf{After D-LDH}} \\
        \hline
        & A1 (3 min) & A2 (5 min) & A3 (6 min) & A4 (7 min) & A5 (8 min) & A6 (9 min) & A7 (10 min) \\
        \hline
        Blank & 0.450 & 0.450 & 0.450 & 0.462 & 0.471 & 0.474 & 0.468 \\
        Standard & 0.525 & 0.761 & 0.902 & 0.952 & 0.994 & 0.990 & 1.024 \\
        Sample 1 & 0.540 & 0.595 & 0.654 & 0.673 & 0.642 & 0.710 & 0.719 \\
        Sample 2 & 0.499 & 0.584 & 0.632 & 0.645 & 0.671 & 0.701 & 0.702 \\
        \hline
        \multicolumn{8}{c}{\textbf{Results}} \\
        \hline
        & \textbf{$\Delta A_{D-lactic \ acid}$} & \textbf{Conc. D-lactic acid [g/L]} & \multicolumn{5}{c}{\textbf{Content D-lactic acid [g/100 g]}}   \\
        \hline
        Blank & 0 & N/A & \multicolumn{5}{c}{N/A} \\
        Standard & 0.236 & 0.756 & \multicolumn{5}{c}{0.00783} \\
        Sample 1 & 0.055 & 0.176 & \multicolumn{5}{c}{0.00182} \\
        Sample 2 & 0.085 & 0.272 & \multicolumn{5}{c}{0.00282} \\
    \end{tabular}
\end{table}

The concentration and content of D-lactic acid were calculated for all samples. The results are presented in table \ref{tab:D_lactic_acid_results}, and an example of the calculation for Sample 2 is shown below.

The concentration of D-lactic acid were calculated using the formula in equation \ref{eq:D_lactic_acid_concentration}:

\begin{equation}
    C = \frac{V \times MW}{\epsilon \times d \times v} \times \Delta A_{D-lactic \ acid}
    \label{eq:D_lactic_acid_concentration}
\end{equation}

Where:
\begin{itemize}
    \item V = final reaction volume [mL]
    \item MW = molecular weight of D-lactic acid [g/mol]
    \item $\epsilon$ = extinction coefficient of NADH at 340 nm (6300 L·mol$^{-1}$·cm$^{-1}$)
    \item d = light path length through the cuvette [cm]
    \item v = sample volume [mL]
    \item $\Delta A_{D-lactic \ acid}$ = absorbance difference between measurements A1 and A2
\end{itemize}

