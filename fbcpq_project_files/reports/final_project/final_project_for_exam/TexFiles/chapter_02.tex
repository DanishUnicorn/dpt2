\setcounter{chapter}{1}
\setcounter{section}{0}
%\chapter{Introduction}
\setlength{\headheight}{12.71342pt}
\addtolength{\topmargin}{-0.71342pt}

\section{Introduction}
\label{sec:Introduction}
Feta cheese is a traditional Greek cheese that is protected under the Protected Designation of Origin (PDO) scheme, ensuring that authentic feta is produced in specific regions of Greece and from specific milk types. Feta must be produced from sheep's milk or from a blend of sheep's and goat's milk, with goat's milk comprising no more than 30\% of the total milk content \cite{s_a06_nutritional_characteristics_feta_2020}.

\vspace{1em}
Most of the PDO-certified feta is produced from pasteurized milk and relies on defined starter cultures to ensure consistent fermentation. Commonly used bacterial strains include, but are not limited to, \textit{Streptococcus thermophilus} and \textit{Lactobacillus delbrueckii} subsp. \textit{bulgaricus}, which dominate commercial feta production however minor variation in microflora between producers can be observed \cite{s_a06_nutritional_characteristics_feta_2020}.

\vspace{1em}
The feta cheese analysed in this course was sourced from the brand Änglamark and manufactured in Greece. According to the product information the feta cheese has a energy value of 278 kcal per 100 g, with 23 g fat, 16 g protein, and 5.5 g salt per 100 g.


\section{Manufacturing}
\label{sec:Manufacturing}

The manufacturing process of Greek feta cheese differs from other semi-hard cheeses such as Gouda, Danbo and is illustrated in figure \ref{fig:feta_manufacturing_process}. Feta is a brined, non-pressed cheese, and its production is characterized by high moisture content, acidic fermentation, and maturation in brine.

\begin{figure}[h]
    \centering
    \includegraphics[width=\textwidth]{figures/feta_flowchart.png}
    \caption{Schematic overview of the feta cheese manufacturing process.}
    \label{fig:feta_manufacturing_process}
\end{figure}

\vspace{1em}
Milk intended for feta production must comply with PDO regulations and consist of 100\% sheep's milk or a mixture of sheep's and goat's milk, with the total amount of goat's milk not exceeding 30\% \cite{s_a06_nutritional_characteristics_feta_2020}.

\vspace{1em}
To ensure microbiological safety and consistent quality, the milk is pasteurized most often using high temperature short time (HTST) pasteurization of 72\textdegree C for 15 seconds. Pasteurization decreases the microbial load (pathogenic and spoilage microorganisms) in the milk while preserving the functional properties of proteins necessary for cheesemaking \cite{s_a06_nutritional_characteristics_feta_2020}.

\vspace{1em}
Following pasteurization, the milk is cooled to inoculation temperature, and thermophilic starter cultures are added. These typically include \textit{Streptococcus thermophilus} and \textit{Lactobacillus delbrueckii} subsp. \textit{bulgaricus}, which contribute with a rapid lactose fermentation and lactic acid production. The resulting decrease in pH is essential for curd formation and contributes to the characteristic acidic flavour of feta cheese. After reaching the targeted pH rennet is introduced, leading to enzymatic cleavage of $\kappa$-casein and the aggregation of casein micelles into a gel.

\vspace{1em}
Once coagulated, the curd is cut into large cubes to facilitate whey expulsion while retaining a high moisture level. Unlike Gouda and Danbo, feta curds are not washed or scalded, as more lactose is needed during early fermentation and ripening. The cut cheese curd is transferred into perforated moulds, where whey drainage occurs under gravity without mechanically added pressure. 

\vspace{1em}
After drainage, the curd is cut into blocks and dry salted for several days. This step increases whey removal, regulates microbial activity and initiates rind formation. After dry salting the blocks are submerged in a salted brine solution and ripening occurs over a minimum of two months as required to be PDO certified. During ripening acidification, proteolysis and salt diffusion continues and contribute to the final texture and flavour of the feta cheese.


\section{Materials and Methods}
\label{sec:Material_and_Methods}
An Änglamark feta cheese was used for all experiments described in this report. The cheese is organic and certified with PDO. It is produced in Hellas, Greece, and made from 70\% sheep's milk and 30\% goat's milk. 

\vspace{1em}
The nutritional information provided on the packaging is shown below and will be references when discussing the laboratory results.

\begin{table}[h]
    \centering
    \caption{Nutritional information of Änglamark feta cheese per 100 g. Data was obtained from the product's webpage \cite{web_feta_composition_2026}.}
    \label{tab:feta_nutrition}
    \rowcolors{2}{white}{gray!7}
    \begin{tabular}{ c | c  }
        \textbf{Nutritional Value} & Per 100 g\\
        \hline
        Content                 & 100 g \\
        Energy                  & 278 kcal \\
        Fat                     & 23 g \\
        Of which saturated fat  & 16 g \\
        Carbohydrates           & 0.7 g\\
        Of which sugars         & 0.7 g\\
        Protein                 & 17 g\\
        Salt                    & 2.2 g\\
    \end{tabular}
\end{table}

\subsection{Preparing the Cheese for the Lab}
\label{subsec:preparing_the_cheese_for_the_lab}

We were given two 150g feta cheeses. Before preparing the cheese sample, a sensory analysis was done outside the laboratory.
The feta was crumbled and divided into two bags with approximately equal amounts in each, with one part being used on the first day in the laboratory and the other part being put in the freezer for day two in the laboratory for later analysis. 

\vspace{1em}
Analysis of the cheese was decided into four areas:

\begin{itemize}
    \item Compositional analysis 
    \item Proteolysis 
    \item Lipolysis
    \item Glycolysis
\end{itemize}

\subsection{Compositional Analysis}
\label{subsec:compositional_analysis}
The compositional analysis included determination of the total solid content (see table \ref{tab:total_solids}), fat content using the Gerber method, and total nitrogen and protein levels using the DUMAS method. Additionally, the pH, salt, and ash content were measured.

\vspace{1em}
For the total solids content about 5 g of feta was added to a beaker containing pumice and mixed with a spatula. Afterwards it was dried for 16-18 hours at 104\textdegree C. The sample was then cooled to room temperature in the desiccator and weighed before it was put back in the oven for 2 hours more. Then it was cooled in the desiccator again and weighed once again to secure that the constant weight was obtained.

\vspace{1em}
\subsection{Gerber method for fat content}
\label{subsec:gerber_method_for_fat_content}
For measurement of fat content with the Gerber method we added 3 g of feta to a rubber stopper, which was fitted with a glass tube and connected to the bottom of a butyrometer. 10 ml of sulfuric acid was added, after which it was heated in a water bath at a temperature of 65-70\textdegree C for 10 minutes. Every 10 minutes, a stopper was inserted into the butyrometer, and it was subsequently inverted ten times to ensure thorough mixing, after which the butyrometer was returned to the water bath. This process continued until the cheese was completely dissolved, at ten-minute intervals. 1 ml of amyl alcohol was added to the butyrometer and mixed with the dissolved cheese to release the fat. Sulfuric acid was added until the liquid surface reached the 30 percent mark, after which the butyrometer was inverted 10 times. The butyrometer was left in the water bath for a period of 5 minutes, after which it was transferred to the Gerber centrifuge. It ran at 65 \textdegree C, at 1000-1200 RPM for 5 minutes. After the centrifuge, the butyrometer was put back in the water bath for another 5 minutes. The percentage value could then be read from the butyrometer.

\vspace{1em}
For the determination of total nitrogen and protein in feta using DUMAS, 0.5 grams of feta was added to a crucible and covered with parafilm. The crucible was placed in the autosampler on top of the Rapid Max N Exceed and subsequently analysed using Dumas. The results were subsequently provided to us.

\vspace{1em}
For the NaCl content, we used a Hach TitraLab AT1000 Series to calculate the percentage. 2 grams of feta were weighed and transferred into a 100 mL beaker and mixed with 10 mL of a 0.5M sodium-tri-citrate solution and 40 mL heated deionized water at 50\textdegree C. Then placed on an external stirring plate at 50\textdegree C for a duration of one hour until the cheese was fully suspended. The Hach TitraLab AT1000 Series calculates the results directly in percentage of NaCl.

\vspace{1em}
To measure the pH of feta, 2-4 g of feta were weighed into a plastic shot glass and mixed with water in a 1:1 ratio to form a paste, after which the pH was measured with a pH meter.

\vspace{1em}
For the ash content 2 g feta was weighed and placed in a pre-weighed crucible, which was then placed in a desiccator and transferred to an oven by the laboratory assistants, where it underwent a dehydration process overnight at a temperature of 104\textdegree C. It was then transferred to the muffle furnace at a temperature of about 525\textdegree C for 20 hours. The crucible was subsequently weighed to determine its ash content.

\vspace{1em}
For proteolysis, the amount of pH 4.6-soluble nitrogen was determined using DUMAS, along with formol-titratable nitrogen in the pH 4.6 fraction and ammonium nitrogen content.

\vspace{1em}
For pH 4.6-SN - soluble nitrogen, 12.5 grams of feta are placed in a 400 ml beaker and 50 ml of neutralized 0.5 M trisodium citrate solution is added, cover with foil and place in a water bath at 55\textdegree C for 30 minutes. Transfer the sample to a 250 ml volumetric flask, cool to room temperature and add deionized water to the 250 ml mark. Return the sample to the beaker and add 28 ml of 1.0 M HCl. The pH is measured and should be between 4.3-4.6.

\vspace{1em}
For the formalin titratable nitrogen, two samples were prepared. 1 cheese sample and 1 blank. For the cheese sample, 20 ml cheese filtrate from the previous experiment is pipetted into a 50 ml beaker. Place on a stirring plate with a magnet in the beaker. Lower the pH electrode into the liquid and add 1 ml 1.0 M NaOH. The titration is carried out with a digital burette. 0.1 M NaOH is added to the desired pH of 8.3. 10 ml neutralized formalin is added, and a new titration is carried out again with 0.1 M NaOH, pH 8.3 again. The amount of 0.1 M NaOH, at each titration is noted as well as the final pH value. For the blank, add 50 ml 0.5 M trisodium citrate and dilute in a 250 ml volumetric flask. The solution is transferred to a 400 ml beaker, and 28 ml 1.0 M HCl is added. 20 ml of the sample is transferred to a 50 ml beaker. Same procedures with magnet, stirring plate and pH electrode. Finally, 1 ml 1.0 M NaOH is added. Same process as before and the amount of 1.0 M NaOH and final pH value are noted. The pH was then adjusted with 1.0 M HCl and the solution filtered into a new 400 ml beaker. The filtrate is then used to determine total nitrogen using the Dumas method and the Formol titratable nitrogen method.

\vspace{1em}
For ammonium N, 2.5 g of feta is weighed into a Kjeldahl digestion tube and mixed with 2 g of barium carbonate and 2-3 drops of anti-foam. A receiving flask is prepared with 50 ml of 1.0\% (w/w) boric acid solution and 4 drops of Kjeldahl indicator. The receiving flask is placed in the Kjeldahl instrument and the digestion tubes containing the cheese samples are fixed in a position so that they are in contact with the rubber stopper. The samples are distilled and the beaker containing the distilled sample is removed from the Kjeldahl instrument. A titration with 0.1 M HCl is performed. A colour change from green to grey is observed and the exact amount of HCl is documented. However, this analysis is not done by us, but by the technicians and the results are uploaded to us for further analysis.

\vspace{1em}
Lipolysis was assessed by titrating the acidity of the fat in the cheese.
Glycolysis was measured by determining the concentrations of D- and L-lactic acid.

\vspace{1em}
For the acidity of fat in cheese, 30 g of feta were mixed with 1.2 g of sodium polyphosphate, 3.0 ml of 1 M NaOH and 50 ml of deionized water. The mixture was stirred as much as possible until a smooth, thin paste is obtained. The cheese paste formed is transferred to a volumetric flask using 50 ml of 40-50\textdegree C warm deionized water. The cheese paste is then heated in a water bath at 100 \textdegree C for 15-20 minutes until the fat is clearly separated. 15 ml of 1.0 M HCl is added to the volumetric flask and swirled gently and placed on the table for 15 minutes. 50 ml of BDI reagent is added to the volumetric flask, swirled gently and placed in the water bath for 10 minutes further, until the fat phase is clearly visible. When the fat phase was clearly visible, the volumetric flask was filled with deionized water for marking and left on the table for 5 minutes until the fat appeared as a distinct clear, clean fat phase at the top of the volumetric flask. An amount of 0.5-0.8 g of the clear fat phase was transferred to a 25 ml beaker, after which 15 ml of fat-dissolving mixture was added. Subsequently, titrated with KOH until a clear turn to pink was observed. The pH and amount of KOH were noted.

\vspace{1em}
For the enzymatic determination of L- and D-lactic acid, 1 g of feta was transferred to 100 ml of cold measuring flask with 80 ml of deionized water and placed in a 60\textdegree C water bath for 15 minutes. The sample was then cooled to room temperature and adjusted to the 100 ml mark in the measuring flask. The sample was transferred to a new beaker via a Whatman 1 filter. Four cuvettes were used for each analysis: a blank, a standard and two samples. Subsequently, two different procedures had to be followed in the exercise manual for D-Lactic Acid and L-Lactic Acid, respectively.

\vspace{1em}
Details regarding the equipment, materials, and methods used are provided in the Manual for Chemical Analysis of Cheese (Hougaard and Danielsen, 2025) 


\section{Results and Calculations}
\label{sec:results_and_calculations}
\subsection{Compositional Analysis Results}
\label{subsec:compositional_analysis_results}

Table \ref{tab:total_solids} summarizes the weights recorded during the total solids determination for the two feta cheese samples.

\begin{table}[h]
    \centering
    \caption{Total solids data for feta cheese samples.}
    \label{tab:total_solids}
    \resizebox{\textwidth}{!}{
    \rowcolors{2}{white}{gray!7}
    \begin{tabular}{ c | p{4cm} | p{4cm} | p{4cm} }
        \textbf{Replicate} & \textbf{Weigh of beaker + pumice + spatula: m$_0$ [g]} & \textbf{Weigh of grated sample: m$_1$ [g]} & \textbf{Weigh after drying (beaker + pumice + spatula + sample): m$_2$ [g]} \\
        \hline
        Sample 1 & 65.6194 & 4.9960 & 67.8421 \\
        Sample 2 & 65.1652 & 4.9935 & 68.4598
    \end{tabular} }
\end{table}

\vspace{1em}
Total solids were calculated using the formula in equation \ref{eq:total_solids_01}.

\begin{equation}
    \text{Total Solids [\%]} = \frac{(m_2 - m_0)}{m_1} \times 100
    \label{eq:total_solids_01}
\end{equation}

Where:
\begin{itemize}
    \item $m_0$ = weight of beaker + pumice + spatula [g]
    \item $m_1$ = weight of grated sample before drying [g]
    \item $m_2$ = weight of the beaker, pumice, spatula, and sample after drying [g]
\end{itemize}

\vspace{1em}
After determination of total solids (TS), the moisture content was calculated as:
\begin{equation}
    \text{Moisture [\%]} = 100-\text{Total Solids}
    \label{eq:total_solids_02}
\end{equation}

\vspace{1em}
The results from the total solids content calculations are summarized below in equations \ref{eq:TS_01} and \ref{eq:TS_02}:

\begin{equation}
    \text{Sample 1: $TS_1$} = \frac{67.8421-65.6194}{4.9960}\times 100 = 44.49 \%
    \label{eq:TS_01}
\end{equation}

And,

\begin{equation}
    \text{Sample 1: $TS_2$} = \frac{68.4598-65.1652}{4.9935}\times 100 = 65.98 \%
    \label{eq:TS_02}
\end{equation}

\vspace{1em}
The difference between the two samples has been calculated as seen in equation \ref{eq:TS_diff}:

\begin{equation}
\Delta \mathrm{TS}\,[\%]
= \lvert TS_1 - TS_2 \rvert
= \lvert 44.49 - 65.98 \rvert
= 21.49\%
\label{eq:TS_diff}
\end{equation}

\vspace{1em}
The moisture content for each sample was calculated using equation \ref{eq:total_moisture_01} and \ref{eq:total_moisture_02}:

\begin{equation}
    \text{Sample 1: Moisture$_1$} = 100 - 44.49 = 55.51 \%
    \label{eq:total_moisture_01}
\end{equation}

And, 

\begin{equation}
    \text{Sample 2: Moisture$_2$} = 100 - 65.98 = 34.02 \%
    \label{eq:total_moisture_02}
\end{equation}


\subsection{Fat Content - Gerber Method}
\label{subsec:fat_content_gerber_method}
The fat content of the cheese was determined using the Gerber method. The fat percentage was determined by reading the butyrometer after completion of the analysis. The results are summarised in Table \ref{tab:gerber_fat_content}.

\begin{table}[h]
    \centering
    \caption{Fat content results from Gerber method for feta cheese samples.}
    \label{tab:gerber_fat_content}
    \rowcolors{2}{white}{gray!7}
    \begin{tabular}{ c | c | c }
        \textbf{} & \textbf{[g]} & \textbf{\% fat} \\
        \hline
        Sample 1 & 3.05 & 24 \\
        Sample 2 & 3.04 & 22.5 \\
    \end{tabular}
\end{table}

\vspace{1em}
The difference between the duplicates was calculated as seen in equation \ref{eq:fat_diff}:

\begin{equation}
    \Delta \mathrm{Fat}[\%]
    = \lvert \text{Fat}_1 - \text{Fat}_2 \rvert
    = \lvert 24 - 22.5 \rvert
    = 1.5\%
\label{eq:fat_diff}
\end{equation}

\vspace{1em}
Lastly the average fat content was calculated using equation \ref{eq:fat_avg}:

\begin{equation}
    \text{Average Fat [\%]} = \frac{\text{Fat}_1 + \text{Fat}_2}{2} = \frac{24 + 22.5}{2} = 23.25 \%
    \label{eq:fat_avg}
\end{equation}

\subsection{Total Nitrogen and Protein Content - DUMAS}
\label{subsec:total_nitrogen_and_protein_content_DUMAS}

The total protein content of the feta cheese was calculated from the nitrogen content using a conversion factor of 6.38, which is commonly used for milk proteins. The results from the DUMAS analysis are summarized in Table \ref{tab:dumas_nitrogen_protein}.

\begin{table}[h]
    \centering
    \caption{Total nitrogen and protein content results from DUMAS analysis for feta cheese samples.}
    \label{tab:dumas_nitrogen_protein}
    \rowcolors{2}{white}{gray!7}
    \begin{tabular}{ c | c | c | c }
        \textbf{} & \textbf{[mg]} & \textbf{[N\%]} & \textbf{g protein/100g} \\
        \hline
        Sample 1 & 512.1 & 2.34 & 14.92 \\
        Sample 2 & 516.3 & 2.43 & 15.50 \\
    \end{tabular}
\end{table}

\vspace{1em}
The protein content was calculated using the formula in equation \ref{eq:protein_calculation}:

\begin{equation}
    \text{Protein [\%]} = \text{Nitrogen [\%]} \times 6.38
    \label{eq:protein_calculation}
\end{equation}

\vspace{1em}
The calculations for the protein \% for each sample are shown in equations \ref{eq:protein_01} and \ref{eq:protein_02}:

\begin{equation}
    \text{Sample 1: Protein}_1 = 2.34 \times 6.38 = 14.92 \%
    \label{eq:protein_01}
\end{equation}
And,
\begin{equation}
    \text{Sample 2: Protein}_2 = 2.43 \times 6.38 = 15.50 \%
    \label{eq:protein_02}
\end{equation}

\vspace{1em}
The average protein content was calculated using equation \ref{eq:protein_avg}:
\begin{equation}
    \text{Average Protein [\%]} = \frac{\text{Protein}_1 + \text{Protein}_2}{2} = \frac{14.92 + 15.50}{2} = 15.21 \%
    \label{eq:protein_avg}
\end{equation}


\subsection{NaCl Content}
\label{subsec:NaCl_content}

The salt (NaCl) content of the feta cheese was determined using the method described in the laboratory manual. The salt concentration was given by the analytical equipment. The results are summarised in Table \ref{tab:salt_content}.

\begin{table}[h]
    \centering
    \caption{Salt content results for feta cheese samples.}
    \label{tab:salt_content}
    \rowcolors{2}{white}{gray!7}
    \begin{tabular}{ c | c | c }
        \textbf{} & \textbf{{g}} & \textbf{\% salt} \\
        \hline
        Sample 1 & 2.030 & 3.00 \\
        Sample 2 & 2.002 & 2.95 \\
    \end{tabular}
\end{table}

\vspace{1em}
The difference between the duplicates was calculated as seen in equation \ref{eq:salt_diff}:
\begin{equation}
    \Delta \mathrm{Salt}[\%]
    = \lvert \text{Salt}_1 - \text{Salt}_2 \rvert
    = \lvert 3.00 - 2.95 \rvert
    = 0.05\%
\label{eq:salt_diff}
\end{equation}

\vspace{1em}
The salt content of the feta cheese samples was measured as 3.00\% and 2.95\%. The difference between duplicate measurements (0.05\%) is below the maximum allowed difference of 0.06\% specified in the laboratory manual and is therefore considered accurate and reliable.

\vspace{1em}
To calculate the salt in moisture, the following formula was used in equation \ref{eq:salt_in_moisture}:

\begin{equation}
    \text{Salt in moisture [\%]} = \frac{\text{Salt [\%]}}{\text{Moisture [\%]}} \times 100
    \label{eq:salt_in_moisture}
\end{equation}

The calculations for the two samples are shown in equations \ref{eq:salt_in_moisture_01} and \ref{eq:salt_in_moisture_02}:

\begin{equation}
    \text{Sample 1: Salt in moisture}_1 = \frac{3.00}{55.51} \times 100 = 5.40 \%
    \label{eq:salt_in_moisture_01}
\end{equation}
And,
\begin{equation}
    \text{Sample 2: Salt in moisture}_2 = \frac{2.95}{34.02} \times 100 = 8.67 \%
    \label{eq:salt_in_moisture_02}
\end{equation}

Lastly, the average salt in moisture was calculated using equation \ref{eq:salt_in_moisture_avg}:
\begin{equation}
    \text{Average Salt in moisture [\%]} = \frac{\text{Salt in moisture}_1 + \text{Salt in moisture}_2}{2} = \frac{5.40 + 8.67}{2} = 7.04 \%
    \label{eq:salt_in_moisture_avg}
\end{equation}


\subsection{pH in Cheese}
\label{subsec:pH_in_cheese}
The pH was measured for the two samples and the results are shown in table \ref{tab:pH_results}.
\begin{table}[h]
    \centering
    \caption{pH results for feta cheese samples.}
    \label{tab:pH_results}
    \rowcolors{2}{white}{gray!7}
    \begin{tabular}{ c | c }
        \textbf{} & \textbf{pH} \\
        \hline
        Sample 1 & 4.28 \\
        Sample 2 & 4.27 \\
    \end{tabular}
\end{table}

\vspace{1em}
The measured pH values of the two feta cheese samples were 4.28 and 4.27. A difference of 0,01 in pH value indicates reliable results.


\subsection{Ash Content}
\label{subsec:ash_content}

Ash contents were calculated using the formula in equation \ref{eq:ash_content_01}.

\begin{equation}
    \text{Ash Content [\%]} = \frac{(m_2 - m_0)}{m_1} \times 100
    \label{eq:ash_content_01}
\end{equation}

Where:
\begin{itemize}
    \item $m_0$ = the mass of the crucible without ash [g]
    \item $m_1$ = the mass of the cheese sample before ashing [g]
    \item $m_2$ = the mass of the crucible and ash after ashing [g]
\end{itemize}

\vspace{1em}
The measured masses used in calculations are summarised in Table \ref{tab:ash_content_masses}.

\begin{table}[h]
    \centering
    \caption{Masses used for ash content calculations of feta cheese samples.}
    \label{tab:ash_content_masses}
    \rowcolors{2}{white}{gray!7}
    \begin{tabular}{ c | c | c | c }
        \textbf{} & \textbf{$m_0$ [g]} & \textbf{$m_1$ [g]} & \textbf{$m_2$ [g]} \\
        \hline
        Sample 1 & 12.8282 & 1.9745 & 12.8947 \\
        Sample 2 & 12.4736 & 2.2283 & 12.5468 \\
    \end{tabular}
\end{table}

\vspace{1em}
The results from the ash content calculations are summarized below in equations \ref{eq:ash_01} and \ref{eq:ash_02}:

\begin{equation}
    \text{Sample 1: Ash Content}_1 = \frac{12.8947-12.8282}{1.9745}\times 100 = 3.36 \%
    \label{eq:ash_01}
\end{equation}
And,
\begin{equation}
    \text{Sample 2: Ash Content}_2 = \frac{12.5468-12.4736}{2.2283}\times 100 = 3.28 \%
    \label{eq:ash_02}
\end{equation}


\section{Proteolysis Assessment Results}
\label{sec:proteolysis_assessment_results}
\subsection{pH 4.6-Soluble Nitrogen}
\label{subsec:ph46_soluble_nitrogen}

Soluble nitrogen content at pH 4.6 was determined according to the method described in the laboratory manual. Soluble nitrogen is expressed as grams of nitrogen per 100 g of product (\% w/w). The nitrogen content of the soluble fraction was measured using DUMAS, see Table \ref{tab:ph46_sn_results}.

\begin{table}[h]
    \centering
    \caption{pH 4.6-soluble nitrogen results for feta cheese samples.}
    \label{tab:ph46_sn_results}
    \rowcolors{2}{white}{gray!7}
    \begin{tabular}{ c | c | c }
        \textbf{} & \textbf{[mg]} & \textbf{[N\%]} \\
        \hline
        Sample 1 & 512.1 & 2.344 \\
        Sample 2 & 516.3 & 2.43 \\
    \end{tabular}
\end{table}

\vspace{1em}
The pH 4.6-soluble nitrogen content was calculated using the formula in equation \ref{eq:ph46_sn_calculation}:

\begin{equation}
    \text{pH 4.6-SN [\%]} = \frac{m_2}{m_1 \times \frac{2 [\text{mL}]}{278 \text{mL}}} \times 100
    \label{eq:ph46_sn_calculation}
\end{equation}

Where:

\begin{itemize}
    \item $m_1$ = the mass of the cheese sample [g]
    \item $m_2$ = the mass of the 2 mL sample used for DUMAS analysis [g]
    \item $N\%$ = nitrogen content in the filtrate determined by DUMAS [N\%]
\end{itemize}

\vspace{1em}
The calculations for the pH 4.6-soluble nitrogen \% for each sample are shown in equations \ref{eq:ph46_sn_01} and \ref{eq:ph46_sn_02}:

\begin{equation}
    \text{Sample 1: pH 4.6-SN}_1 = \frac{0.5121}{12.5 \times \frac{2 [\text{[mL]}]}{278 \text{[mL]}}} \times 100 = 13.35 \%
    \label{eq:ph46_sn_01}
\end{equation}
And,
\begin{equation}
    \text{Sample 2: pH 4.6-SN}_2 = \frac{0.5163}{12.5 \times \frac{2 [\text{[mL]}]}{278 \text{[mL]}}} \times 100 = 13.95 \%
    \label{eq:ph46_sn_02}
\end{equation}


\subsection{Formol-Titratable Nitrogen}
\label{subsec:formol_titratable_nitrogen}

Table \ref{tab:formol_titratable_nitrogen_results} summarizes the results from the formol-titratable nitrogen analysis.
\begin{table}[h]
    \centering
    \caption{Formol-titratable nitrogen results for feta cheese samples.}
    \label{tab:formol_titratable_nitrogen_results}
    \rowcolors{2}{white}{gray!7}
    \begin{tabular}{ c | c | c }
        \textbf{} & \textbf{0.1 M NaOH [mL]} & \textbf{pH} \\
        \hline
        Sample 1 & 0.21 & 8.34 \\
        Sample 2 & 0.22 & 8.29 \\
        Blank 1  & 0.00 & 8.30 \\
        Blank 2  & 0.00 & 8.30 \\
    \end{tabular}
\end{table}

\vspace{1em}
The formol-titratable nitrogen content was calculated using the formula in equation \ref{eq:formol_titratable_nitrogen_calculation}:

\begin{equation}
    \text{Ammonium} = \frac{1.40 \times C \times (V_1 - V_2)}{m \times \frac{20}{278}}
    \label{eq:formol_titratable_nitrogen_calculation}
\end{equation}

Where:
\begin{itemize}
    \item $C$ = concentration of NaOH [mol/L]
    \item $V_1$ = volume of NaOH used for sample titration [mL]
    \item $V_2$ = volume of NaOH used for blank titration [mL]
    \item $m$ = mass of cheese sample [g]
\end{itemize}


\vspace{1em}
The calculations for the formol-titratable nitrogen [mg] for each sample are shown in equations \ref{eq:formol_titratable_nitrogen_01} and \ref{eq:formol_titratable_nitrogen_02}:

\begin{equation}
    \text{Sample 1:} Ammonium_1 = \frac{1.40 \times 0.10 \times (0.21 - 0.00)}{12.5} \times \frac{20}{278} = 0.032
    \label{eq:formol_titratable_nitrogen_01}
\end{equation}
And,
\begin{equation}
    \text{Sample 2:} Ammonium_2 = \frac{1.40 \times 0.10 \times (0.22 - 0.00)}{12.5} \times \frac{20}{278} = 0.034
    \label{eq:formol_titratable_nitrogen_02}
\end{equation}


\subsection{Ammonium - Nitrate}
\label{subsec:ammonium_nitrogen}

Table \ref{tab:ammonium_nitrogen_results} summarizes the results from the ammonium nitrogen analysis.
\begin{table}[h]
    \centering
    \caption{Data of the analysed feta cheese and blanks from the exercise, made by Tania, Oliver and Bente.}
    \label{tab:ammonium_nitrogen_results}
    \rowcolors{2}{white}{gray!7}
    \begin{tabular}{ c | c | c }
        \textbf{} & \textbf{g cheese} & \textbf{mL 0.1 M HCl} \\
        \hline
        Sample 1 & 2.5740 & 4.13 \\
        Sample 2 & 2.5782 & 5.2 \\
        Blank 1  & -      & 0.11 \\
        Blank 2  & -      & 0.04 \\
    \end{tabular}
\end{table}

\vspace{1em}
The equation from \ref{eq:formol_titratable_nitrogen_calculation}. is used to calculate ammonium-nitrogen content with the fraction removed, as seen in equation \ref{eq:ammonium_nitrogen_calculation}:
\begin{equation}
    \text{Ammonium}_1 = \frac{1.40 \times C \times (V_1 - V_2)}{m}
    \label{eq:ammonium_nitrogen_calculation}
\end{equation}

\vspace{1em}
The calculations for the ammonium-nitrogen \% for each sample are shown in equations \ref{eq:ammonium_nitrogen_calculation_01} and \ref{eq:ammonium_nitrogen_calculation_02}:

\begin{equation}
    \text{Ammonium}_1 = \frac{1.40 \times 0.10 \times (4.13 - 0.11)}{2.5740} = 0.2186 N\%
    \label{eq:ammonium_nitrogen_calculation_01}
\end{equation}
And,
\begin{equation}
    \text{Ammonium}_2 = \frac{1.40 \times 0.10 \times (5.2 - 0.04)}{2.5782} = 0.2802 N\%
    \label{eq:ammonium_nitrogen_calculation_02}
\end{equation}

\vspace{1em}
The ammonium percentage for sample 1 is 0.2186 N\% and for sample 2 is 0.2802 N\%.


\section{Assessment of Lipolysis}
\label{sec:assessment_of_lipolysis}
\subsection{Acidity of Fat in Cheese}
\label{subsec:acidity_of_fat_in_cheese}

Table \ref{tab:acidity_of_fat_results} summarizes the results from the acidity of fat in cheese analysis.

\begin{table}[h]
    \centering
    \caption{Acidity of fat in cheese results for feta cheese samples.}
    \label{tab:acidity_of_fat_results}
    \rowcolors{2}{white}{gray!7}
    \begin{tabular}{ c | c | c | c }
        \textbf{} & \textbf{Fat [g]} & \textbf{KOH [mL]} & \textbf{Degree of acidity [\%]} \\
        \hline
        Sample 1 & 0.55885 & 2.57 & 4.59 \\
        Sample 2 & 0.55380 & 2.18 & 3.94 \\
    \end{tabular}
\end{table}

\vspace{1em}
The degree of acidity was calculated using the formula in equation \ref{eq:degree_of_acidity_calculation}:

\begin{equation}
    \text{Degree of Acidity [\%]} = \frac{C \times V}{m} \times 100
    \label{eq:degree_of_acidity_calculation}
\end{equation}

Where:
\begin{itemize}
    \item $C$ = concentration of KOH [mol/L]
    \item $V$ = volume of KOH used for titration [mL]
    \item $m$ = mass of fat in the sample [g]
\end{itemize}

\vspace{1em}
The calculations for the degree of acidity \% for each sample are shown in equations \ref{eq:degree_of_acidity_01} and \ref{eq:degree_of_acidity_02}:

\begin{equation}
    \text{Sample 1: Degree of Acidity}_1 = \frac{0.1 \times 2.57}{0.55885} \times 100 = 4.59 \%
    \label{eq:degree_of_acidity_01}
\end{equation}
And,
\begin{equation}
    \text{Sample 2: Degree of Acidity}_2 = \frac{0.1 \times 2.18}{0.55380} \times 100 = 3.94 \%
    \label{eq:degree_of_acidity_02}
\end{equation}


\section{Glycolysis Analysis}
\label{sec:glycolysis_analysis}
\subsection{Enzymatic Determination of L- and D-Lactic Acids}
\label{subsec:enzymatic_determination_of_l_and_d_lactic_acids}
\subsubsection{D-Lactic Acid Results}
\label{subsubsec:d-lactic_acid_results}

The table \ref{tab:D_lactic_acid_results} summarizes the results from the D-lactic acid analysis.
\begin{table}[h]
    \centering
    \caption{D-lactic acid results for feta cheese samples.}
    \label{tab:D_lactic_acid_results}
    \rowcolors{2}{white}{gray!7}
    \begin{tabular}{ c | c | p{1.55cm} | c | c | c | c | c }
        & \multicolumn{1}{c|}{\textbf{Before D-LDH}}
        & \multicolumn{6}{c}{\textbf{After D-LDH}} \\
        \hline
        & A1 (3 min) & A2 (5 min) & A3 (6 min) & A4 (7 min) & A5 (8 min) & A6 (9 min) & A7 (10 min) \\
        \hline
        Blank & 0.450 & 0.450 & 0.450 & 0.462 & 0.471 & 0.474 & 0.468 \\
        Standard & 0.525 & 0.761 & 0.902 & 0.952 & 0.994 & 0.990 & 1.024 \\
        Sample 1 & 0.540 & 0.595 & 0.654 & 0.673 & 0.642 & 0.710 & 0.719 \\
        Sample 2 & 0.499 & 0.584 & 0.632 & 0.645 & 0.671 & 0.701 & 0.702 \\
        \hline
        \multicolumn{8}{c}{\textbf{Results}} \\
        \hline
        & \textbf{$\Delta A_{D-lactic \ acid}$} & \textbf{Conc. D-lactic acid [g/L]} & \multicolumn{5}{c}{\textbf{Content D-lactic acid [g/100 g]}}   \\
        \hline
        Blank & 0 & N/A & \multicolumn{5}{c}{N/A} \\
        Standard & 0.236 & 0.756 & \multicolumn{5}{c}{0.00783} \\
        Sample 1 & 0.055 & 0.176 & \multicolumn{5}{c}{0.00182} \\
        Sample 2 & 0.085 & 0.272 & \multicolumn{5}{c}{0.00282} \\
    \end{tabular}
\end{table}

\vspace{1em}
The concentration and content of D-lactic acid were calculated for all samples. The results are presented in table \ref{tab:D_lactic_acid_results}, and an example of the calculation for Sample 2 is shown below.

\vspace{1em}
The concentration of D-lactic acid were calculated using the formula in equation \ref{eq:D_lactic_acid_concentration}:

\begin{equation}
    C = \frac{V \times MW}{\epsilon \times d \times v} \times \Delta A_{D-lactic \ acid}
    \label{eq:D_lactic_acid_concentration}
\end{equation}

Where:
\begin{itemize}
    \item V = final reaction volume [mL]
    \item MW = molecular weight of D-lactic acid [g/mol]
    \item $\epsilon$ = extinction coefficient of NADH at 340 nm (6300 L·mol$^{-1}$·cm$^{-1}$)
    \item d = light path length through the cuvette [cm]
    \item v = sample volume [mL]
    \item $\Delta A_{D-lactic \ acid}$ = absorbance difference between measurements A1 and A2
\end{itemize}

\vspace{1em}
Determination of $\Delta A_{D-lactic \ acid}$ is done by subtracting the absorbance of the blank from the absorbance of the sample:

\begin{equation}
    \Delta A_{blank} = A_{blank \ 2} - A_{blank \ 1} = 0.450 - 0.450 = 0.000 
    \label{eq:delta_A_blank}
\end{equation}
And, 
\begin{equation}
    \Delta A_{sample} = A_{sample \ 2.2} - A_{sample \ 2.1} = 0.584 - 0.499 = 0.085 
    \label{eq:delta_A_sample_2}
\end{equation}

\vspace{1em}
Now the total $\Delta A$ can be calculated:
\begin{equation}
    \Delta A_{D-lactic \ acid} = \Delta A_{sample} - \Delta A_{blank} = 0.085 - 0.000 = 0.085
    \label{eq:delta_A_D-lactic_acid}
\end{equation}

\vspace{1em}
Now that the $\Delta A_{D-lactic \ acid}$ is known, the concentration of D-lactic acid can be calculated using equation \ref{eq:D_lactic_acid_concentration_02}:

\begin{equation}
    C = \frac{V \times MW}{\epsilon \times d \times v} \times \Delta A_{D-lactic \ acid} = \frac{2.24 \times 90.10}{6300 \times 1.0 \times 0.1} \times 0.085 = 0.272 \ g/L
    \label{eq:D_lactic_acid_concentration_02}
\end{equation}

\vspace{1em}
As the sample was diluted, the content of D-lactic acid was calculated relative to the sample weight using equation \ref{eq:D_lactic_acid_content}:
\begin{equation}
    c_{D-lactic \ acid} = \frac{c_{D-lactic \ acid}}{m_{sample}} = \frac{0.0272}{\frac{0.9653}{100} \times 1000} = 0.00282 
    \label{eq:D_lactic_acid_content}
\end{equation}

\vspace{1em}
With an average D-lactic acid content of 0.00232 g/100 g for the two samples, as it can be seen in equation \ref{eq:average_D_lactic_acid_content}.

\begin{equation}
    Average_{D-lactic \ acid \ content} = \frac{Sample_1 + Sample_2}{2} = \frac{0.00282 + 0.00182}{2} = 0.00232
    \label{eq:average_D_lactic_acid_content}
\end{equation}

\subsubsection{L-Lactic Acid Results}
\label{subsubsec:l-lactic_acid_results}

The table \ref{tab:L_lactic_acid_results} summarizes the results from the L-lactic acid analysis.

\begin{table}[h]
    \centering
    \caption{L-lactic acid results for feta cheese samples.}
    \label{tab:L_lactic_acid_results}
    \rowcolors{2}{white}{gray!7}
    \begin{tabular}{ c | c | c | c | c }
        & \multicolumn{1}{c|}{\textbf{Before L-LDH}}
        & \multicolumn{3}{c}{\textbf{After L-LDH}} \\
        \hline
        & A1 (3 min) & A2 (10 min) & A3 (15 min) & A4 (20 min) \\
        \hline
        Blank       & 0.540 & 0.731 & 0.772 & 0.798 \\
        Standard    & 0.493 & 1.400 & 1.853 & 2.048 \\
        Sample 1    & 0.517 & 0.649 & 0.681 & 0.698 \\
        Sample 2    & 0.518 & 0.740 & 0.939 & 1.015 \\
        \hline
        \multicolumn{5}{c}{\textbf{Results}} \\
        \hline
        & \textbf{$\Delta A_{L-lactic \ acid}$} & \textbf{Conc. L-lactic acid [g/L]} & \multicolumn{2}{c}{\textbf{Content L-lactic acid [g/100 g]}} \\
        \hline
        Blank & 0.191 & N/A & \multicolumn{2}{c}{N/A} \\
        Standard & 0.907 & 0.229 & \multicolumn{2}{c}{0.237} \\
        Sample 1 & 0.132 & 0.0189 & \multicolumn{2}{c}{0.0196} \\
        Sample 2 & 0.222 & 0.00993 & \multicolumn{2}{c}{0.0103} \\
    \end{tabular}
\end{table}

\vspace{1em}
The concentration and content of L-lactic acid were calculated for all samples. The results are presented in Table \ref{tab:L_lactic_acid_results}, and the calculations for the concentration of L-lactic acid follows the same principle as for D-lactic acid, see equation \ref{eq:D_lactic_acid_concentration}.

\vspace{1em}
The absorbance difference for L-Lactic acid follows equation \ref{eq:delta_A_D-lactic_acid} but with values from the L-lactic acid analysis:

\begin{equation}
    \Delta A_{L-lactic \ acid \ Blank} = A_{2} - A_{1} = 0.731 - 0.540 = 0.191 
    \label{eq:delta_A_L-lactic_acid_blank}
\end{equation}
And,
\begin{equation}
    \Delta A_{L-lactic \ acid \ Sample} = A_{2} - A_{1} = 0.649 - 0.517 = 0.1325
    \label{eq:delta_A_L-lactic_acid_sample}
\end{equation}

\vspace{1em}
Now the total $\Delta A$ can be calculated:
\begin{equation}
    \Delta A_{L-lactic \ acid} = \Delta A_{blank} - \Delta A_{sample} = 0.191 - 0.1325 = 0.0585
\end{equation}

\vspace{1em}
The concentration of L-lactic acid was calculated following equation \ref{eq:L_lactic_acid_concentration_02}:

\begin{equation}
    C = \frac{V \times MW}{\epsilon \times d \times v} \times \Delta A_{L-lactic \ acid} = \frac{2.24 \times 90.10}{6300 \times 1.0 \times 0.1} \times 0.059 = 0.0272
    \label{eq:L_lactic_acid_concentration_02}
\end{equation}

\vspace{1em}
As the sample was diluted, the content of L-lactic acid was calculated relative to the sample weight, using equation \ref{eq:L_lactic_acid_content_02}:

\begin{equation}
    c_{L-lactic \ acid} = \frac{c_{L-lactic \ acid}}{m_{sample}} = \frac{0.0189}{\frac{0.9653}{100} \times 1000} = 0.0196 
    \label{eq:L_lactic_acid_content_02}
\end{equation}  

\vspace{1em}
The average was calculated using equation \ref{eq:average_L_lactic_acid_content}.
\begin{equation}
    Average_{L-lactic \ acid \ content} = \frac{Sample_1 + Sample_2}{2} = \frac{0.0196 + 0.0103}{2} = 0.01495
    \label{eq:average_L_lactic_acid_content}
\end{equation}

\section{Discussion}
\label{sec:discussion}
\subsection{Composition of the Cheese}
\label{subsec:composition_of_the_cheese}
\subsubsection{Total Solids/Moisture and Composition}
\label{subsubsec:total_solids/moisture_and_composition}

First, we will look at the total solids/moisture and the composition of feta cheese. We calculated the moisture to be around 55\% and 34\%, after we found the total solids from the two samples. Compared to literature \cite{n_a01_compositional_characteristics_feta_2021}, we would expect a moisture percentage between 45 and 60. Therefore we think that something went wrong in the second sample, since it so far from the first sample. And as mentioned we should have made a new sample as our duplicates exceeded the 0.20\% maximum described in the laboratory manual. But we accept the first sample as usable.

\vspace{1em}
As mentioned in the introduction we would expect a content of 23\% fat, 17\% protein, and 2.2\% salt. The Gerber method gave us an average fat content of 23.25\%. Which is considered close enough to be accepted. For the protein content we used the DUMAS, which gave us an average protein content of 15.21\%. A bit off, compared to the fat content, but still in an acceptable range. The salt content shows a bigger deviation between the declared amount and what we measured. We would expect a salt content around 2.2\%, but measured an average content of 2.975\%, which is a bit higher than expected.  

\vspace{1em}
Overall to get a more accurate content of the compositions, we could have made more samples and not just two. But we accept the results we got. A summary of the composition results can be seen in table \ref{tab:summary_composition_results}.

\begin{table}[h]
    \centering
    \caption{Summary of the composition results for feta cheese samples.}
    \label{tab:summary_composition_results}
    \rowcolors{2}{white}{gray!7}
    \begin{tabular}{ c | c | c | c }
        \textbf{Component} & \textbf{Declaration [\%]} & \textbf{Results [\%]} & \textbf{Deviation [\%]} \\
        \hline
        Fat         & 23  & 23.25 & 1.09 \\
        Protein     & 17  & 15.21 & 10.53 \\
        Salt        & 2.2 & 2.975 & 35.23 \\
    \end{tabular}
\end{table}

\vspace{1em}
So overall, we find the results we got for total solids and composition acceptable.


\subsubsection{pH}
\label{subsubsec:pH_discussion}
Feta cheese typically has a pH of 4.4-4.6 \cite{n_a01_compositional_characteristics_feta_2021}. Our measurements gave pH 4.28 and 4.27, which is a bit lower, but we consider it acceptable. Another possibility could be that the pH-meter was not calibrated correctly, as it was already sat up for the lab. We could except a smaller variability here, since we used to pack feta, so parameters such as starter culture, fermentation, brining and storage could have an impact on our measurements compared to the values from literature.

\subsubsection{Ash Content}
\label{subsubsec:ash_content_discussion}
Ash content is a great indicator for minerals such as Calcium, Magnesium and Sodium. From other results reported in literature, we found that the ash content should be somewhere between 4.10 to 4.60 \cite{n_a02_non-starter_lactic_acid_bacteria_2025}. For our first sample we had a content of around 3.37\% and for the second sample we had 3.28\%. So we are a bit lower than what we expected. 

\subsection{Proteolysis Assessment}
\label{subsec:proteolysis_assessment_discussion}
Proteolysis is a key biochemical process during cheese ripening. The degree of proteolysis in cheese is strongly dependent on milk composition, the type and activity of proteolytic enzymes present, and processing factors such as salt content and ripening conditions. The feta cheese analysed in this exercise was matured and stored in brine, resulting in a high NaCl concentration, which is known to inhibit enzymatic activity and microbial growth. The proteolysis potential in brined cheeses, such as feta, is generally less expressed than in hard, long-ripened cheeses, i.e. parmesan, where lower salt-in-moisture ratios and prolonged ripening favour protein degradation.

\vspace{1em}
The proteolytic development was evaluated using the ratio between pH 4.6-soluble nitrogen (pH 4.6-SN) and total nitrogen. At pH 4.6, intact caseins precipitate while smaller peptides remain soluble. Therefore, a low ratio reflects limited secondary proteolysis, whereas a high ratio indicates extensive degradation of caseins. The low pH 4.6-SN to total nitrogen ratio observed for the analysed feta cheese is consistent with the cheeses production and ripening conditions.

\vspace{1em}
The nitrogen fractionation further supported this interpretation. The low pH 4.6-SN relative to total nitrogen indicates that most nitrogen remains associated with intact or weakly degraded caseins. Formol-titratable nitrogen was low (0.032-0.034\%), indicating a limited accumulation of free amino acids. Ammonium nitrogen was also low, suggesting that deamination was low and extensive proteolysis. As a result, the SN-fraction is dominated by peptide nitrogen.


\subsection{Lipolysis}
\label{subsec:lipolysis_discussion}
In this report, the fat acidity of feta cheese was determined by measuring the content of free fatty acids (FFA) through titration. The concentration provides an indication of amount of lipolysis, as FFAs are released through the enzymatic degradation of triglycerides in the cheese fat.

\vspace{1em}
As presented in subsection \ref{subsec:acidity_of_fat_in_cheese}, the degree of acidity was determined to be 3.94\% and 4.59\% for the duplicate samples, indicating a relatively high level of lipolysis in the analysed feta cheese. In comparison, similar studies have reported fat acidity values ranging between 0.51\% and 2.07\% in white cheeses stored in brine,  suggesting that the values obtained in this experiment are high \cite{s_a01_influence_starters_2005}.

\vspace{1em}
The observed difference between the duplicate measurements may indicate experimental errors, particularly related to sample handling during the experiment. Determination of FFA involved several handling and transfer steps, and incomplete phase separation during fat extraction could have resulted in partial fat loss. Such losses would typically lead to an underestimation of fat acidity in one or both samples. However, given the relatively high acidity values presented in subsection \ref{subsec:acidity_of_fat_in_cheese}, this is not considered to be the primary cause of the high values.

\vspace{1em}
The high level of free fatty acids may instead be influenced by the storage conditions of the cheese. The feta cheese was stored in a freezer for seven days prior to the experiment. During storage, oxidative reactions may occur, contributing to an increase in fat acidity. Oxidation products, such as low-molecular-weight organic acids, can react with the titrant during analysis, thereby increasing the measured degree of acidity. Consequently, the measured acidity percent likely show a combination of lipolytic release of free fatty acids and oxidative degradation of lipids.

\vspace{1em}
Finally, the titration endpoint was determined by visual observation of a colour change, which introduces a degree of subjectivity. Variations in interpretation of the endpoint may have resulted in the addition of excess potassium hydroxide (KOH) before the titration was stopped leading to an overestimation of the fat acidity.

\subsection{Glycolysis}
\label{subsec:glycolysis_discussion}
Lactic acid is formed under fermentation of lactose by lactic acid bacteria (LAB) typically added as starter cultures for feta production. Lactose is metabolised through glycolysis, where glucose is converted into pyruvate. Pyruvate is then reduced to lactic acid to regenerate $NAD^+$ allowing glycolysis to continue for multiple cycles. 

\vspace{1em}
Lactic acid exists as two stereoisomers, L-lactic acid and D-lactic acid, and the proportion of each depends on the type of lactate dehydrogenase (LDH) present in the LAB. LAB containing L-lactate dehydrogenase (L-LDH) produce L-lactic acid, whereas LAB containing D-lactate dehydrogenase (D-LDH) produce D-lactic acid. Some LAB species have both enzymes and may therefore produce both isomers of lactic acid \cite{s_a02_regulation_mechanisms_l-lactic_2025}.  

\vspace{1em}
As described in the introduction of this report, \textit{S. thermophilus} and \textit{L. delbrueckii} are commonly used in the production of feta cheese. \textit{S. thermophilus} primarily produces the L-isomer of lactic acid, while \textit{L. delbrueckii} mainly produces the D-isomer \cite{s_web01_introduction_lactic_acid_bacteria_nd, s_a03_cloning_d-lactate_2017}.  

\vspace{1em}
The contents of both L- and D-lactic acid were analysed in the feta cheese given in this course, and the average concentrations were determined to be 0.0149 and 0.00232, respectively, as presented in subsection \ref{subsec:enzymatic_determination_of_l_and_d_lactic_acids}. The concentration of L-lactic acid were considerably higher than D-lactic acid, which is consistent with expectations for feta cheese and can be explained by differences in the growth behaviour of the used LAB. \textit{S. thermophilus} has a higher growth rate than \textit{L. delbrueckii} and is therefore able to convert a large proportion of the available lactose into lactic acid at an early stage of cheese production \cite{s_a04_milk_fermentation_monocultures_2022}. As a fast-growing LAB, \textit{S. thermophilus} dominates the microflora during the initial phase of fermentation, which partly explains the higher concentration of L-lactic acid compared to D-lactic acid.

\vspace{1em}
Additionally, as the pH decreases and salt is added during cheese manufacture, \textit{L. delbruckii} is inhibited more rapidly than \textit{S. thermophilus} \cite{s_a05_prevalence_lactobacillus_bulgaricus_2016}. As a result, the formation of D-lactic acid decreases faster than production of L-lactic acid, further contributing to the observed difference in lactic acid concentrations.

\vspace{1em}
Great differences were observed between the individual samples in the determination of L- and D-lactic acid, see subsection \ref{subsec:enzymatic_determination_of_l_and_d_lactic_acids}. These variations are likely attributable to experimental inaccuracies during sample preparation, including weighing and pipetting.

\subsection{Comparison Across Analysed Cheeses}
\label{subsec:comparison_across_analysed_cheeses}

During the laboratory exercises for this course, multiple types of cheese were analysed using similar methodologies. Comparing the results obtained for feta cheese with those from other cheeses, such as Grana Padano, Brie, and Havarti, provides insights into how cheese type and processing conditions influence biochemical properties.  

\vspace{1em}
Two tables were compiled using data from each group attending the course. For this report table \ref{tab:comparison_01} and \ref{tab:comparison_02} has been compiled to focus on comparison of our feta cheese, with group 4's Grana Padano, group 12's Brie, and group 15's Havarti.

\begin{table}[h]
    \centering
    \caption{Comparative compositional parameters of the analysed Feta cheese and selected reference cheeses from other groups: Grana Padano (Group 4), Brie (Group 12), and Havarti (Group 15).}
    \label{tab:comparison_01}
    \rowcolors{2}{white}{gray!7}
    \begin{tabular}{ c | c | c | c | c | c | c | c | c }
        \textbf{Gr.} & \textbf{Cheese} & \textbf{Total Solids} & \textbf{Ash} & \textbf{pH} & \textbf{NaCl} & \textbf{Salt-In-Moisture} & \textbf{D-Lactate} & \textbf{L-Lactate} \\
        \hline
        1  & Feta & 55.24 & 3.33 & 4.28 & 2.98 & 7.035 & 0.00232 & 0.01495 \\
        4  & Grana Padano & 65.60 & 3.91 & 5.56 & 1.065 & N/A & 0.93 & 0.73 \\
        12 & Brie & 48.45 & 2.47 & 7.17 & 1.21 & 2.37 & -0.0091 & 0.0022 \\
        15 & Havarti & 65.06 & 3.40  & 5.40 & 1.66  & 4.72 & N/A & 0.0547 \\
    \end{tabular}
\end{table}

A lower pH and a higher salt-in-moisture content were observed in the analysed feta cheese compared to the other cheeses, as shown in table \ref{tab:comparison_01}. This reflects the brined manufacture and limited ripening of feta, which contrasts with the longer ripening times and lower salt levels in Grana Padano, Brie, and Havarti. The total solids content of feta was intermediate between the high-moisture Brie and the low-moisture, long-ripened Grana Padano. The elevated salt level in feta is consistent with reduced lactate accumulation, as indicated by comparatively low D- and L-lactate concentrations. In contrast, Grana Padano cheese, which undergoes a prolonged fermentation and maturation, showed higher lactate levels and total solids.

\vspace{1em}
\begin{table}[h]
    \centering
    \caption{Comparative proteolysis- and lipolysis-related parameters of the analysed Feta cheese and selected reference cheeses from other groups: Grana Padano (Group 4), Brie (Group 12), and Havarti (Group 15).}
    \label{tab:comparison_02}
    \rowcolors{2}{white}{gray!7}
    \begin{tabular}{ c | c | c | c | c | c | c | c | c }
        \textbf{Gr.} & \textbf{Cheese} & \textbf{N$_{Total}$} & \textbf{Protein} & \textbf{pH 4.6-SN} & \textbf{N$_{Formol}$} & \textbf{N$_{Ammonium}$} & \textbf{Fat} & \textbf{Fat acidity} \\
        \hline
        1  & Feta & 2.38 & 15.21 & 0.1365 & 0.033 & 0.2494 & 23.25 & 4.26 \\
        4  & Grana Padano & 4.838 & 30.86 & 1.4089 & 0.898 & 0.603 & 28.4 & 3.628 \\
        12 & Brie & 4.89 & 31.2 & 0.09 & 0.825 & N/A & 34.6 & 4.58 \\
        15 & Havarti & 3.72 & 23.7 & 0.214 & 0.67 & N/A & 36.2 & 2.21 \\
    \end{tabular}
\end{table}

The results showed significantly lower total nitrogen and protein content in the analysed feta cheese compared to the other cheeses, as presented in table \ref{tab:comparison_02}. This reflects the limited proteolysis occurring during the short ripening time and high salt conditions of feta production. The pH 4.6-soluble nitrogen fraction was significantly lower in feta compared to Grana Padano, however, it was higher than in Brie, indicating some degree of protein degradation. Formol-titratable nitrogen and ammonium nitrogen were also low in feta, consistent with limited amino acid release and deamination.


\section{Conclusion}
\label{sec:conclusion}
Based on the analytical data obtain in this report (sections \ref{sec:results_and_calculations}, \ref{sec:proteolysis_assessment_results}, \ref{sec:assessment_of_lipolysis}, and \ref{sec:glycolysis_analysis}), minor discrepancies were observed between the declared nutritional values and the experimentally determined composition of the feta cheese. These differences are likely attributable to sampling variability, storage conditions prior to analysis, and minor experimental errors during analysis.

\vspace{1em}
The measured pH values of the cheese (4.28 and 4.27) were slightly lower than the typical pH range reported for traditional feta cheese, which is generally between 4.4 and 4.6. Furthermore, the ash content was found to be lower than the value stated on the product label. This deviation may be related to variations in milk composition or differences in brining and mineral uptake.

\vspace{1em}
The distribution of nitrogen in the analysed cheese confirmed that proteolysis was limited. This was reflected in the combined results showing a low pH 4.6-SN/TN ratio, low formol-titratable nitrogen, and a low ammonium nitrogen. These results indicates that the proteolytic potential was dominated by the primary events and less pronounced during the secondary, and that a minimal amino acid release was minimal. These results are consistent with feta cheeses, which are brine-ripened and stored, at high NaCl concentrations. The short ripening time and environment, constrains the enzymatic and microbial activity resulting in the low proteolytic potential. 

\vspace{1em}
Most of the nitrogen remained associated with intact or weakly degraded caseins, which contributes to the feta's characteristic firm and crumbly texture with a mild flavour profile. 

\vspace{1em}
Lipolysis analysis showed relatively high levels of FFA when compared to values reported in the literature. This is likely influenced by oxidative reactions occurring during storage, which may contribute to an overestimation of FFA originating from lipolysis. 

\vspace{1em}
Glycolysis analysis demonstrated a higher concentration of L-lactic acid compared to D-lactic acid, which is consistent with the starter cultures used in feta cheese production and aligns well with existing literature.

\vspace{1em}
Overall, the analyses performed as part of this course, illustrates how manufacturing practices, brining, and ripening conditions influence the properties of feta cheese. Despite minor and major variation, the experimentally determined values generally correspond well with the declared values on the label.

