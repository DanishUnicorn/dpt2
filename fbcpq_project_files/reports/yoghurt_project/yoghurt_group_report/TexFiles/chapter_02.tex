\setcounter{chapter}{1}
\setcounter{section}{0}
%\chapter{Introduction}
\setlength{\headheight}{12.71342pt}
\addtolength{\topmargin}{-0.71342pt}

\section{Introduction}
The purpose of this laboratory exercise is to analyse how changes in process parameters and yoghurt composition impact the final product. Specifically, the exercise focuses on variations in fat and protein content, as well as the influence of different backpressure levels during post-treatment. The yoghurts were produced on day 1, followed by a sensory evaluation. On day 7, a second sensory evaluation was conducted, along with measurements of viscosity and water-binding capacity. 

\section{Methods}

\subsection{1\textsuperscript{st} day of production: Production and sensory evaluation of yoghurt.}
Commercially homogenized milk with either 1.5\% or 3.5\% fat was used for production. Before producing the yoghurt, the milk was pasteurized at 85\textdegree C for 15 minutes by the supervisors. After pasteurization, whey protein isolate was added to four of the six batches as described below and in table 1.

\begin{itemize}
    \item Batches 1 and 2 consisted of milk with 1.5\% fat and 1\% added protein. 
    \item Batches 3 and 4 consisted of milk with 3.5\%  fat and 1\%  added protein.
    \item Batches 5 and 6 consisted of milk with 3.5\%  fat with no added protein. 
\end{itemize}

\begin{table}[h]
    \centering
    \caption{A table with an overview of the different yoghurt samples produced.}
    \label{tab:tab_01_samples}
    \rowcolors{2}{white}{gray!7}
    \begin{tabular}{ c | c | c | c }
        \textbf{Sample} & \textbf{Fat [\%]} & \textbf{Added Protein} & \textbf{Back Pressure} \\ 
        \hline
        1 & 1.5\% & 1\% & 1\\ 

        2 & 1.5\% & 1\% & 4\\

        3 & 3.5\% & 1\% & 1\\ 

        4 & 3.5\% & 1\% & 4\\

        5 & 3.5\% & -   & 1\\

        6 & 3.5\% & -   & 4\\
    \end{tabular}
\end{table}


After pasteurization and addition of protein the milk was warmed to 44\textdegree C in a water bath. The starter culture “YF-L706” was thawed, and 20 g was mixed with approximately 200 g of cold 1.5\% milk before addition in to the yoghurt base. 

Each batch received 20 g of the starter culture, after which the pH was measured. pH Measurements were taken every 20 minutes until the pH declined from roughly 6.5 to 4.6 (\ref{tab:tab_02_pH}). Once the desired acidity was reached, the yoghurt was cooled to 20\textdegree C.  

After reaching the desired pH all batches (1 thorugh 6) underwent additional mechanical treatment, where one sample in each pair (1 \& 2, 3 \& 4, 5 \& 6) was processed at 1 bar backpressure and the other at 4 bar. The finished yoghurts were transferred into containers and stored at 5\textdegree C in the refrigerator. 

Following a brief cooling stage, the yoghurt batches were assessed based on taste and visual appearance \ref{tab:tab_02_pH}. 

\subsection{2\textsuperscript{nd} day of production: Measurements and sensory evaluation of yoghurt.}
The yoghurt samples from day 1 were removed from the refrigerator, tasted, and compared with the initial sensory evaluation.  

After the sensory evalution, the yoghurt was poured into a cloth tea bag, allowing the whey to drain. The difference in weight before and after draining was used to determine the water-binding capacity [\%]. 
The pH was measured using a calibrated pH meter, and the viscosity was measured at different RPM settings using a Brookfield viscometer. 
Finally, the post-humus value was determined using the designated small-hole funnel where the time was recorded for yoghurt to pass the funnel. 

\section{Results}

\begin{table}
    \centering
    \caption{results from day 1, pH and sensory evaluation of yoghurt samples.}
    \label{tab:tab_02_pH}
    \resizebox{\textwidth}{!}{
    \rowcolors{2}{white}{gray!7}
    \begin{tabular}{ c | c | c | c | c | c | p{4cm} }
        \textbf{Sample} & \textbf{Fat [\%]} & \textbf{Added Protein} & \textbf{Back Pressure} & \textbf{Initial pH} & \textbf{Final pH} & \textbf{Sensory Evaluation} \\ 
        \hline
        1 & 1.5\% & 1\% & 1 & 6.51 & 4.58 & Grainy, but primarily in appearance. \\ 

        2 & 1.5\% & 1\% & 4 & 6.54 & 4.55 & Fairly fine but somewhat neutral. \\

        3 & 3.5\% & 1\% & 1 & 6.55 & 4.57 & Very grainy and unpleasant mouthfeel. \\ 

        4 & 3.5\% & 1\% & 4 & 6.55 & 4.57 & Very nice and appealing mouthfeel. \\

        5 & 3.5\% & -   & 1 & 6.55 & 4.55 & Boring and neutral, fair mouthfeel \\

        6 & 3.5\% & -   & 4 & 6.56 & 4.58 & Has a drinking-yoghurt–like consistency.\\
    \end{tabular} }
\end{table}

\begin{figure}[h]
    \centering
    \includegraphics[width=0.9\textwidth]{Figures/fig_01.png}
    \caption{Acidification curve.}
    \label{fig:acidification_curve}
\end{figure}

\begin{table}
    \centering
    \caption{Sensory evaulation for day 7}
    \label{tab:tab_03_sensory_day7}
    \resizebox{\textwidth}{!}{
    \rowcolors{2}{white}{gray!7}
    \begin{tabular}{ c | c | c | c | c | c | p{4cm} }
        \textbf{Sample} & \textbf{Fat [\%]} & \textbf{Added Protein} & \textbf{Back Pressure} & \textbf{Initial pH} & \textbf{Final pH} & \textbf{Sensory Evaluation} \\ 
        \hline
        1 & 1.5\% & 1\% & 1 & 6.51 & 4.58 & Grainy, but primarily in appearance. (No significant changes in texture, though a little in taste)\\ 

        2 & 1.5\% & 1\% & 4 & 6.54 & 4.55 & Fairly fine but somewhat neutral (No significant changes in texture, though a little in taste)\\

        3 & 3.5\% & 1\% & 1 & 6.55 & 4.57 & Very grainy and unpleasant mouthfeel (No significant changes in texture, though a little in taste).\\ 

        4 & 3.5\% & 1\% & 4 & 6.55 & 4.57 & Very nice and appealing mouthfeel (No significant changes in texture, though a little in taste).\\

        5 & 3.5\% & -   & 1 & 6.55 & 4.55 & Boring and neutral, fair mouthfeel (No significant changes in texture, though a little in taste).\\

        6 & 3.5\% & -   & 4 & 6.56 & 4.58 & Has a drinking-yoghurt–like consistency (Quite a lot of syneresis).\\
    \end{tabular} }
\end{table}



\begin{table}
    \centering
    \caption{Viscosity results (Day 7) for yoghurt samples.}
    \label{tab:tab_viscosity}
    \resizebox{\textwidth}{!}{
    \rowcolors{2}{white}{gray!7}
    \begin{tabular}{ c | c | c | c | c | c | c | c | c }
        \textbf{Sample} & \textbf{Fat [\%]} & \textbf{Added Protein} & \textbf{Back Pressure [bar]} & \textbf{10 rpm} & \textbf{20 rpm} & \textbf{30 rpm} & \textbf{40 rpm} & \textbf{50 rpm} \\
        \hline
        1 & 1.5\% & + & 1 & 9530 & 4560 & 3600 & 2780 & 2180 \\

        2 & 1.5\% & + & 4 & 10500 & 6000 & 3840 & 2670 & 2130 \\

        3 & 3.5\% & + & 1 & 4290 & 2185 & 1530 & 1120 & 930 \\

        4 & 3.5\% & + & 4 & 8630 & 2115 & 1377 & 992 & 832 \\

        5 & 3.5\% & - & 1 & 7340 & 3535 & 2357 & 1685 & 1370 \\

        6 & 3.5\% & - & 4 & 2620 & 1465 & 940 & 707 & 584 \\
    \end{tabular}}
\end{table}


\begin{figure}[h]
    \centering
    \includegraphics[width=0.9\textwidth]{Figures/fig_02.png}
    \caption{An illustration of the rpm's influence on both viscosity and shear rate.}
    \label{fig:rpm_influence}
\end{figure}

Post hummus and water-binding capacity results:

\begin{table} 
    \centering 
    \caption{pH, Posthumus time, and water binding results for yoghurt samples.} 
    \label{tab:tab_ph_posthumus_wb}
    \resizebox{\textwidth}{!}{ 
    \rowcolors{2}{white}{gray!7} 
    \begin{tabular}{ c | c | c | c | c | c | c } 
        \textbf{Sample} & \textbf{Fat [\%]} & \textbf{Added Protein} & \textbf{Back Pressure [bar]} & \textbf{pH} & \textbf{Posthumus [s]} & \textbf{Water Binding [\%]} \\ 
        \hline 
        1 & 1.5\% & + & 1 & 4.37 & 83 & 21.68 \\ 

        2 & 1.5\% & + & 4 & 4.31 & 28 & 18.61 \\ 

        3 & 3.5\% & + & 1 & 4.33 & 270 & 18.05 \\ 

        4 & 3.5\% & + & 4 & 4.33 & 73 & 14 \\ 

        5 & 3.5\% & - & 1 & 4.25 & 25 & 26 \\ 

        6 & 3.5\% & - & 4 & 4.24 & 7 & 43 \\ 
    \end{tabular} }
\end{table}

Measurements of water binding:  

\begin{equation}
\label{equation_01}
\text{Water Binding [\%]} = \frac   {\text{Weight of whey}}
                                    {\text{Weight of yoghurt}} \times 100
\end{equation}

Equation \ref{equation_01} shows how to calculate water binding capacity [\%] 

\begin{equation}
\label{equation_02}
\frac   {38.76}
        {182.16} \times 100 = 21.68\%
\end{equation}

Equation \ref{equation_02} shows our data integrated, where 38.76 g of whey was drained from 182.16 g of yoghurt, resulting in a water binding capacity of 21.68\%.

\section{How does the fat content affect yoghurt quality and why?}
Reducing fat content in yogurt is associated with poor texture. Compared to full-fat yogurt, reduced-fat yogurt exhibits lower tension, firmness, and adhesiveness. The removal of fat can also lead to quality defects such as a powdery taste, excessive firmness, and higher whey expulsion. Fat globules are crucial as structure promoters because they act as linking protein agents in the network. Protein-coated fat spheres, formed during processing, reinforce the gel by associating with casein micelles. Consequently, the reduced-fat yogurt protein network is less dense, more open, and contains more void spaces than full-fat yogurt. This results from the casein micelles forming chains instead of extensively fused aggregates, due to fewer fat globules being present to link the proteins. 

\section{How does added milk protein affect yoghurt quality and why?}
Added milk proteins, like Whey Protein Concentrate, can improve reduced-fat yogurt texture, sometimes mimicking full-fat quality. It increases strength and firmness because its denatured proteins become completely integrated into the protein network. protein additions (like skim milk powder) in large amounts can lead to quality defects such as excessive firmness, grainy texture, and higher whey expulsion. The protein matrix is primarily responsible for the firmness and springiness of yogurt 

\section{How does increased back pressure affect yoghurt quality and why?}
Increasing backpressure during yoghurt production can improve texture by strengthening the protein gel, resulting in a thicker, creamier product. Applying too much backpressure may lead to syneresis and can also harm live cultures, compromising flavour. 

\section{What would you recommend as the optimal processing parameters for the manufactured yoghurt?}
Depeding on which type of yoghurt that is desired, we would recommend the processing parameters that were used for the sample 4 for a classical Greek-stylish yoghurt, but if the desired product should be like a drinking yoghurt we would recommend the processing parameters for sample 6. The two types of samples were the same except that sample 4 had 1\% extra protein added. They had the same fat content and had the same backpressure.  