\setcounter{chapter}{1}
\setcounter{section}{0}
%\chapter{Introduction}
\setlength{\headheight}{12.71342pt}
\addtolength{\topmargin}{-0.71342pt}

\section{Introduction}
Feta cheese holds a Protected Designation of Origin (PDO) status, meaning that authentic feta may only be produced in specific regions of Greece. As part of the PDO, the used milk must come from sheep or from a mixture of sheep and goat milk, where the amount of goat milk cannot exceed 30\% \cite{n01_sandoval2004microstructure}. 

The production steps follow a general cheesemaking process, but with some difference unique to feta. After coagulation, the curd is transferred into perforated moulds and allowed to drain under its own weight, as feta is not a pressed cheese. Once the moisture level has decreased sufficiently, the cheese is dry salted for several days before placed into a brine solution, where it matures for at least two months \cite{n01_sandoval2004microstructure}. Traditionally, the ripening took place in wooden barrels or stainless-steel containers, although modern production has started to incorporate plastic containers. 

Most PDO feta is produced from pasteurized milk and inoculated with standardized starter cultures, typically Streptococcus thermophilus and Lactobacillus delbrueckii subsp. bulgaricus \cite{n01_sandoval2004microstructure}. Although the micro flora can vary among producers, beforementioned species dominate commercial production. 

The feta cheese analysed in this course was purchased from Angelmark and produced in Greece. It contained 23\% fat and had a maximum moisture content of 56\%. According to the nutrition label, the cheese provides 278 kcal per 100 g, consisting of 23 g fat, 16 g protein, and 5.5 g salt. 

Sensory-wise, the cheese had the characteristic salty, tangy and aromatic flavour associated with long brining and a subtle goat-like note was also perceived.  


\section{Phenotypic Clustering}

The heatmap of scaled relative abundance of aroma compounds shows a clear phenotypic separation among the seven cheeses. Feta forms its own distinct cluster, placed on the far left, indicating that its volatile profile differs substantially from all other cheeses (Danbo, Brie, Emmental, Grana, Cheddar, Havarti, and Gouda). This positioning results from a unique set of compounds present at relatively high abundance in Feta, many of which are either absent or present at much lower levels in the other cheeses. 

\begin{figure}
    \centering
    \includegraphics[height=0.9\textheight]{Figures/fig_01_cheese.JPG}
    \caption{Zooming into the Feta column on the heatmap (top ~25 rows), the most intense red signals include: Ethyl acetate, 3-methyl-1-butanol and som ethyl esters like: Hexanoic acid, Heptanoic acid, ethyl ester and Octanoic acid, ethyl ester.}
    \label{fig:cheese_heatmap}
\end{figure}

These compounds show high positive scores for Feta but appear blue/neutral for nearly all other cheeses which indicates that Feta has significantly higher levels of these volatiles than the rest. Compared with harder, longer-ripened cheeses like Grana, Cheddar and Emmentale, Feta shows low levels of ketones, lactones and low aldehyde complexity. 

Ethyl acetate is one of the most dominant markers, appearing at a much higher relative abundance in Feta. Its fruity, slightly fusel-like aroma reflects active ester formation during brining and fermentation, and its absence in comparable levels in the other cheeses makes it a clear contributor to Feta's unique position in the cluster analysis. 3-methyl-1-butanol is strongly elevated in Feta but remains low in all other cheeses. Its malty and fermented notes correspond well with the metabolic activity typical of fresh brined cheeses and help explain why Feta separates from the more matured cheese types. Hexanoic acid and its corresponding ethyl ester also appear prominently in Feta. These compounds contribute to characteristic tangy, goat-like, and slightly rancid notes and are associated with lipolysis of sheep and goat milk fat. Together, these volatiles create a distinctive aroma profile that makes Feta the sole member of its cluster, clearly separated from the other cheeses. 

When comparing Feta to the remaining cheeses, it becomes clear that its volatile profile differs fundamentally from those of both long-ripened and mould-ripened cheeses. This contrast reflects the absence of oxidative and mould-driven pathways in Feta's short maturation process. Feta also separates markedly from Brie, which shows high levels of mould-derived metabolites such as methyl ketones and sulfur compounds. Because Feta is not surface-ripened and undergoes no fungal fermentation, these characteristic volatiles are essentially absent 

Altogether, the clustering reflects clear biochemical differences linked to milk type, brining, and limited ripening. Feta's distinct combination of short-chain fatty acids, fusel alcohols, and esters positions it as an isolated cluster with no close similarity to the other cheeses in the dataset. 


\section{Most Abundant Compounds}
The volatile profile of Feta cheese is characterized by its distinct set of aroma compounds that differentiate from other cheese varieties. The most abundant volatiles include free fatty acids, alcohols, and esters, respectively \cite{a01_kondyli2012effect}. In this section, these  three groups will be described with emphasis on the most abundant individual compounds with respective tables for visualisation. Data has been derived from the study by \citeauthor{a01_kondyli2012effect} \cite{a01_kondyli2012effect}.

\subsection{Free Fatty Acids}
In the study by \citeauthor{a01_kondyli2012effect}, free fatty acids were found to be the most abundant compounds in the volatile fraction of Feta cheese. After 180 days of ripening, the five dominant FFAs were C10:0 (decanoic acid), C2:0 (acetic acid), C16:0 (palmitic acid), C14:0 (myristic acid) and C12:0 (lauric acid), as shown in Table \ref{tab:tab_ffa_180days}. These fatty acids contribute to the characteristic flavour profile of Feta, with C10:0 and C2:0 in particular.

\vspace{0.5em}
It was shown that packaging affected the quantitative distribution of these FFAs. Cheeses that was stored in tin vessels contained significantly (P < 0.05) lower levels of C10:0 compared with those ripened in wooden barrels. The levels of C16:0 and C14:0  was slightly lower, and C12:0 showed only a minor decrease in the tin vessels \textcite{a01_kondyli2012effect}.


\begin{table}[h]
    \centering
    \caption{Concentrations of the five most abundant free fatty acids (FFAs) in Feta cheese stored for 180 days in wooden barrels and tin vessels. Values are given as $\mu$g\,g$^{-1}$ cheese $\pm$ SD.}
    \label{tab:tab_ffa_180days}
    \rowcolors{2}{white}{gray!7}
    \begin{tabular}{ c | c | c }
        \textbf{FFA} & \textbf{Wood Barrels} & \textbf{Tin Vessels} \\
        \hline
        C10:0  & 299.03 $\pm$ 10.13 & 253.92 $\pm$ 20.17 \\
        C2:0  & 284.53 $\pm$ 20.15   & 184.87 $\pm$ 15.12   \\
        C16:0  & 226.27 $\pm$ 4.62  & 207.48 $\pm$ 12.72   \\
        C14:0  & 159.78 $\pm$ 4.97  & 129.20 $\pm$ 15.85   \\
        C12:0 & 138.74 $\pm$ 4.27 & 134.25 $\pm$ 3.57 \\
    \end{tabular}
\end{table}

To get a better understanding of the overall FFA profile, Table \ref{tab:tab_ffa_descriptors} has been compiled, illustrating the top five most abundant FFAs in the feta cheese with sensoric descriptors.

\begin{table}[h]
    \centering
    \caption{Sensoric descriptors of the five most abundant free fatty acids (FFAs) in Feta cheese. Data has been adapted from The Good Scents Company database at their respective minimum thresholds in the database \citeauthor{w01_GoodScentsDatabase} \cite{w01_GoodScentsDatabase}.}
    \label{tab:tab_ffa_descriptors}
    \resizebox{\textwidth}{!}{
    \rowcolors{2}{white}{gray!7}
    \begin{tabular}{ c | p{6cm} | p{6cm} }
        \textbf{FFA} & \textbf{Odour} & \textbf{Flavour} \\
        \hline
        C10:0  & Rancid, Sour, fatty, and citrus & Soapy, waxy, and fruity \\

        C2:0  & Pungent acidic and dairy-like   & Acidic, dairy with a pronounced fruity lift   \\

        C16:0  & Low heavy waxy, with a creamy, candle waxy nuance  & Waxy, creamy fatty, soapy with a crisco like fatty, lard and tallow like mouth feel and a dairy nuanc   \\        
        
        C14:0  & Faint, waxy and fatty with a hint of pineapple and citrus peel  & Waxy, fatty, soapy, creamy, cheesy, with a good mouth feel   \\

        C12:0 & mild fatty & Fatty, waxy \\ 
    \end{tabular} }
\end{table}


\subsection{Alcohols}
The second most abundant group of volatiles in feta cheese is represented by alcohols. At 180 days of ripening the five dominant alcohols were ethanol, butan-2-ol, 3-methylbutan-1-ol, phenylethanol and butan-1-ol \cite{a01_kondyli2012effect}, as shown in Table \ref{tab:tab_alcohols_180days}. Ethanol was present at the highest concentration, reflecting the fermentative activity during brining and storage. Butan-2-ol and 3-methylbutan-1-ol were also present at significantly high levels.

\vspace{0.5em}
Packaging had also an influence on the quantitative profile of the alcohols. Cheeses that were ripened in wooden barrels contained substantially higher levels of ethanol, butan-2-ol, phenylethanol and butan-1-ol compared with cheeses stored in tin vessels, indicating a more active fermentation and microbial pathways in the wooden environment. Conversely, 3-methylbutan-1-ol showed similar concentrations in both packaging types. Despite these differences, the dominant alcohols remained consistent across packaging type \textcite{a01_kondyli2012effect}.


\begin{table}[h]
    \centering
    \caption{Concentrations of the five most abundant alcohols in Feta cheese stored for 180 days in wooden barrels and tin vessels. Values are given as peak area in TIC $\times 10^{5}$ $\pm$ SD.}
    \label{tab:tab_alcohols_180days}
    \rowcolors{2}{white}{gray!7}
    \begin{tabular}{ c | c | c }
        \textbf{Alcohol} & \textbf{Wood Barrels} & \textbf{Tin Vessels} \\
        \hline
        Ethanol             & 54{,}634.4 $\pm$ 3512.30 & 35{,}458.0 $\pm$ 2160.70 \\
        Butan-2-ol          & 20{,}990.60 $\pm$ 1549.10   & 9{,}004.70 $\pm$ 512.30  \\
        3-Methylbutan-1-ol  & 17{,}671.2 $\pm$ 2085.0    & 18{,}641.4 $\pm$ 1999.8   \\
        Phenylethanol       & 8{,}560.7 $\pm$ 343.5    & 1{,}226.7 $\pm$ 229.7   \\
        Butan-1-ol          & 3{,}114.5 $\pm$ 48.5     & 381.1 $\pm$ 103.7   \\
    \end{tabular}
\end{table}

Table \ref{tab:tab_alcohols_descriptors} was compiled to illustrate the top five most abundant alcohols in the feta cheese. Sensoric descriptors have been added for better understanding of their individual contributions to the overall aroma profile of the cheese.

\begin{table}[h]
    \centering
    \caption{Sensoric descriptors of the five most abundant alcohols in Feta cheese. Data has been adapted from The Good Scents Company database at their respective minimum thresholds in the database \citeauthor{w01_GoodScentsDatabase} \cite{w01_GoodScentsDatabase}.}
    \label{tab:tab_alcohols_descriptors}
    \resizebox{\textwidth}{!}{
    \rowcolors{2}{white}{gray!7}
    \begin{tabular}{ c | p{6cm} | p{6cm} }
        \textbf{Alcohol} & \textbf{Odour} & \textbf{Flavour} \\
        \hline
        Ethanol  & Strong alcoholic ethereal, medical & n.d. \\

        Butan-2-ol  & Sweet apricot   & n.d.   \\

        3-Methylbutan-1-ol  & Fusel, alcoholic, pungent, etherial, cognac, fruity, banana and molasses  & Fusel, fermented, fruity, banana, etherial and cognac   \\        
        
        Phenylethanol  & Sweet, floral, fresh and bready with a rosey honey nuance  & Floral, sweet, rosey and bready   \\

        Butan-1-ol & Fusel oil sweet balsam whiskey & Banana fusel \\ 
    \end{tabular} }
\end{table}


\subsection{Esters}
Esters formed the third most abundant group of volatile compounds in Feta cheese at 180 days of ripening. The five dominant esters were ethyl hexanoate, ethyl octanoate, ethyl butanoate, ethyl decanoate and 2-phenylethyl acetate, as shown in Table \ref{tab:tab_esters_180days}. These compounds are primarily formed through esterification between free fatty acids and ethanol or other alcohols \cite{a01_kondyli2012effect}, and contribute with fruity, floral and sweet notes that balance the sharper acidic components of the cheese.

\vspace{0.5 em}
A clear effect of packaging was observed on the quantitative levels of esters. Cheeses ripened in wooden barrels contained significantly higher concentrations of all five esters compared with cheeses stored in tin vessels, indicating that the wooden environment favoured ester synthesis. The differences were especially pronounced for ethyl hexanoate, ethyl octanoate and ethyl butanoate \textcite{a01_kondyli2012effect}.

\begin{table}[h]
    \centering
    \caption{Concentrations of the five most abundant esters in Feta cheese stored for 180 days in wooden barrels and tin vessels. Values are given as peak area in TIC $\times 10^{5}$ $\pm$ SD.}
    \label{tab:tab_esters_180days}
    \rowcolors{2}{white}{gray!7}
    \begin{tabular}{ c | c | c }
        \textbf{Ester} & \textbf{Wood Barrels} & \textbf{Tin Vessels} \\
        \hline
        Ethyl hexanoate        & 17{,}310.2 $\pm$ 947.1   & 1{,}437.4 $\pm$ 157.8 \\
        Ethyl octanoate        & 16{,}930.3 $\pm$ 1001.3 & 3{,}198.7 $\pm$ 394.1 \\
        Ethyl butanoate        & 13{,}189.6 $\pm$ 844.7   & 1{,}561.8 $\pm$ 87.2 \\
        Ethyl decanoate        & 9{,}861.6 $\pm$ 449.8    & 733.8 $\pm$ 44.1 \\
        2-Phenylethyl acetate  & 7{,}862.3 $\pm$ 298.9    & 821.5 $\pm$ 104.7 \\
    \end{tabular}
\end{table}

A compilation of the top five most abundant esters in Feta cheese is shown in Table \ref{tab:tab_esters_descriptors}, along with their sensoric descriptors to illustrate their individual contributions to the overall aroma profile of the cheese.

\begin{table}[h]
    \centering
    \caption{Sensoric descriptors of the five most abundant esters in Feta cheese. Data has been adapted from The Good Scents Company database at their respective minimum thresholds in the database \citeauthor{w01_GoodScentsDatabase} \cite{w01_GoodScentsDatabase}.}
    \label{tab:tab_esters_descriptors}
    \resizebox{\textwidth}{!}{
    \rowcolors{2}{white}{gray!7}
    \begin{tabular}{ c | p{6cm} | p{6cm} }
        \textbf{Ester} & \textbf{Odour} & \textbf{Flavour} \\
        \hline
        Ethyl hexanoate    & Sweet, fruity, pineapple, waxy, fatty and estry with a green banana nuance & Sweet, pineapple, fruity, waxy and banana with a green, estry nuance \\

        Ethyl octanoate        & Waxy, sweet, musty, pineapple and fruity with a creamy, dairy nuance   & Sweet, waxy, fruity and pineapple with creamy, fatty, mushroom and cognac notes   \\

        Ethyl butanoate  & Sweet, fruity, tutti frutti, lifting and diffusive  & Fruity, sweet, tutti frutti, apple, fresh and lifting, ethereal   \\        
        
        Ethyl decanoate  & Sweet, waxy, fruity, apple  & Waxy, fruity, sweet apple   \\

        2-Phenylethyl acetate  & Sweet, honey, floral rosy, with a slight yeasty honey note with a cocoa and balsamic nuance & Sweet, honey, floral, rosy with a slight green nectar fruity body and mouth feel \\ 
    \end{tabular} }
\end{table}

