\setcounter{chapter}{1}
\setcounter{section}{0}
%\chapter{Introduction}
\setlength{\headheight}{12.71342pt}
\addtolength{\topmargin}{-0.71342pt}

\section{Introduction}
Feta cheese holds a Protected Designation of Origin (PDO) status, meaning that authentic feta may only be produced in specific regions of Greece. As part of the PDO, the used milk must come from sheep or from a mixture of sheep and goat milk, where the amount of goat milk cannot exceed 30\% \cite{n01_sandoval2004microstructure}. 

The production steps follow a general cheesemaking process, but with some difference unique to feta. After coagulation, the curd is transferred into perforated moulds and allowed to drain under its own weight, as feta is not a pressed cheese. Once the moisture level has decreased sufficiently, the cheese is dry salted for several days before placed into a brine solution, where it matures for at least two months \cite{n01_sandoval2004microstructure}. Traditionally, the ripening took place in wooden barrels or stainless-steel containers, although modern production has started to incorporate plastic containers. 

Most PDO feta is produced from pasteurized milk and inoculated with standardized starter cultures, typically \textit{Streptococcus thermophilus} and \textit{Lactobacillus delbrueckii} subsp. \textit{bulgaricus} \cite{n01_sandoval2004microstructure}. Although the micro flora can vary among producers, beforementioned species dominate commercial production. 

The feta cheese analysed in this course was purchased from Angelmark and produced in Greece. It contained 23\% fat and had a maximum moisture content of 56\%. According to the nutrition label, the cheese provides 278 kcal per 100 g, consisting of 23 g fat, 16 g protein, and 5.5 g salt. 

Sensory-wise, the cheese had the characteristic salty, tangy and aromatic flavour associated with long brining and a subtle goat-like note was also perceived.  


\section{Phenotypic Clustering}

The heatmap of scaled relative abundance of aroma compounds shows a clear phenotypic separation among the seven cheeses. Feta forms its own distinct cluster, placed on the far left, indicating that its volatile profile differs from all other cheeses (Danbo, Brie, Emmental, Grana, Cheddar, Havarti, and Gouda). This positioning results from a unique set of compounds present at high abundance in Feta cheese, many of which are either absent or present at much lower levels in the other cheeses. 

These compounds show high positive scores for Feta but appear blue/neutral for nearly all other cheeses which indicates that Feta has higher levels of these volatiles than the rest. Compared with harder, longer-ripened cheeses like Grana, Cheddar and Emmentale, Feta shows low levels of ketones, lactones and low aldehyde complexity. 

\subsection{Ethyl acetate}
Ethyl acetate is an ester known to contribute pleasant fruity and floral notes to cheese, helping to create a flavor balance by reducing the sharpness originating from free fatty acids. 

The formation of esters occurs through esterification reactions between short- to medium-chain fatty acids and alcohols, a process catalyzed by lipases and/or esterases produced by yeasts and bacteria in the cheese. In Feta-type cheese, the formation of esters, including ethyl acetate, has been associated with the use of specific auxiliary cultures such as \textit{Lb. paracasei} subsp. \textit{paracasei} and \textit{D. hansenii}, resulting in high levels of ethyl acetate in the cheeses produced with these cultures \cite{n02_bintsis2004study}. 

In addition, many types of esters, including ethyl acetate, have been found in sheep's Feta cheese, suggesting that these compounds have a particular importance for the characteristic flavor of Feta made from sheep's milk. It is likely that the high fat content of sheep's milk enhances the activity of certain lipases and/or esterases involved in ester formation \cite{n02_bintsis2004study}. 

\subsection{3-Methyl-1-butanol}
3-Methyl-1-butanol is classified as an alcohol. This compound is responsible for a pleasant alcoholic floral note in some soft cheeses. It is formed from the amino acid leucine via Strecker degradation \cite{n02_bintsis2004study}. 

High concentrations of 3-methyl-1-butanol have previously been found in Feta cheese. Alcohols such as this can be formed rapidly from aldehydes under the highly reducing conditions present in cheese, or via other metabolic pathways, such as amino acid catabolism \cite{n02_bintsis2004study}. 

\subsection{Hexanoic acid}
Hexanoic acid is a volatile fatty acid and is one of the dominant volatile fatty acids in traditional Feta cheese. It is the most abundant fatty acid in traditional Feta cheese made from raw milk. This high concentration is likely attributed to increased hexanoic acid-specific lipase activity from the wild lactic acid bacteria strains found in Feta cheese. Hexanoic acid contributes to a "goat/barn"-like odor but is often considered a desirable characteristic in mature cheese \cite{n03_tuohy2023sensory}. 


\section{Most Abundant Compounds}
The volatile profile of Feta cheese is characterized by its distinct set of aroma compounds that differentiate from other cheese varieties. The most abundant volatiles include free fatty acids, alcohols, and esters, respectively \cite{a01_kondyli2012effect}. In this section, these  three groups will be described with emphasis on the most abundant individual compounds with respective tables for visualisation. Data has been derived from the study by \citeauthor{a01_kondyli2012effect} \cite{a01_kondyli2012effect}.

\subsection{Free Fatty Acids}
In the study by \citeauthor{a01_kondyli2012effect}, free fatty acids were found to be the most abundant compounds in the volatile fraction of Feta cheese. After 180 days of ripening, the five dominant FFAs were C10:0 (decanoic acid), C2:0 (acetic acid), C16:0 (palmitic acid), C14:0 (myristic acid) and C12:0 (lauric acid), as shown in Table \ref{tab:tab_ffa_180days}. These fatty acids contribute to the characteristic flavour profile of Feta, with C10:0 and C2:0 in particular.

\vspace{0.5em}
It was shown that packaging affected the quantitative distribution of these FFAs. Cheeses that was stored in tin vessels contained significantly (P < 0.05) lower levels of C10:0 compared with those ripened in wooden barrels. The levels of C16:0 and C14:0  was slightly lower, and C12:0 showed only a minor decrease in the tin vessels \textcite{a01_kondyli2012effect}.


\begin{table}[h]
    \centering
    \caption{Concentrations of the five most abundant free fatty acids (FFAs) in Feta cheese stored for 180 days in wooden barrels and tin vessels. Values are given as $\mu$g\,g$^{-1}$ cheese $\pm$ SD.}
    \label{tab:tab_ffa_180days}
    \rowcolors{2}{white}{gray!7}
    \begin{tabular}{ c | c | c }
        \textbf{FFA} & \textbf{Wood Barrels} & \textbf{Tin Vessels} \\
        \hline
        C10:0  & 299.03 $\pm$ 10.13 & 253.92 $\pm$ 20.17 \\
        C2:0  & 284.53 $\pm$ 20.15   & 184.87 $\pm$ 15.12   \\
        C16:0  & 226.27 $\pm$ 4.62  & 207.48 $\pm$ 12.72   \\
        C14:0  & 159.78 $\pm$ 4.97  & 129.20 $\pm$ 15.85   \\
        C12:0 & 138.74 $\pm$ 4.27 & 134.25 $\pm$ 3.57 \\
    \end{tabular}
\end{table}

To get a better understanding of the overall FFA profile, Table \ref{tab:tab_ffa_descriptors} has been compiled, illustrating the top five most abundant FFAs in the feta cheese with sensoric descriptors.

\begin{table}[h]
    \centering
    \caption{Sensoric descriptors of the five most abundant free fatty acids (FFAs) in Feta cheese. Data has been adapted from The Good Scents Company database at their respective minimum thresholds in the database \citeauthor{w01_GoodScentsDatabase} \cite{w01_GoodScentsDatabase}.}
    \label{tab:tab_ffa_descriptors}
    \resizebox{\textwidth}{!}{
    \rowcolors{2}{white}{gray!7}
    \begin{tabular}{ c | p{6cm} | p{6cm} }
        \textbf{FFA} & \textbf{Odour} & \textbf{Flavour} \\
        \hline
        C10:0  & Rancid, Sour, fatty, and citrus & Soapy, waxy, and fruity \\

        C2:0  & Pungent acidic and dairy-like   & Acidic, dairy with a pronounced fruity lift   \\

        C16:0  & Low heavy waxy, with a creamy, candle waxy nuance  & Waxy, creamy fatty, soapy with a crisco like fatty, lard and tallow like mouth feel and a dairy nuanc   \\        
        
        C14:0  & Faint, waxy and fatty with a hint of pineapple and citrus peel  & Waxy, fatty, soapy, creamy, cheesy, with a good mouth feel   \\

        C12:0 & mild fatty & Fatty, waxy \\ 
    \end{tabular} }
\end{table}


\subsection{Alcohols}
The second most abundant group of volatiles in feta cheese is represented by alcohols. At 180 days of ripening the five dominant alcohols were ethanol, butan-2-ol, 3-methylbutan-1-ol, phenylethanol and butan-1-ol \cite{a01_kondyli2012effect}, as shown in Table \ref{tab:tab_alcohols_180days}. Ethanol was present at the highest concentration, reflecting the fermentative activity during brining and storage. Butan-2-ol and 3-methylbutan-1-ol were also present at significantly high levels.

\vspace{0.5em}
Packaging had also an influence on the quantitative profile of the alcohols. Cheeses that were ripened in wooden barrels contained substantially higher levels of ethanol, butan-2-ol, phenylethanol and butan-1-ol compared with cheeses stored in tin vessels, indicating a more active fermentation and microbial pathways in the wooden environment. Conversely, 3-methylbutan-1-ol showed similar concentrations in both packaging types. Despite these differences, the dominant alcohols remained consistent across packaging type \textcite{a01_kondyli2012effect}.


\begin{table}[h]
    \centering
    \caption{Concentrations of the five most abundant alcohols in Feta cheese stored for 180 days in wooden barrels and tin vessels. Values are given as peak area in TIC $\times 10^{5}$ $\pm$ SD.}
    \label{tab:tab_alcohols_180days}
    \rowcolors{2}{white}{gray!7}
    \begin{tabular}{ c | c | c }
        \textbf{Alcohol} & \textbf{Wood Barrels} & \textbf{Tin Vessels} \\
        \hline
        Ethanol             & 54{,}634.4 $\pm$ 3512.30 & 35{,}458.0 $\pm$ 2160.70 \\
        Butan-2-ol          & 20{,}990.60 $\pm$ 1549.10   & 9{,}004.70 $\pm$ 512.30  \\
        3-Methylbutan-1-ol  & 17{,}671.2 $\pm$ 2085.0    & 18{,}641.4 $\pm$ 1999.8   \\
        Phenylethanol       & 8{,}560.7 $\pm$ 343.5    & 1{,}226.7 $\pm$ 229.7   \\
        Butan-1-ol          & 3{,}114.5 $\pm$ 48.5     & 381.1 $\pm$ 103.7   \\
    \end{tabular}
\end{table}

Table \ref{tab:tab_alcohols_descriptors} was compiled to illustrate the top five most abundant alcohols in the feta cheese. Sensoric descriptors have been added for better understanding of their individual contributions to the overall aroma profile of the cheese.

\begin{table}[h]
    \centering
    \caption{Sensoric descriptors of the five most abundant alcohols in Feta cheese. Data has been adapted from The Good Scents Company database at their respective minimum thresholds in the database \citeauthor{w01_GoodScentsDatabase} \cite{w01_GoodScentsDatabase}.}
    \label{tab:tab_alcohols_descriptors}
    \resizebox{\textwidth}{!}{
    \rowcolors{2}{white}{gray!7}
    \begin{tabular}{ c | p{6cm} | p{6cm} }
        \textbf{Alcohol} & \textbf{Odour} & \textbf{Flavour} \\
        \hline
        Ethanol  & Strong alcoholic ethereal, medical & n.d. \\

        Butan-2-ol  & Sweet apricot   & n.d.   \\

        3-Methylbutan-1-ol  & Fusel, alcoholic, pungent, etherial, cognac, fruity, banana and molasses  & Fusel, fermented, fruity, banana, etherial and cognac   \\        
        
        Phenylethanol  & Sweet, floral, fresh and bready with a rosey honey nuance  & Floral, sweet, rosey and bready   \\

        Butan-1-ol & Fusel oil sweet balsam whiskey & Banana fusel \\ 
    \end{tabular} }
\end{table}


\subsection{Esters}
Esters formed the third most abundant group of volatile compounds in Feta cheese at 180 days of ripening. The five dominant esters were ethyl hexanoate, ethyl octanoate, ethyl butanoate, ethyl decanoate and 2-phenylethyl acetate, as shown in Table \ref{tab:tab_esters_180days}. These compounds are primarily formed through esterification between free fatty acids and ethanol or other alcohols \cite{a01_kondyli2012effect}, and contribute with fruity, floral and sweet notes that balance the sharper acidic components of the cheese.

\vspace{0.5 em}
A clear effect of packaging was observed on the quantitative levels of esters. Cheeses ripened in wooden barrels contained significantly higher concentrations of all five esters compared with cheeses stored in tin vessels, indicating that the wooden environment favoured ester synthesis. The differences were especially pronounced for ethyl hexanoate, ethyl octanoate and ethyl butanoate \textcite{a01_kondyli2012effect}.

\begin{table}[h]
    \centering
    \caption{Concentrations of the five most abundant esters in Feta cheese stored for 180 days in wooden barrels and tin vessels. Values are given as peak area in TIC $\times 10^{5}$ $\pm$ SD.}
    \label{tab:tab_esters_180days}
    \rowcolors{2}{white}{gray!7}
    \begin{tabular}{ c | c | c }
        \textbf{Ester} & \textbf{Wood Barrels} & \textbf{Tin Vessels} \\
        \hline
        Ethyl hexanoate        & 17{,}310.2 $\pm$ 947.1   & 1{,}437.4 $\pm$ 157.8 \\
        Ethyl octanoate        & 16{,}930.3 $\pm$ 1001.3 & 3{,}198.7 $\pm$ 394.1 \\
        Ethyl butanoate        & 13{,}189.6 $\pm$ 844.7   & 1{,}561.8 $\pm$ 87.2 \\
        Ethyl decanoate        & 9{,}861.6 $\pm$ 449.8    & 733.8 $\pm$ 44.1 \\
        2-Phenylethyl acetate  & 7{,}862.3 $\pm$ 298.9    & 821.5 $\pm$ 104.7 \\
    \end{tabular}
\end{table}

A compilation of the top five most abundant esters in Feta cheese is shown in Table \ref{tab:tab_esters_descriptors}, along with their sensoric descriptors to illustrate their individual contributions to the overall aroma profile of the cheese.

\begin{table}[h]
    \centering
    \caption{Sensoric descriptors of the five most abundant esters in Feta cheese. Data has been adapted from The Good Scents Company database at their respective minimum thresholds in the database \citeauthor{w01_GoodScentsDatabase} \cite{w01_GoodScentsDatabase}.}
    \label{tab:tab_esters_descriptors}
    \resizebox{\textwidth}{!}{
    \rowcolors{2}{white}{gray!7}
    \begin{tabular}{ c | p{6cm} | p{6cm} }
        \textbf{Ester} & \textbf{Odour} & \textbf{Flavour} \\
        \hline
        Ethyl hexanoate    & Sweet, fruity, pineapple, waxy, fatty and estry with a green banana nuance & Sweet, pineapple, fruity, waxy and banana with a green, estry nuance \\

        Ethyl octanoate        & Waxy, sweet, musty, pineapple and fruity with a creamy, dairy nuance   & Sweet, waxy, fruity and pineapple with creamy, fatty, mushroom and cognac notes   \\

        Ethyl butanoate  & Sweet, fruity, tutti frutti, lifting and diffusive  & Fruity, sweet, tutti frutti, apple, fresh and lifting, ethereal   \\        
        
        Ethyl decanoate  & Sweet, waxy, fruity, apple  & Waxy, fruity, sweet apple   \\

        2-Phenylethyl acetate  & Sweet, honey, floral rosy, with a slight yeasty honey note with a cocoa and balsamic nuance & Sweet, honey, floral, rosy with a slight green nectar fruity body and mouth feel \\ 
    \end{tabular} }
\end{table}


\subsection{}


\section{Compare the volatile composition of your cheese with the tasting you did in the beginning of this course.}
Based on question 3 \& 4 the compounds with the highest concentrations in the feta cheese was ethyl acetate, 1-butanol-3-methyl, and hexanoic acid, all of which contribute to the final flavour in the cheese. In the initial tasting performed earlier in the course, the feta was described as salty, tangy, slightly sweet and sour, with a subtle goaty flavour and an overall mild balanced aroma. These sensory observations align with the identified volatile compounds as described below. 

\paragraph{Ethyl acetate}
is a common ester in fermented dairy products and is typically associated with fruity and sweet flavours. Although feta is not known for complex fruity notes, low to moderate levels of ethyl acetate contribute to the fresh, mildly aromatic flavours in brined cheeses aligned with the observation made during the cheese tasting

\paragraph{1-butanol -3-methyl}
contributed with malty, fusel-like, tangy and fermented notes that can vary in pungency depending on the concentration. In the feta cheese the compound contributes to the tangy notes and balancing the flavour profile which was also identified during the cheese tasting 

\paragraph{Hexanoic acid, ethyl ester} is an ester formed from the medium-chain fatty acid hexanoic acid which is associated with the goaty, tangy flavour typical of sheep and goat milk cheeses. 
Hexanoic acid itself has a this pungent goaty aroma however esterification transforms the compound into a fruitier and more balanced compound. Esterification results in a milder expression of the goaty aroma rather than an intense or pungent flavour from hexanoic acid. The presence of this ester aligns well with the sensory evaluation of the cheese, which was described as having a subtle but noticeable goaty flavour consistent with traditional feta cheese.

Overall, the dominant volatiles detected in the heat map (question 3\&4) align with the sensory evaluation and the expected characteristics of authentic Greek feta cheese. 


\section{Conclusion}
The Three dominant aroma compounds identified in the heat map \& aroma sheet (Ethyl acetate, 3-methyl-1-butanol and Hexanoic acid) has been further investigated and are considered classical flavour compounds in feta cheese. The three compounds contribute to the fruity sweetness, acidity earthy and goaty flavour that aligns with cheese tasting in the beginning of the course. 


